%%%%%%%%%%%%%%%%%%%%%%%%%%%%%%%%%%%%%%%%%
% Dictionary
% LaTeX Template
% Version 1.0 (20/12/14)
%
% This template has been downloaded from:
% http://www.LaTeXTemplates.com
%
% Original author:
% Vel (vel@latextemplates.com) inspired by a template by Marc Lavaud
%
% License:
% CC BY-NC-SA 3.0 (http://creativecommons.org/licenses/by-nc-sa/3.0/)
%
%%%%%%%%%%%%%%%%%%%%%%%%%%%%%%%%%%%%%%%%%

%----------------------------------------------------------------------------------------
%	PACKAGES AND OTHER DOCUMENT CONFIGURATIONS
%----------------------------------------------------------------------------------------

\documentclass[10pt,a4paper,twoside]{article} % font size, A4 paper and two-sided margins

\usepackage[top=3.5cm,bottom=3.5cm,left=3.7cm,right=4.7cm,columnsep=30pt]{geometry} % Document margins and spacings

\usepackage[utf8]{inputenc} % Required for inputting international characters
\usepackage[T1]{fontenc} % Output font encoding for international characters
\usepackage{textcomp} % Provides additional symbols
\usepackage{tipa} % Provides additional symbols for phonetic characters

\usepackage{palatino} % Font

\usepackage{microtype} % Improves spacing

\usepackage{graphicx}

\usepackage{multicol} % Required for splitting text into multiple columns

\usepackage[bf,center]{titlesec} % Required for modifying section titles - bold, sans-serif, centered

\usepackage{fancyhdr} % Required for modifying headers and footers
\fancyhead[L]{\rightmark} % Top left header
\fancyhead[R]{\leftmark} % Top right header
\renewcommand{\headrulewidth}{1.4pt} % Rule under the header
\fancyfoot[L]{\footnotesize{© Mahîr Dogan 2025}} % Bottom left footer
\fancyfoot[C]{\textbf{\textsf{\thepage}}} % Bottom center footer
\renewcommand{\footrulewidth}{1.4pt} % Rule under the footer
\pagestyle{fancy} % Use the custom headers and footers throughout the document

\usepackage{ifthen} % Required for conditional statements
\usepackage{xifthen} % Needed for the \isempty command

\newcommand{\entry}[6]{
  \markboth{#1}{#1} % Print the first word on the page in the top left header and the last word in the top right header
  \textbf{#1} % Kurdish entry in bold
    \ifthenelse{\isempty{#4}}{}{ (\textit{#4})} % Gender, if provided, in italics and parentheses
  \ifthenelse{\isempty{#2}}{}{({\textasciitilde} #2)} % Variants, if provided, in (~ variant) format
  \ \textsc{#3} % Part of speech in smallcaps
    $\bullet$ % Bullet point after the entry
  {#6} % German translation
  \ifthenelse{\isempty{#5}}{}{\par\noindent --- {Std.:} #5} % Standard, if provided, starts on a new line
}

\newcommand{\gentry}[5]{%
  \markboth{#1}{#1} % Header markers
  \textbf{#1} % German entry in bold
  \ \textsc{#2} % Part of speech in smallcaps
  $\bullet$ % Bullet point
  #3 % Kurdish translation
  \ifthenelse{\isempty{#4}}{}{ (\textit{#4})} % Gender, if provided
  \ifthenelse{\isempty{#5}}{}{\par\noindent --- Std.: #5} % Standard, if provided
}

%----------------------------------------------------------------------------------------

\begin{document}

%----------------------------------------------------------------------------------------
%	Title
%----------------------------------------------------------------------------------------

\begin{titlepage}
   \begin{center}
    \vspace*{1cm}

       
        {\Huge\textbf{Kirmanckî (Zazakî) Mini-Wörterbuch}}

        \vspace{0.5cm}
        {\Large Dêrsimer Dialekt}

        \vspace{0.8cm}
           
        \textbf{\copyright{} Mahîr Dogan}\\
        2025
            
        \vspace{1.5cm}

        \vfill
            
       Begleitmaterial für den Anfängerkurs\\
       in Kirmanckî (Zazakî)\\
       \vspace{0.8cm}
       Version 1.0     
       

   \end{center}
\end{titlepage}

%----------------------------------------------------------------------------------------
%	INTRO
%----------------------------------------------------------------------------------------

\section*{Vokabeln zum Kirmanckî-Kurs}

Dieses Mini-Wörterbuch dient als Nachschlagewerk und beinhaltet zum einen alle Vokabeln, die im Anfängerkurs gebraucht werden, sowie zusätzliche Wörter, die im Alltag nützlich sein werden. Die Einträge sind alphabetisch gemäß dem Dêrsimer Dialekt des Kirmanckî (Zazakî) aufgelistet. Das Wörterbuch beinhaltet darüber hinaus Einträge aus dem Standardkirmanckî, welche in der Schriftsprache und in diesem Kurs genutzt werden.\\
\\
Ein Eintrag ist wie folgt aufgebaut:
\\
\begin{figure}[h]
    \centering
    \includegraphics[width=1\linewidth]{Wörterbuch.png}
\end{figure}
\\
Viele Wortklassen des Kirmanckî können wechseln, so kann ein Adjektiv oder Verb auch als Nomen genutzt oder ein Nomen in ein Adjektiv umgewandelt werden. Diese Variation wird hier nicht berücksichtigt. Sollte ein Wort beispielsweise als Nomen (\textsc{nom}) aufgelistet sein, so heißt das nicht zwangsläufig, dass die Nutzung darauf beschränkt ist.
\subsection*{\textbf{Abkürzungen}}

\begin{table}[h]
    \centering
    \begin{tabular}{l l}
        \textsc{adj} & Adjektiv \\ 
        \textsc{adv} & Adverb \\
        \textsc{dem} & Demonstrativpronomen \\
        \textsc{frg} & Fragewort \\
        \textsc{intrj} & Interjektion \\
        \textsc{konj} & Konjunktion \\
        \textsc{nom} & Nomen \\
        \textsc{post} & Postposition \\
        \textsc{präp} & Präposition \\
        \textsc{pron} & Pronomen \\
        \textsc{ptkl} & Partikel \\
        \textsc{verb} & Verb \\
        \textsc{vok} & Vokativ \\
        \textsc{zahl} & Zahl\\
        \textsc{zirk} & Zirkumposition \\
         & \\
        \textit{-e, -îye} & \textit{Femininendung}\\
        \textit{n} & nêrî (maskulin)\\
        \textit{m} & makî (feminin)\\
        \textit{zh} & zafhûmar (Plural)\\
    \end{tabular}
\end{table}
\clearpage

%----------------------------------------------------------------------------------------
%	Kirmanckî - Deutsch
%----------------------------------------------------------------------------------------
\begin{titlepage}
    \begin{center}
        \vspace*{\stretch{1}}
        {\Huge\textbf{Kirmanckî - Deutsch}} 
        \vspace*{\stretch{1}}
    \end{center}
\end{titlepage}

%----------------------------------------------------------------------------------------
%	SECTION A
%----------------------------------------------------------------------------------------

\section*{A}

\begin{multicols}{2}

\entry{a\textsuperscript{1}}{}{pron}{m}{}{sie (Singular)}

\entry{a\textsuperscript{2}}{}{dem}{m}{}{jene; die/das dort (Singular)}

\entry{a\textsuperscript{3}}{ya}{post}{}{}{mit; anhand}

\entry{acêba}{}{}{}{}{$\rightarrow$ \textbf{ecêba}}

\entry{adare}{}{nom}{m}{}{März $\rightarrow$ \textbf{marte}}

\entry{adir}{}{nom}{n}{}{Feuer}

\entry{adirge}{}{nom}{n}{}{Feuerzeug}

\entry{aferîn}{}{intrj}{}{}{gut gemacht}

\entry{Afrîka}{}{nom}{m}{}{Afrika}

\entry{afrîkayiz}{afrîkayic}{nom}{-e}{afrîkayij}{Afrikaner/in}

\entry{aha}{}{intrj}{}{}{\textit{Ausruf für Vorsicht, Verwunderung oder Erleichterung}}

\entry{Akrê}{}{nom}{m}{}{\textit{Stadt in Kurdistan}}

\entry{alaqe}{}{nom}{m}{}{$\rightarrow$ \textbf{eleqe}}

\entry{alaqedar}{}{adj}{}{}{$\rightarrow$ \textbf{eleqedar}}

\entry{alawî}{}{nom}{-ye}{}{$\rightarrow$ \textbf{elewî}}

\entry{albaz}{alboz}{nom}{-e}{embaz}{\textsuperscript{1}Freund/in, Kumpel/in $\rightarrow$ \textbf{heval} \textsuperscript{2}Gefährte/in, Begleiter/in \par\noindent\textit{Xizir albazê to bo} - Möge Xizir dein Begleiter sein / an deiner Seite stehen}

\entry{album}{}{nom}{n}{}{Album}

\entry{alfabe}{}{nom}{m}{}{Alphabet}

\entry{Alî}{Elî}{nom}{n}{}{\textit{männl. Vorname}}

\entry{Alîşêr}{Elîşêr}{nom}{n}{}{\textit{männl. Vorname}}

\entry{alman}{}{nom}{-e}{}{Deutsche/r}

\entry{almankî}{}{nom}{m}{}{Deutsch, die deutsche Sprache}

\entry{Almanya}{}{nom}{m}{}{Deutschland}

\entry{alo}{}{}{}{}{Hallo \textit{(Am Telefon)}}

\entry{alunça}{}{nom}{m}{}{Mirabelle, Braunelle, Pflaume $\rightarrow$ \textbf{hêrige}}

\entry{ama}{}{konj}{}{}{$\rightarrow$ \textbf{hama}}

\entry{Amed}{}{nom}{n}{}{$\rightarrow$ \textbf{Dîyarbekir}}

\entry{Amerîka}{Hemilka}{nom}{m}{}{Amerika}

\entry{ameyene}{amayene}{verb}{}{}{kommen, ankommen \par\noindent\textit{Bê!} - Komm!}

\entry{amike}{ame}{nom}{-e}{emike, eme}{Tante väterlicherseits}

\entry{amiza}{lazê amike, lacê amike}{nom}{n}{emiza, lajê emike}{Cousin, Sohn der Tante väterlicherseits}

\entry{amnan}{amnon, amlon}{nom}{n}{hamnan}{Sommer}

\entry{amnanî}{amnonî, amlonî}{adv}{}{hamnanî}{im Sommer, in der Sommerzeit}

\entry{amnano peyên}{amnono peyên, amlono peyên}{nom}{n}{}{August $\rightarrow$ \textbf{a\"{x}ustose}}

\entry{amnano virên}{amnono virên, amlono virên}{nom}{n}{}{Juni $\rightarrow$ \textbf{hazîrane}}

\entry{amnano wertên}{amnono wertên, amlono wertên}{nom}{n}{}{Juli $\rightarrow$ \textbf{temmuze}}

\entry{antene}{ontene}{verb}{}{}{ziehen}

\entry{ap}{}{nom}{n}{}{Onkel väterlicherseits $\rightarrow$ \textbf{ded}}

\entry{Apil}{}{nom}{n}{}{\textit{männl. Vorname}}

\entry{ara}{}{nom}{m}{arayî}{Frühstück}

\entry{arab}{areb}{nom}{-e}{ereb}{Araber/in}

\entry{arabkî}{arebkî}{nom}{m}{erebkî}{Arabisch}

\entry{ardene}{}{verb}{}{}{(her)bringen, (her)holen, herbeiholen, heranbringen, einbringen}

\entry{ardim}{}{nom}{n}{}{Hilfe, Mithilfe, Aushilfe}

\entry{ardî}{}{nom}{zh}{}{Mehl}

\entry{arebe}{araba, \textasciitilde \textit{m}}{nom}{n}{erebe}{Auto; Wagen}

\entry{armanc}{amaç}{nom}{n}{}{Zweck, Ziel, Absicht}

\entry{asayene}{osayene}{verb}{}{}{(er)scheinen, aussehen, sichtbar werden, in Erscheinung treten}

\entry{asin}{osin}{nom}{n}{}{Eisen}

\entry{asme}{}{nom}{m}{aşme}{\textsuperscript{1}Mond \textsuperscript{2}Monat}

\entry{asmên}{}{nom}{n}{}{Himmel}

\entry{aspar}{aspor, ospor}{nom}{-e}{espar}{Reiter/in}

\entry{astare}{}{nom}{n}{estare}{Stern}

\entry{astor}{ostor}{nom}{-e}{estor}{Pferd}

\entry{aşt}{haşt}{adj}{-e}{}{versöhnt, vertragen}

\entry{aştîye}{haştîye}{nom}{m}{}{Frieden, Harmonie}

\entry{astronot}{}{nom}{-e}{}{Astronaut/in, Raumfahrer/in}

\entry{aşîre}{eşîre}{nom}{m}{eşîre}{Stamm, Clan, Sippe}

\entry{aver}{}{adv}{}{}{$\rightarrow$ \textbf{raver}}

\entry{avê}{}{adv}{}{}{$\rightarrow$ \textbf{raver}}

\entry{avûkat}{}{nom}{-e}{}{Anwalt/Anwältin, Rechts\-anwalt / Rechts\-anwältin}

\entry{awe}{awke, aw\"{x}e}{nom}{m}{}{Wasser; Gewässer}

\entry{awe û werd}{}{nom}{n}{}{Essen und Getränke, Speis und Trank; Speisen $\rightarrow$ \textbf{nan û awe}}

\entry{awrês}{arwes, awreş}{nom}{-e}{hargûş}{Hase}

\entry{axirî}{}{adv}{}{}{endlich, schließlich, zu guter Letzt}

\entry{axure}{}{nom}{m}{}{Stall}

\entry{a\"{x}ustose}{}{nom}{m}{tebaxe}{August}

\entry{a\"{x}û}{}{nom}{n}{}{Gift, Giftstoff}

\entry{aye}{}{pron}{}{}{\textit{2. Fall von}$\rightarrow$ \textbf{a\textsuperscript{1}}}

\entry{Azad}{}{nom}{n}{}{\textit{männl. Vorname}}

\entry{azad}{}{adj}{-e}{}{frei, unabhängig, eigenständig}

\entry{azadênî}{}{nom}{m}{}{$\rightarrow$ \textbf{azadîye}}

\entry{azadîye}{}{nom}{m}{}{Freiheit, Unabhängigkeit, Eigenständigkeit}

\entry{azeb}{}{adj}{-e}{}{\textsuperscript{1}ledig, unverheiratet \textsuperscript{2}Jungfrau}

\entry{azne}{}{nom}{n}{ajne}{Schwimmen; Schwimmsport}

\entry{azne kerdene}{}{verb}{n}{ajne kerdene}{schwimmen}

\end{multicols}

%----------------------------------------------------------------------------------------
%	SECTION B
%----------------------------------------------------------------------------------------

\section*{B}

\begin{multicols}{2}

\entry{babete}{}{nom}{m}{}{\textsuperscript{1}Thema, Betreff \textsuperscript{2}Angelegenheit, Sachverhalt}

\entry{bacenax}{}{nom}{n}{}{Schwippschwager, Mann der der Schwester der Ehefrau $\rightarrow$ \textbf{hevalzava}}

\entry{bacike}{}{nom}{m}{}{Schwägerin \textit{(Schwester des Ehemanns)} $\rightarrow$ \textbf{zeyîye}}

\entry{badê}{}{adv}{}{}{danach, später, nachher}

\entry{badêna}{}{adv}{}{}{danach, später, nachher $\rightarrow$ \textbf{badê}}

\entry{balduze}{}{nom}{m}{}{Schwägerin, Schwester der Ehefrau}

\entry{balîsna}{}{nom}{m}{balîşna}{Kissen, Kopfkissen; Polster}

\entry{balkêş}{}{adj}{-e}{}{interessant, achtungswert, beeindruckend, erstaunlich $\rightarrow$ \textbf{enteresan}}

\entry{balon}{}{nom}{n}{}{Luftballon, Ballon}

\entry{ban}{bon}{nom}{n}{}{Haus, Gebäude}

\entry{banyo}{}{nom}{n}{}{\textsuperscript{1}Bad \textsuperscript{2}Badezimmer}

\entry{baqil}{}{adj}{-e}{}{$\rightarrow$ \textbf{biaqil}}

\entry{bar}{}{nom}{n}{}{\textsuperscript{1}Last, Ladung, Beladung; Ballast \textsuperscript{2}Bürde, Belastung, Beschwernis \textsuperscript{3}Bar}

\entry{Baran}{}{nom}{n}{}{\textit{männl. Vorname}}

\entry{barber}{}{nom}{-e}{}{Lieferant/in; Kurier/in; Lastenträger $\rightarrow$ \textbf{hemal}}

\entry{bardaxe}{}{nom}{m}{}{Becher, Trinkglas, Glas}

\entry{barî}{}{adj}{-ye}{}{dünn}

\entry{barzingan}{}{nom}{-e}{}{$\rightarrow$ \textbf{bazirgan}}

\entry{bav û \.{k}al}{}{nom}{n}{}{Vorfahren, Ahnen}

\entry{bawo}{}{vok}{}{}{\textsuperscript{1}\textit{Anredefall von} $\rightarrow$ \textbf{pî} \textsuperscript{2}\textsc{intrj} \textit{Ausdruf der Wehklage}}

\entry{baxçe}{}{nom}{n}{}{Garten; Gärtnerei}

\entry{baxçewan}{}{nom}{-e}{}{Gärtner/in}

\entry{ba\"{x}}{}{nom}{n}{bax}{\textsuperscript{1}Gemüsegarten \textsuperscript{2}Weinberg}

\entry{bayram}{}{nom}{n}{roşan}{\textsuperscript{1}Feiertag, Festtag \textsuperscript{2}Feier, Fest}

\entry{bazar}{}{nom}{n}{}{\textsuperscript{1}Sonntag \textsuperscript{2}Basar, Markt, Marktplatz; Wochenmarkt \textsuperscript{3}Handel, Markt}

\entry{bazirgan}{barzingan}{nom}{-e}{}{Händler/in, Kaufmann/Kauffrau}

\entry{be}{bi, ebe, ebi}{präp}{}{bi}{mit; via, mittels}

\entry{be ... ra}{(e)ve ... ra, (e)bi ... (y)a}{zirk}{}{bi ... ra}{\textsuperscript{1}mit \textsuperscript{2}und}

\entry{bekmez}{}{nom}{n}{}{Maulbeersirup}

\entry{beledîya}{beledîye}{nom}{m}{beledîya, şaredarîye}{\textsuperscript{1}Stadtverwaltung, Gemeindeverwaltung, Stadtrat \textsuperscript{2}Gemeinde, Kommune}

\entry{belê}{}{ptkl}{}{}{ja, jawohl $\rightarrow$ \textbf{heya}}

\entry{belkî}{}{adv}{}{}{$\rightarrow$ \textbf{beno ke}}

\entry{beno ke}{}{adv}{}{}{vielleicht, möglicherweise \par\noindent\textit{Beno ke ewro derse çîn a} - Vielleicht findet heute kein Unterricht statt}

\entry{beran}{}{nom}{n}{}{Widder, Schafbock}

\entry{berber}{}{nom}{-e}{}{Friseur/in}

\entry{berdene}{}{verb}{}{}{\textsuperscript{1}(weg)bringen, fortführen, fortbringen \textsuperscript{2}transportieren, tragen}

\entry{bereqîyayene}{}{verb}{}{beriqîyayene}{scheinen, glänzen, funkeln}

\entry{Berfîne}{}{nom}{m}{}{\textit{weibl. Vorname}}

\entry{berxûdar}{}{adj}{-e}{}{glücklich, gesegnet} \par\noindent\textit{Berxûdar be! - Gesegnet sei's du!}

\entry{berz}{}{adj}{-e}{}{hoch}

\entry{berzîye}{berjîye}{nom}{m}{}{Höhe, Anhöhe; Hoheit}

\entry{bes}{}{adj}{-e}{}{genug, genügend, ausreichend, hinlänglich}

\entry{bes bîyene}{}{verb}{}{}{reichen, ausreichen, hinreichen, langen, genügen \par\noindent\textit{Bes o.} - Es reicht.}

\entry{besekerdene}{}{verb}{}{}{können, fähig sein, zu tun vermögen $\rightarrow$ \textbf{şîkîyayene\textsuperscript{1}} \par\noindent\textit{Mi besenêkerd bikerî.} - Ich konnte es nicht machen.}

\entry{bexêr}{bixêr, bexeyr}{adj}{-e}{bixeyr}{segensreich, Segen bringend, Heil bringend; $\rightarrow$ \textbf{xêr} \par\noindent\textit{Ma bexêr dî!} - Hallo!}

\entry{bext}{}{nom}{n}{}{\textsuperscript{1}Schicksal, Los, Bestimmung \textsuperscript{2}Glück}

\entry{bê}{}{präp}{}{}{ohne, -los \par\noindent\textit{Bê mi meso} - Geh nicht ohne mich}

\entry{bêbext}{}{adj}{-e}{}{\textsuperscript{1}unglücklich, Pechvogel \textsuperscript{2}hinterlistig, hinterhältig, verräterisch, heimtückisch $\rightarrow$ \textbf{xayîn}}

\entry{bêçike}{}{nom}{m}{}{Finger}

\entry{bêkar}{}{adj}{-e}{}{arbeitslos, erwerbslos}

\entry{bêpere}{}{adj}{-e}{}{umsonst, gratis, kostenlos}

\entry{bicêk}{bijêk}{nom}{-e}{bizêk}{Zicklein}

\entry{bijêk}{bicêk}{nom}{-e}{bizêk}{Zicklein}

\entry{biko}{}{vok}{}{}{\textit{Anredfall für junge Männer oder Buben}}

\entry{bilusk}{}{nom}{n}{}{Blitz}

\entry{biaqil}{baqil}{adj}{-e}{}{\textsuperscript{1}klug, clever, intelligent \textsuperscript{2}artig, gehorsam. gescheit}

\entry{bimbarek}{}{adj}{-e}{}{\textsuperscript{1}heilig, gesegnet \textsuperscript{2}froh \par\noindent\textit{Roşanê sima bimbarek bo!} - (Euch einen) Frohen Feiertag!}

\entry{bin}{}{nom}{n}{}{Unten; Unter-}

\entry{bin de}{}{adv, post}{}{}{darunter, drunter, unterhalb, unter $\rightarrow$ \textbf{binê ... de}}

\entry{bin kotene}{}{verb}{}{}{unterliegen, verlieren}

\entry{binê ... de}{}{zirk}{}{}{darunter, drunter, unterhalb, unter $\rightarrow$ \textbf{bin de}}

\entry{binge}{}{nom}{n}{}{Grund, Grundlage, Basis, Fundament}

\entry{bira}{}{nom}{n}{}{Bruder}

\entry{birak}{}{nom}{,-e}{}{Liebhaber/in; Affäre}

\entry{birayênî}{}{nom}{m}{}{$\rightarrow$ \textbf{birayîye}}

\entry{birayîye}{}{nom}{m}{}{Brüderlichkeit, Brüderschaft}

\entry{biraza}{birarza}{nom}{n}{}{Neffe \textit{(Sohn des Bruders)}}

\entry{birazaye}{birarzaye}{nom}{m}{}{Nichte \textit{(Tochter des Bruders)}}

\entry{birinc}{}{nom}{n}{}{Reis}

\entry{birrak}{}{nom}{n}{}{$\rightarrow$ \textbf{birrek}}

\entry{birrek}{birrak}{nom}{n}{}{Säge}

\entry{bişî}{}{nom}{m}{}{\textit{trad. Pfannkuchengebäck}}

\entry{bitaybetî}{}{adv}{}{}{insbesondere, im Besonderen, vorzugsweise}

\entry{bitir pêrar}{}{adv}{}{}{vorvorletztes Jahr, vor drei Jahren}

\entry{biza \.{k}irre}{}{nom}{m}{}{Bergziege $\rightarrow$ \textbf{pezkovî}}

\entry{bize}{}{nom}{m}{}{Ziege}

\entry{bîn}{}{adj}{-e}{}{andere/r/s; sonstige/r/s}

\entry{bîna}{}{nom}{m}{}{Bauwerk, Baulichkeit; Gebäude $\rightarrow$ \textbf{ban}}

\entry{bîra}{}{nom}{m}{}{Bier}

\entry{bîsîklete}{}{nom}{m}{}{Fahrrad}

\entry{bîyene\textsuperscript{1}}{}{verb}{}{}{sein; existieren}

\entry{bîyene\textsuperscript{2}}{}{verb}{}{}{\textsuperscript{1}werden \textsuperscript{2}geschehen \par\noindent\textit{Beno nêbeno} - Es war einmal}

\entry{burî}{}{nom}{zh}{birûyî}{Augenbraue(n)}

\entry{buro}{}{nom}{n}{}{\textsuperscript{1}Büro, Dienststelle \textsuperscript{2}Kanzlei, Praxis $\rightarrow$ \textbf{ofîs}}

\entry{bom}{}{adj}{-e}{}{\textsuperscript{1}dumm, blöd \textsuperscript{2}närrisch, verrückt, irre}

\entry{bon}{}{nom}{n}{}{$\rightarrow$ \textbf{ban}}

\entry{bot}{}{nom}{n}{}{Stiefel}

\entry{bote}{}{nom}{m}{}{Boot}

\entry{bostan}{}{nom}{n}{}{Gemüsegarten, Kräutergarten}

\entry{bover}{}{nom}{n}{}{Gegenseite, gegenüberliegende Seite, Gegenufer}

\entry{boye}{}{nom}{m}{}{Geruch, Duft}

\entry{bravo}{}{intrj}{}{}{Bravo}

\end{multicols}

%----------------------------------------------------------------------------------------
%	SECTION C
%----------------------------------------------------------------------------------------

\section*{C}

\begin{multicols}{2}

\entry{ca}{}{nom}{n}{}{\textsuperscript{1}Platz, Ort, Stelle, Position \textsuperscript{2}Raum, (freier) Platz \textsuperscript{3}Gegend}

\entry{camord}{}{nom}{n}{camêrd}{\textsuperscript{1}Herr, Mann \textsuperscript{2}Edelmann, Recke, Gentlemann \textsuperscript{3}Held}

\entry{camusqirane}{}{nom}{m}{}{$\rightarrow$ \textbf{payîze\textsuperscript{1}}}

\entry{can}{}{nom}{n}{}{\textsuperscript{1}Seele, Geist \textsuperscript{2}Leben}

\entry{candar}{}{adj}{-e}{}{\textsuperscript{1}beseelt, belebt \textsuperscript{2}lebendig, animiert}

\entry{cayê}{}{adv}{}{}{\textsuperscript{1}irgendwo; irgendein Ort \textsuperscript{2}\textit{(verneint)} nirgendwo, gar kein Ort}

\entry{cem}{}{nom}{n}{}{Cem \textit{(alevitische Zeremonie)}}

\entry{cem girêdayene}{}{verb}{}{}{$\rightarrow$ \textbf{Cem} abhalten}

\entry{cemat}{}{nom}{n}{}{Gemeinde, Gesellschaft, Community}

\entry{cemed}{}{nom}{n}{}{Eis}

\entry{cenet}{}{nom}{n}{}{Paradies}

\entry{ceng}{}{nom}{n}{}{Krieg, Gefecht \par\noindent\textit{Cengê Cîhanî yo I.} - der 1. Weltkrieg}

\entry{cerebnayene}{}{verb}{}{ceribnayene}{versuchen, probieren, testen}

\entry{ceza}{}{nom}{m}{}{Strafe, Buße, Bestrafung}

\entry{cêb}{}{nom}{n}{cêbe}{Tasche (am Kleidungsstück)}

\entry{cênî}{cînî}{nom}{m}{cinî}{Ehefrau, Gattin}

\entry{cênî û mêrde}{cînî û mêrde}{nom}{zh}{cinî û mêrde}{Ehepaar, Eheleute, Ehegatten; Ehemann und Ehefrau}

\entry{cênîke}{cînîke}{nom}{m}{cinîke}{Frau}

\entry{cêr}{}{nom}{n}{}{\textsuperscript{1}das Untere, unten \textsuperscript{2}runter, hinunter, hinab \par\noindent\textit{Cêr ro vengê yeno} - Eine Stimme kommt von unten}

\entry{cêrayene}{}{verb}{}{gêrayene}{\textsuperscript{2}kreisen, umlaufen \textsuperscript{2}spazieren, umhergehen, herumgehen, herumspazieren \textsuperscript{2}bereisen, begehen, streifen \textsuperscript{3}suchen $\rightarrow$ \textbf{fetelîyayene}}

\entry{cêrîye}{}{nom}{m}{}{Schwägerin \textit{(Die Frau des Schwagers/Bruders des Ehemannes)}}

\entry{cêrnus}{}{nom}{n}{}{Untertitel}

\entry{ci}{}{pron}{}{}{\textit{Pronomen der 3. Person im 2. Fall}}

\entry{cigêrayîş}{}{nom}{n}{}{\textsuperscript{1}Studie, Untersuchung, Forschung \textsuperscript{2}Suche, Erforschung}

\entry{cile}{}{nom}{m}{}{Bett}

\entry{cite}{}{nom}{m}{}{Paar}

\entry{citkar}{}{nom}{-e}{}{Landwirt, Bauer; Feldarbeiter}

\entry{cizdan}{}{nom}{n}{}{Geldbeutel, Portemonnaie}

\entry{cîgere}{}{nom}{m}{}{Leber}

\entry{cîld}{}{nom}{n}{}{\textsuperscript{1}Haut \textsuperscript{2}Band, Einband}

\entry{cînî}{nom}{}{}{}{$\rightarrow$ \textbf{cênî}}

\entry{cînîke}{nom}{}{}{}{$\rightarrow$ \textbf{cênîke}}

\entry{cîran}{}{nom}{-e}{cîran, embiryan}{Nachbar/in}

\entry{cîya}{}{adv}{}{}{\textsuperscript{1}getrennt, auseinander, gesondert \textsuperscript{2}verschieden, andersartig}

\entry{coka}{}{konj}{}{}{darum, deswegen, daher, deshalb}

\entry{cor}{}{nom}{n}{}{\textsuperscript{1}der Obere, oben \textsuperscript{2}hoch, rauf, hinauf}

\entry{cuye}{}{nom}{m}{cu}{Leben $\rightarrow$ \textbf{heyat}}

\entry{cün}{}{nom}{n}{cuwen}{Dreschplatz, Tenne}

\end{multicols}

%----------------------------------------------------------------------------------------
%	SECTION Ç
%----------------------------------------------------------------------------------------

\section*{Ç}

\begin{multicols}{2}

\entry{çakêt}{}{nom}{n}{}{Jacke; Sakko}

\entry{çakuç}{kakuç}{nom}{n}{}{Hammer}

\entry{çamure}{}{nom}{m}{}{Schlamm}

\entry{çand}{}{frg}{}{}{$\rightarrow$ \textbf{çend}}

\entry{çante}{}{nom}{n}{}{Tasche, Handtasche}

\entry{çar}{çor}{zahl}{}{}{vier}

\entry{çarnayene}{}{verb}{}{}{\textsuperscript{1}wenden, drehen \textsuperscript{2}übersetzen \textsuperscript{3}\textit{(am Telefon etc.)} eine Nummer wählen}

\entry{çarseme}{çorseme}{nom}{n}{çarşeme}{Mittwoch}

\entry{çawres}{}{zahl}{}{}{$\rightarrow$ \textbf{çewres}}

\entry{çax}{}{nom}{n}{}{\textsuperscript{1}Zeit \textsuperscript{2}Ära, Periode}

\entry{çay}{}{nom}{n}{}{Tee}

\entry{çayê}{ça}{frg}{}{}{warum, weshalb, wieso}

\entry{çayxane}{}{nom}{n}{}{Teehaus}

\entry{çekuye}{}{nom}{m}{}{Wort}

\entry{çele}{}{nom}{n}{}{Januar}

\entry{çend}{çand, çond}{frg}{}{çend}{\textsuperscript{1}wie viel \textit{(zählbar)} \textsuperscript{2}mehrere, einige \par\noindent\textit{Çend teneyî est ê?} - Wie viel Stück gibt es? \par\noindent\textit{Çend serrî vêrdî ra.} - Es sind einige Jahre vergangen.}

\entry{çem}{}{nom}{n}{}{Fluss}

\entry{çember}{}{nom}{n}{}{Ring, Reif, Kessel}

\entry{Çemîşgezek}{}{nom}{n}{}{\textit{Stadt in Dêrsim (türk. Çemişgezek)}}

\entry{çep}{}{adv}{}{}{links}

\entry{çep a}{}{adv}{}{}{verkehrt, verkehrtherum \par\noindent\textit{Sima çep a şîyê.} - Ich seid den falschen Weg gegangen.}

\entry{çeper}{}{nom}{n}{}{Zaun, Abzäuning; Abgrenzung}

\entry{çeqer}{}{adj}{-e}{}{gelb}

\entry{çerdene}{}{verb}{}{}{weiden, hüten}

\entry{çerme}{}{nom}{n}{}{\textsuperscript{1}Leder \textsuperscript{2}Haut $\rightarrow$ \textbf{poste}}

\entry{çewres}{çawres, çores}{zahl}{}{}{vierzig}

\entry{çê}{kê}{nom}{n}{keye}{\textsuperscript{1}Heim, Zuhause, Daheim \textsuperscript{2}Familie}

\entry{çêber}{}{nom}{n}{keyber}{Tür, Haustür}

\entry{çêf}{}{nom}{n}{}{$\rightarrow$ \textbf{kêf}}

\entry{çêna}{}{nom}{m}{kêna}{Tochter}

\entry{çêna amike}{}{nom}{m}{emkêna}{Cousine \textit{(Tochter der Tante väterlicherseits)}}

\entry{çêna apî}{çêna dedî, derezaye}{nom}{m}{datkêna}{Cousine \textit{(Tochter des Onkels väterlicherseits)}}

\entry{çêna xalî}{}{nom}{m}{xalkêna}{Cousine \textit{(Tochter des Onkels mütterlicherseits)}}

\entry{çêna xalike}{}{nom}{m}{xalekêna}{Cousine \textit{(Tochter der Tante mütterlicherseits)}}

\entry{çêneke}{}{nom}{m}{kêneke}{Mädchen}

\entry{çêr}{}{adj}{-e}{}{mutig, tapfer, kühn, wacker, beherzt}

\entry{çêvêsaye}{}{intrj}{}{keyveşaye}{\textit{Ausruf für Wut, Empörung, Bedauern}}

\entry{çi}{çik}{frg}{n}{}{was \par\noindent\textit{O çik o?} - Was ist das? \par\noindent\textit{Mi rê çi.} - Mir egal.}

\entry{çik}{}{frg}{}{}{$\rightarrow$ \textbf{çi}}

\entry{çike}{}{konj}{}{}{weil, denn}

\entry{çila}{}{nom}{m}{}{\textsuperscript{1}Fackel \textsuperscript{2}Lampe $\rightarrow$ \textbf{lamba}}

\entry{çim}{}{nom}{n}{}{Auge \par\noindent\textit{Çiman ser} - Gern geschehen}

\entry{çime}{}{nom}{n}{}{Quelle, Quell}

\entry{çinayî}{çikî}{frg}{}{}{\textit{2. Fall von} $\rightarrow$ \textbf{çi}}

\entry{çinayî ra}{}{frg}{}{}{$\rightarrow$ \textbf{çira}}

\entry{çinayî rê}{}{frg}{}{}{wozu, wofür}

\entry{çiqas}{çiqaşî, çiqa, çixa}{frg}{}{}{\textsuperscript{1}wie viel \textit{(unzählbar)} \textsuperscript{2}wie (sehr) \par\noindent\textit{Çiqas keno?} - Wie viel macht das? \par\noindent\textit{Çiqas rindek o!} - Wie schön das ist!}

\entry{çira}{çirê}{frg}{}{}{warum, weshalb, wieso}

\entry{çirê}{çira}{frg}{}{}{warum, weshalb, wieso}

\entry{çita}{}{frg}{}{}{$\rightarrow$ \textbf{çi}}

\entry{çitan}{çiton}{frg}{}{çitur}{wie; inwiefern $\rightarrow$ \textbf{senî} \par\noindent\textit{Ti çitan a?} - Wie geht's dir?}

\entry{çiturî}{}{frg}{}{çitur}{wie; inwiefern $\rightarrow$ \textbf{senî} \par\noindent\textit{Ti çiturî ya?} - Wie geht's dir?}

\entry{çî}{}{nom}{n}{}{\textsuperscript{1}etwas \textsuperscript{2}Sache, Ding, Gegenstand}

\entry{çîçege}{}{nom}{m}{}{Blume $\rightarrow$ \textbf{vîlike}}

\entry{çî-mî}{}{nom}{n}{}{Sachen, Gegenstände}

\entry{çîn bîyene}{}{verb}{}{çin bîyene}{nicht sein, nicht existieren, nicht da sein, nicht vorhanden sein \par\noindent\textit{Kes çîn o.} - Niemand ist da. \par\noindent\textit{Perê mi çîn o.} - Ich habe kein Geld.}

\entry{çînitene}{çînayene}{verb}{}{}{ernten, abernten}

\entry{çîyê}{}{nom}{n}{}{\textsuperscript{1}irgendetwas, irgendeine Sache \textsuperscript{2}\textit{(verneint)} nichts, keine Sache}

\entry{çok}{}{nom}{n}{}{Knie $\rightarrow$ \textbf{zanî}}

\entry{çok ronayene}{}{verb}{}{}{knien, niederknien, auf die Knie gehen, auf die Knie fallen}

\entry{çol}{}{nom}{n}{}{Wüste}

\entry{çolîstan}{}{nom}{n}{}{Wüste, Wüstenlandschaft}

\entry{çop}{}{nom}{n}{}{Müll, Abfall}

\entry{çopdane}{}{nom}{m}{}{Mülleimer, Abfalleimer}

\entry{çopole}{}{nom}{m}{}{Ohrfeige, Backpfeife, Klaps $\rightarrow$ \textbf{şîlpa\"{x}e}}

\entry{çor}{}{zahl}{}{}{$\rightarrow$ \textbf{çar}}

\entry{çorês}{}{zahl}{}{}{$\rightarrow$ \textbf{çewres}}

\entry{çorr}{}{nom}{n}{}{ansteckende/virale Krankheit, Seuche $\rightarrow$ \textbf{jan}}

\entry{çü}{çüye}{nom}{n}{}{Stock, Stab}

\end{multicols}

%----------------------------------------------------------------------------------------
%	SECTION D
%----------------------------------------------------------------------------------------

\section*{D}

\begin{multicols}{2}

\entry{da-}{}{ptkl}{}{}{\textit{(vor Zahlen)} circa, etwa, ungefähr, um die \par\noindent\textit{Da-vîşt mordemî ameyî} - Ca. zwanzig Männer sind gekommen}

\entry{dahîna}{dayîna}{ptkl}{}{}{mehr \textit{(im Komparativ)}}

\entry{daîma}{dayma}{adv}{}{}{immer, stets, allzeit; unaufhörlich}

\entry{danî}{}{nom}{n}{}{Weizen-Kichererbsengrütze}

\entry{dar û ber}{dar-ber}{nom}{n}{}{\textsuperscript{1}Bäume, Pflanzen \textit{(Kollektivum)} \textsuperscript{2}Natur}

\entry{dare}{}{nom}{m}{}{Baum}

\entry{darlox}{darlo\"{x}}{nom}{n}{}{\textit{die hölzerne Halterung für} $\rightarrow$ \textbf{lo\"{x}e}}

\entry{dawa}{}{nom}{m}{}{Fall, (An)Klage, Prozess $\rightarrow$ \textbf{doze}}

\entry{dawul}{}{nom}{n}{}{Trommel}

\entry{daye}{}{pron}{}{}{$\rightarrow$ \textbf{aye}}

\entry{dayene}{}{verb}{}{}{geben \par\noindent\textit{Bide!} - Gib!}

\entry{dayê}{}{vok}{}{}{\textsuperscript{1}\textit{Anredefall von} $\rightarrow$ \textbf{maye} \textsuperscript{2}\textsc{intrj} \textit{Ausdruf der Wehklage}}

\entry{dayîre}{}{nom}{n}{}{Kreis}

\entry{de\textsuperscript{1}}{}{post}{}{}{\textsuperscript{1}in, im; bei \textsuperscript{2}mit}

\entry{de\textsuperscript{2}}{}{intrj}{}{}{\textit{Ausruf der Anrede oder der Hervorhebung (vor allem in Liedern)}}

\entry{ded}{}{nom}{n}{}{Onkel väterlicherseits $\rightarrow$ \textbf{ap}}

\entry{delal}{}{adj}{-e}{}{hübsch, schön, stattlich}

\entry{Delale}{}{nom}{m}{}{\textit{weibl. Vorname}}

\entry{dele}{}{nom}{m}{}{Hündin}

\entry{deleverge}{}{nom}{m}{}{Wölfin}

\entry{dem}{}{nom}{n}{}{Zeit, Zeitraum}

\entry{demokrat}{}{nom}{-e}{}{Demokrat/in}

\entry{dengiz}{}{nom}{n}{}{$\rightarrow$ \textbf{derya}}

\entry{deqa}{}{nom}{m}{}{Minute}

\entry{der}{}{}{}{}{$\rightarrow$ \textbf{de\textsuperscript{1}}}

\entry{der û cîran}{}{nom}{n}{}{Nachbarn \textit{(Kollektivum)}, Nachbarschaft}

\entry{derd}{}{nom}{n}{}{Leid, Schmerz, Kummer \par\noindent\textit{Derdê to çik o?} - Was ist dein Problem?}

\entry{dere}{}{nom}{n}{}{Bach, Strom, Flüsschen}

\entry{dereza}{}{nom}{n}{}{Cousin (Sohn des Onkels väterlicherseits)}

\entry{derê}{}{ptkl}{}{}{wohl, etwa \textit{(Fragepartikel)} $\rightarrow$ \textbf{tirêm}}

\entry{derg}{}{adj}{-e}{}{lang, groß \textit{(Mensch)}}

\entry{derheqa ... de}{}{zirk}{}{derheqê ... de}{bezüglich, in Bezug auf, betreffend, hinsichtlich}

\entry{derse}{}{nom}{m}{}{\textsuperscript{1}Unterricht, Vorlesung \textsuperscript{2}Lektion, Lehre \par\noindent\textit{dersa xo kerdene} - (für den Unterricht) lernen}

\entry{derse dayene}{}{verb}{}{}{\textsuperscript{1}unterrichten, Unterricht geben \textsuperscript{2}eine Lektion erteilen, eine Lehre erteilen}

\entry{derûdor}{}{nom}{n}{}{Umgebung, Umland}

\entry{derya}{}{nom}{n}{}{Meer}

\entry{des}{}{zahl}{}{}{zehn}

\entry{desinde}{}{adv}{}{}{sofort, eilig, alsbald, auf der Stelle}

\entry{dest}{}{nom}{n}{}{Hand}

\entry{dest ci kerdene}{}{verb}{}{dest pêkerdene}{anfangen, beginnen \par\noindent\textit{O dest be karê xo keno} - Er fängt mit seiner Arbeit an}

\entry{dest\.{t}al}{dest\.{t}ol}{adj}{-e}{}{mit leeren Händen}

\entry{destûr}{}{nom}{n}{}{Erlaubnis, Genehmigung, Berechtigung \par\noindent\textit{Destûr waştene} - Um Erlaubnis fragen/bitten}

\entry{deşte}{}{nom}{m}{}{Ebene, Tiefland, Flachland}

\entry{deve}{}{nom}{-îye}{}{Kamel}

\entry{dewe}{}{nom}{m}{}{Dorf}

\entry{dewiz}{dewic}{nom}{-e}{dewij}{\textsuperscript{1}Dorfbewohner/in, Dörfler/in \textsuperscript{2}Farmpächter/in}

\entry{dewlemend}{}{adj}{-e}{}{reich, wohlhabend, vermögend}

\entry{dewlete}{}{nom}{n}{}{Staat}

\entry{dewletî}{}{adj}{-ye}{}{reich, wohlhabend, vermögend}

\entry{dewr}{}{nom}{n}{}{Ära, Epoche, Zeit, Zyklus}

\entry{dewrês}{}{adj}{-e}{}{\textsuperscript{1}Dervisch/in \textsuperscript{2}heilig}

\entry{dey}{}{pron}{}{}{$\rightarrow$ \textbf{ey}}

\entry{deyîrbaz}{}{nom}{-e}{}{Sänger/in $\rightarrow$ \textbf{ozan}}

\entry{dez}{dec}{nom}{n}{}{Schmerz}

\entry{Dêrsim}{Dêsim}{nom}{n}{}{\textit{Gebiet in Kurdistan (türk. Tunceli)}}

\entry{dêrsimiz}{dêrsimic}{nom}{-e}{dêrsimij}{Dêrsimer/in; jemand, der aus Dêrsim stammt}

\entry{dês}{}{nom}{n}{}{Wand, Mauer}

\entry{di}{dide}{zahl}{}{}{zwei}

\entry{di-hîrê}{}{adv}{}{}{ein paar}

\entry{dibistane}{}{nom}{m}{}{Grundschule}

\entry{didan}{dizan, dizon}{nom}{n}{}{Zahn}

\entry{didansaz}{}{nom}{-e}{}{Zahnarzt/Zahnärztin}

\entry{diganî}{dicanî}{adj}{m}{}{schwanger}

\entry{dikan}{}{nom}{n}{}{Laden, Geschäft}

\entry{dikandar}{}{nom}{-e}{}{Ladenbesitzer/in, (Gemischtwaren)Händler/in}

\entry{dim}{}{nom}{n}{}{Schwanz, Schweif}

\entry{dima}{}{adv}{}{}{hinterher, nach, danach}

\entry{dima şîyene}{}{verb}{}{}{nachgehen, hinterhergehen, verfolgen}

\entry{dime}{}{adv}{}{}{dahinter, danach \par\noindent\textit{O ra dime} - Danach, anschließend}

\entry{dime ra}{}{}{}{}{$\rightarrow$ \textbf{dima}}

\entry{dimilî}{}{nom}{-ye}{}{\textsuperscript{1}Dimilî, Zaza \textsuperscript{2}\textsc{nom} \textit{m} Dimilî, Zazakî $\rightarrow$ \textbf{kirmanckî}}

\entry{ding}{}{nom}{n}{}{Haltestelle, Station}

\entry{diqet}{diqat}{nom}{n}{}{\textsuperscript{1}Acht, Vorsicht, Umsicht \textsuperscript{2}Sorgfalt, Gründlichkeit}

\entry{diqet kerdene}{}{verb}{}{}{Acht geben, achten, vorsehen, aufpassen}

\entry{diseme}{}{nom}{n}{dişeme}{Montag}

\entry{dismen}{}{nom}{n}{}{Feind}

\entry{ditene}{}{verb}{}{}{melken}

\entry{dizan}{dizon}{nom}{n}{}{$\rightarrow$ \textbf{didan}}

\entry{dizd}{}{nom}{-e}{}{Dieb/in}

\entry{dizdî ya}{dijdî ya}{adv}{}{}{\textsuperscript{1}heimlich, geheim \textsuperscript{2}diebisch, verstohlen}

\entry{Dîcle}{}{nom}{m}{}{\textsuperscript{1}Tigris \textsuperscript{2}\textit{weibl. Vorname}}

\entry{dîk}{}{nom}{n}{}{Hahn, Gockel}

\entry{dîna}{dînya}{nom}{m}{dinya}{\textsuperscript{1}Welt \textsuperscript{2}(Planet) Erde}

\entry{dînan}{}{pron}{}{}{$\rightarrow$ \textbf{înan}}

\entry{dîploma}{}{nom}{m}{}{Diplom, Studienabschluss}

\entry{Dîyarbekir}{}{nom}{n}{}{\textit{Stadt u. Gebiet in Kurdistan (türk. Diyarbakır)}}

\entry{dîyarbekiriz}{dîyarbekiric}{nom}{-e}{dîyarbekirij}{Dîyarbekirer/in; jemand, der aus Dîyarbekir stammt}

\entry{do}{}{nom}{n}{}{\textit{Joghurtgetränk}}

\entry{doktor}{toxtor}{nom}{-e}{}{\textsuperscript{1}Arzt/Ärztin \textsuperscript{2}Doktor/in}

\entry{dolabe}{}{nom}{m}{}{Schrank}

\entry{doman}{}{nom}{-e}{}{Kind}

\entry{dorme}{}{nom}{n}{}{Umfeld, Umgebung, Umkreis}

\entry{dorûver}{}{nom}{n}{}{Umwelt, Umgebung, Umfeld}

\entry{dorûvernas}{}{nom}{-e}{}{Ökologe/in, Umweltforscher/in}

\entry{dorûvernasîye}{}{nom}{m}{}{Ökologie, Umweltforschung}

\entry{dost}{}{nom}{-e}{}{Freund/in, Kumpan, Kumpel \par\noindent\textit{Dost û dismen} - Freund und Feind}

\entry{dosya}{}{nom}{m}{}{Ordner}

\entry{dot}{}{nom}{n}{}{drüben, die andere Seite; jenseits}

\entry{doze}{}{nom}{m}{}{Fall, (An)Klage, Prozess $\rightarrow$ \textbf{dawa} \par\noindent\textit{doze kerdene} - (be)klagen, anklagen, verklagen}

\entry{duşt}{}{nom}{n}{}{\textsuperscript{1}Lage, Stand \textsuperscript{2}Niveau, Ebene, Höhe}

\entry{duştê ... de}{}{zirk}{}{}{gegen, gegenüber}

\entry{Duzgin Baba}{Duzgin Bava}{nom}{n}{}{\textit{ein heiliger Berg in Dêrsim}}

\entry{dû}{}{nom}{n}{}{$\rightarrow$ \textbf{dü}}

\entry{dûrî}{dürî}{adv}{}{}{entfernt, fern, weit (weg), entlegen $\rightarrow$ \textbf{dürî}}

\entry{dûrîvîn}{}{nom}{n}{}{Fernglas, Teleskop}

\entry{dü}{}{nom}{n}{dû}{Rauch}

\entry{dürî}{dûrî}{adv}{}{}{entfernt, fern, weit (weg), entlegen $\rightarrow$ \textbf{dûrî}}

\entry{DYA (Dewletê Yewbîyayeyî yê Amerîka)}{}{nom}{zh}{}{USA (Vereinigte Staaten von Amerika)}

\end{multicols}

%----------------------------------------------------------------------------------------
%	SECTION E
%----------------------------------------------------------------------------------------

\section*{E}

\begin{multicols}{2}

\entry{e}{}{ptkl}{}{}{ja $\rightarrow$ \textbf{heya}}

\entry{ebe}{ebi, eve}{}{}{}{$\rightarrow$ \textbf{be}}

\entry{ebe ... ra}{eve ... ra}{}{}{}{$\rightarrow$ \textbf{be ... ra}}

\entry{ecêba}{}{ptkl}{}{}{wohl, etwa \textit{(Fragepartikel)} $\rightarrow$ \textbf{tirêm}}

\entry{ede ... de}{edi ... de}{zirk}{}{}{in, im (räumlich)}

\entry{edebîyat}{}{nom}{n}{}{Literatur}

\entry{efsane}{}{nom}{n}{}{Legende, Sage, Mythos}

\entry{eke}{}{adv}{}{}{\textsuperscript{1}wenn, falls \textsuperscript{2}als, sobald}

\entry{ekîpe}{ekîbe}{nom}{m}{}{Manschaft, Team}

\entry{ekonomî}{}{nom}{m}{}{Wirtschaft, Ökonomie}

\entry{ekonomîk}{}{adj}{-e}{}{\textsuperscript{1}wirtschaftlich, ökonomisch \textsuperscript{2}sparsam}

\entry{ekonomîst}{}{nom}{-e}{}{Wirtschaftler/in, Ökonom/in, Wirtschaftswissenschaftler/in}

\entry{ekran}{}{nom}{n}{}{Bildschirm}

\entry{eleqe}{alaqe}{nom}{m}{}{\textsuperscript{1}Interesse, Anteilnahme \textsuperscript{2}Bezug, Zusammenhang}

\entry{eleqedar}{alaqedar}{adj}{-e}{}{\textsuperscript{1}interessiert, teilnahmsvoll \textsuperscript{2}relevant, zuständig, betroffen}

\entry{elewî}{alawî}{nom}{-ye}{}{Alevite/in}

\entry{emser}{esmer}{adv}{}{}{heuer, dieses Jahr}

\entry{emso}{esmo}{adv}{}{emşo}{heute Abend}

\entry{endî}{êndî}{adv}{}{êdî}{\textsuperscript{1}nunmehr, mehr, forthin, fortan \textsuperscript{2}schließlich, endlich \textsuperscript{3}nicht mehr}

\entry{enstruman}{}{nom}{n}{}{Instrument}

\entry{enteresan}{}{adj}{-e}{}{\textsuperscript{1}interessant, achtungswert \textsuperscript{2}beeindruckend, erstaunlich $\rightarrow$ \textbf{balkêş}}

\entry{ereb}{}{nom}{}{}{$\rightarrow$ \textbf{arab}}

\entry{erebe}{}{}{}{}{$\rightarrow$ \textbf{arebe}}

\entry{erê}{}{intrj}{}{}{\textit{Umgangssprachliche Anrede für Frauen}}

\entry{ero}{}{intrj}{}{}{\textit{Umgangssprachliche Anrede für Männer}}

\entry{err}{}{}{}{}{$\rightarrow$ \textbf{errik}}

\entry{errik}{}{intrj}{}{}{\textit{Ausruf für Verwunderung, Bestürztheit oder Entrüstung}}

\entry{Erzingan}{}{nom}{n}{}{\textit{Stadt in Kurdistan (türk. Erzincan)}}

\entry{erzinganiz}{erzinganic}{nom}{-e}{}{Erzinganer/in; jemand, der aus Erzingan stammt}

\entry{esker}{asker}{nom}{-e}{}{Soldat/in $\rightarrow$ \textbf{leşker}}

\entry{eskerîye}{askerîye}{nom}{m}{}{\textsuperscript{1}Militär \textsuperscript{2}Soldatentum \textsuperscript{3}Wehrdienst, Militärdienst $\rightarrow$ \textbf{leşkerîye}}

\entry{eskerênî}{}{nom}{m}{}{$\rightarrow$ \textbf{eskerîye}}

\entry{esmer}{}{adv}{}{}{$\rightarrow$ \textbf{emser}}

\entry{esmo}{}{adv}{}{}{$\rightarrow$ \textbf{emso}}

\entry{est bîyene}{}{verb}{}{}{sein (im Sinne von \textit{geben}), existieren, da sein, vorhanden sein \par\noindent\textit{Çi est o?} - Was gibt's?}

\entry{Estamol}{Astamol}{nom}{n}{}{$\rightarrow$ \textbf{Îstanbul}}

\entry{eşqîya}{}{nom}{n, m}{}{Räuber/in, Marodeur/in, Brigant/in}

\entry{eştene}{}{verb}{}{}{werfen, schmeißen}

\entry{eteg}{etege}{nom}{n}{}{Rock}

\entry{ewro}{öyro, ewre}{adv}{}{}{heute}

\entry{Ewropa}{}{nom}{m}{}{Europa}

\entry{ey}{ê}{pron}{}{}{\textit{2. Fall von}$\rightarrow$ \textbf{o\textsuperscript{1}}}

\entry{ez}{az, e}{pron}{}{}{ich}

\end{multicols}

%----------------------------------------------------------------------------------------
%	SECTION Ê
%----------------------------------------------------------------------------------------

\section*{Ê}

\begin{multicols}{2}

\entry{ê\textsuperscript{1}}{}{pron}{m}{}{sie (Plural)}

\entry{ê\textsuperscript{2}}{}{dem}{m}{}{jene; die dort (Plural)}

\entry{ê\textsuperscript{3}}{}{}{}{}{$\rightarrow$ \textbf{yê}}

\entry{êlule}{}{nom}{m}{}{September}

\entry{êndî}{}{}{adv}{}{$\rightarrow$ \textbf{endî}}

\entry{êne}{}{nom}{n}{}{$\rightarrow$ \textbf{yene}}

\entry{êzîdî}{}{nom}{-e}{}{Jeside/Jesidin}

\end{multicols}

%----------------------------------------------------------------------------------------
%	SECTION F
%----------------------------------------------------------------------------------------

\section*{F}

\begin{multicols}{2}

\entry{fam}{}{nom}{n}{}{Verständnis, das Begriffsvermögen; Verstand}

\entry{fam kerdene}{}{verb}{}{fehm kerdene}{verstehen, begreifen}

\entry{fanîla}{}{nom}{m}{}{\textsuperscript{1}Flanell(hemd), dünner Pollover \textsuperscript{2}Unterhemd}

\entry{faqat}{}{}{}{}{$\rightarrow$ \textbf{feqet}}

\entry{faris}{}{nom}{-e}{}{Perser/in}

\entry{fariskî}{}{nom}{m}{}{Persisch}

\entry{faşîla}{}{nom}{m}{fasulya}{Bohne}

\entry{Fatike}{}{nom}{m}{}{\textit{weiblicher Vorname}}

\entry{fatura}{}{nom}{m}{}{Rechnung, Abrechnung}

\entry{favorî}{}{adj}{-ye}{}{Favorit, Lieblings-}

\entry{fay}{}{nom}{m}{}{Mal \textbf{$\rightarrow$ rey} \par\noindent\textit{Na fay} - Diesmal}

\entry{fayde}{fayda}{nom}{n}{fayde}{\textsuperscript{1}Vorteil, Nutzen, Wohl \textsuperscript{2}Gewinn, Profit}

\entry{fayna}{}{adv}{}{}{nochmal, erneut, wieder \textbf{$\rightarrow$ reyna}}

\entry{fecîr}{}{nom}{n}{fecir, sipêde}{Morgendämmerung, Tagesanbruch}

\entry{fek}{}{nom}{n}{}{Mund}

\entry{feqet}{faqat}{}{konj}{}{jedoch}

\entry{feqîr}{}{adj}{-e}{}{arm, bedürftig, unvermögend}

\entry{ferheng}{}{nom}{n}{}{Wörterbuch}

\entry{ferq}{}{nom}{n}{}{Unterschied, Differenz}

\entry{festîvale}{}{nom}{m}{}{Fest, Festival}

\entry{fetelîyayene}{}{verb}{}{fetilîyayene}{spazieren, spazieren gehen, herumgehen, herumwandern}

\entry{filan}{}{pron}{}{}{jene/r/s, irgendein/e/r/s}

\entry{fikr}{}{nom}{n}{}{\textsuperscript{1}Idee, Gedanke, Einfall, Überlegung \textsuperscript{2}Meinung, Urteil}

\entry{fikr kerdene}{}{verb}{}{}{\textsuperscript{1}denken, überlegen \textsuperscript{2}meinen, erachten \textsuperscript{3}erwägen, besinnen, in Erwägung ziehen}

\entry{fikirîyayene}{}{verb}{}{}{\textsuperscript{1}denken, überlegen \textsuperscript{2}meinen, erachten \textsuperscript{3}erwägen, besinnen, in Erwägung ziehen}

\entry{Firat}{}{nom}{n}{}{\textsuperscript{1}Euphrat \textsuperscript{2}\textit{männl. Vorname}}

\entry{firine}{}{nom}{m}{}{\textsuperscript{1}Ofen \textsuperscript{2}Bäckerei}

\entry{firne}{}{nom}{m}{}{Nasenloch}

\entry{fîîl}{}{nom}{n}{}{Verb}

\entry{fîl}{}{nom}{-e}{}{Elefant}

\entry{fîncane}{}{nom}{m}{}{Tasse}

\entry{fotograf}{}{nom}{n}{}{Foto}

\entry{fotografkêş}{}{nom}{-e}{}{Fotograf/in}

\entry{fransiz}{}{nom}{-e}{}{Franzose/Französin}

\entry{franskî}{}{nom}{m}{}{Französisch}

\entry{Fransa}{}{nom}{m}{}{Frankreich}

\entry{futbol}{}{nom}{n}{}{Fußball}

\entry{futbolbaz}{}{nom}{-e}{}{Fußballer/in, Fußballspieler/in}

\end{multicols}

%----------------------------------------------------------------------------------------
%	SECTION G
%----------------------------------------------------------------------------------------

\section*{G}

\begin{multicols}{2}

\entry{ga}{}{nom}{n}{}{Ochse}

\entry{game}{}{nom}{m}{}{\textsuperscript{1}Schritt \textsuperscript{2}Augenblick, Moment \par\noindent\textit{Gama ke} - Als, wenn}

\entry{gavare}{}{nom}{m}{}{$\rightarrow$ \textbf{payîzo peyên}}

\entry{ga\"{x}an\textsuperscript{1}}{gaxan, gaxand}{nom}{n}{}{\textit{ein im Dezember und Januar stattfindendes Winterfest}}

\entry{ga\"{x}an\textsuperscript{2}}{gaxan, gaxand}{nom}{n}{kanûne}{Dezember}

\entry{ge... ge...}{}{konj}{}{}{mal... mal..., ab und zu, ab und an, hier und da \par\noindent\textit{Ge ti kena, ge ez keno} - Mal machst du's, mal mache ich's}

\entry{ge-gane}{}{adv}{}{}{manchmal, mitunter, gelegentlich, ab und zu, ab und an}

\entry{gemî}{}{nom}{m}{}{Schiff $\rightarrow$ \textbf{keştî}}

\entry{genim}{}{nom}{n}{}{Weizen}

\entry{germ}{}{adj}{-e}{}{warm \textit{(i.d.R. Wetter, Luft)}}

\entry{germin}{}{adj}{-e}{}{warm \textit{(i.d.R. Oberflächen, Gegenstände etc.)}}

\entry{germî}{}{nom}{m}{}{\textsuperscript{1}Suppe $\rightarrow$ \textbf{sorbike} \textsuperscript{2}Aschura}

\entry{gerre}{}{nom}{n}{}{\textsuperscript{1}Beschwerde, Beanstandung \textsuperscript{2}Klage, Anklage}

\entry{gerre kerdene}{}{verb}{}{}{\textsuperscript{1}sich beschweren, beanstanden, reklamieren, lamentieren \textsuperscript{2}klagen, Beschwerde einlegen}

\entry{gewr}{}{adj}{-e}{}{grau}

\entry{gilorik}{}{adj}{}{}{$\rightarrow$ \textbf{gilorin}}

\entry{gilorin}{gilorik, gilorikin}{adj}{-e}{}{rund, kugelig}

\entry{giran}{gira}{adj}{-e}{}{\textsuperscript{1}schwer, beschwert \textsuperscript{2}\textsc{adv} langsam, schwerfällig \textsuperscript{3}mühsam, wuchtig  \textsuperscript{4}schwerwiegend}

\entry{giran-giran}{}{adv}{}{}{langsam, allmählich}

\entry{girê}{}{nom}{n}{}{Knoten}

\entry{girê dayene}{}{verb}{}{}{\textsuperscript{1}binden, fesseln, verschnüren \textsuperscript{2}verbinden, verknoten, verknüpfen \textsuperscript{3}befestigen, anbinden, anmachen, anheften}

\entry{girmike}{}{nom}{m}{}{Faust}

\entry{girs}{}{adj}{-e}{}{groß}

\entry{gizgizîyayene}{}{verb}{}{}{\textsuperscript{1}(sich) schütteln, beben, erzitternn \textsuperscript{2}aufgewiegelt werden, aufgehetzt werden, provoziert werden}

\entry{gîtare}{}{nom}{m}{}{Gitarre}

\entry{Gola Wanî}{}{nom}{m}{}{Wan-See}

\entry{gole}{gol \textit{(n)}}{nom}{m}{}{See}

\entry{gonî}{}{nom}{m}{}{Blut}

\entry{gore}{}{post}{}{}{gemäß, laut, nach, entsprechend \par\noindent\textit{To ra gore} - Laut dir}

\entry{goristan}{}{nom}{n}{}{Friedhof}

\entry{gos}{}{nom}{n}{goş}{Ohr}

\entry{gos ro ci nayene}{}{verb}{}{goş re ci nayene}{\textsuperscript{1}befolgen, gehorchen \textsuperscript{2}Glauben schenken}

\entry{gos û kerrik}{gos-kerrik}{nom}{n, zh}{}{Gehör, Hörvermögen}

\entry{gosare}{}{nom}{n}{goşare}{Ohrring \par\noindent\textit{Qesê mi to rê gosare bo.} - Merk dir meine Worte.}

\entry{gos dayene}{}{verb}{}{goş dayene}{\textsuperscript{1}hinhören, anhören, horchen, Gehör schenken \textsuperscript{2}aufpassen \textsuperscript{3}befolgen, gehorchen}

\entry{gosdarî kerdene}{gosdarênî kerdene}{verb}{}{goşdarî kerdene}{zuhören, anhören}

\entry{goşt}{}{nom}{n}{}{Fleisch}

\entry{govende}{}{nom}{m}{}{Govend \textit{(kurd. Volkstanz)}}

\entry{goynayene}{}{verb}{}{}{loben, preisen, rühmen}

\entry{goze}{}{nom}{m}{}{Walnuß}

\entry{gram}{}{nom}{n}{}{Gramm}

\entry{gramer}{}{nom}{n}{}{Grammatik, Sprachlehre}

\entry{gucige}{guzige}{nom}{m}{}{Februar}

\entry{gude}{}{nom}{m}{}{Ball, Kugel}

\entry{guk}{}{nom}{-e}{}{Kalb, Milchkalb}

\entry{gulane}{}{nom}{m}{}{Mai}

\entry{gule\textsuperscript{1}}{}{nom}{n}{}{Geschoss, Projektil}

\entry{gule\textsuperscript{2}}{}{nom}{m}{}{\textsuperscript{1}Rose \textsuperscript{2}Hals \textsuperscript{3}\textit{weibl. Vorname} \par\noindent\textit{Gul û sosinî} - Blumen \textit{(Kollektivum)}}

\entry{gure}{}{nom}{n}{}{(zweckgebundene oder entgeltliche) Arbeit, Betätigung, Beschäftigung}

\entry{gurekar}{}{adj}{-e}{}{arbeitsam, fleißig, emsig}

\entry{guretene}{gurtene}{verb}{}{girewtene}{\textsuperscript{1}nehmen, abnehmen, einholen \textsuperscript{2}bekommen, kriegen, empfangen, erhalten}

\entry{gurenayene}{gurênayene}{verb}{}{}{\textsuperscript{1}betätigen, betreiben, beschäftigen, einsetzen, in Betrieb nehmen \textsuperscript{2}benutzen, nutzen, verwenden, anwenden}

\entry{gureyayene}{gurîyayene}{verb}{}{}{arbeiten, schaffen, werken, beschäftigen}

\entry{guvayene}{}{verb}{}{}{\textit{(Wind)} sausen, säuseln}

\entry{guzayene}{}{verb}{}{}{$\rightarrow$ \textbf{guvayene}}

\end{multicols}

%----------------------------------------------------------------------------------------
%	SECTION H
%----------------------------------------------------------------------------------------

\section*{H}

\begin{multicols}{2}

\entry{ha}{}{intrj}{}{} {\textit{Ausruf des Begreifens}}

\entry{ha... ha...}{}{konj}{}{}{ob... oder... \par\noindent\textit{Ha ti ha ez.} - Ob du oder ich.}

\entry{haca}{}{adv}{}{}{$\rightarrow$ \textbf{haza}}

\entry{hafta}{}{nom}{n}{}{$\rightarrow$ \textbf{hefte}}

\entry{hahû}{}{intrj}{}{}{\textit{Ausruf des Verwunderung oder Empörung}}

\entry{hak}{}{nom}{n}{}{Ei}

\entry{hal}{}{nom}{n}{}{\textsuperscript{1}Lage, Situation, Stand \textsuperscript{2}Zustand, Status, Sachlage \textsuperscript{3}Vorfall, Vorkommnis \textsuperscript{4}Fall, Kasus}

\entry{halete}{}{nom}{m}{}{\textbf{$\rightarrow$ xelate}}

\entry{hama}{hema, ama}{konj}{}{}{aber, jedoch $\rightarrow$ \textbf{labelê}}

\entry{haq}{}{nom}{n}{}{$\rightarrow$ \textbf{heq}}

\entry{Haq}{}{nom}{n}{}{$\rightarrow$ \textbf{Heq}}

\entry{har}{}{adj}{-e}{}{\textsuperscript{1}tollwütig \textsuperscript{2}tobsüchtig, wild, rappelig}

\entry{hard}{}{nom}{n}{}{$\rightarrow$ \textbf{herd}}

\entry{harme}{herme}{nom}{n}{}{Arm}

\entry{has kerdene}{hes kerdene}{verb}{}{hes kerdene}{\textsuperscript{1}lieben \textsuperscript{2}mögen, gern haben  \par\noindent\textit{Ez to ra has keno.} - Ich liebe dich.}

\entry{haştîye}{}{nom}{m}{}{$\rightarrow$ \textbf{aştîye}}

\entry{hata}{heta, hetanî, heyanî}{prep}{}{heta}{bis, bis zu $\rightarrow$ \textbf{heta}}

\entry{hawt}{}{zahl}{}{}{$\rightarrow$ \textbf{hewt}}

\entry{hayde}{hayderê}{intrj}{}{}{\textit{Ausruf zur Aufforderung}}

\entry{haza}{haca}{}{adv}{}{dort, an jenem Ort}

\entry{hazar}{}{zahl}{}{}{$\rightarrow$ \textbf{hezar}}

\entry{hazir}{}{adj}{-e}{}{bereit, fertig; anwesend}

\entry{hazir û nazir}{}{adj}{-e}{}{allzeit bereit, allgegenwärtig; anwesend und bereitstehend}

\entry{hazîrane}{}{nom}{m}{hezîrane}{Juni}

\entry{hebe}{}{nom}{m}{}{Stück}

\entry{hebê}{}{adv}{}{}{ein Stück, ein bisschen, ein wenig}

\entry{hebikê}{}{adv}{}{}{$\rightarrow$ \textbf{hebê}}

\entry{hefte}{hafta}{nom}{n}{}{Woche $\rightarrow$ \textbf{heşte, hewte}}

\entry{helwa}{hewla}{nom}{m}{}{\textit{eine Süßspeise aus Mehl, Zucker (und Nüssen)}}

\entry{hem... hem kî...}{}{konj}{}{}{sowohl, als auch \par\noindent\textit{Hem mi ra hem kî to ra va.} - Er hat's sowohl dir als auch mir gesagt.}

\entry{hema}{}{adv}{}{}{\textsuperscript{1}noch, immer noch $\rightarrow$ \textbf{hona} \textsuperscript{2}$\rightarrow$ \textbf{hama}}

\entry{hema-hema}{}{adv}{}{}{\textsuperscript{1}fast, beinahe \textsuperscript{2}circa, etwa, mehr oder weniger}

\entry{hemal}{}{nom}{-e}{}{Kurier/in, Lieferant/in}

\entry{hemdem}{}{adj}{-e}{}{\textsuperscript{1}gleichzeitig, zeitgleich \textsuperscript{2}zeitgenössisch, zeitgemäß, modern}

\entry{hemgên}{hengmên}{nom}{n}{hingimên}{Honig}

\entry{Hemilka}{}{nom}{m}{}{$\rightarrow$ \textbf{Amerîka}}

\entry{hemkar}{}{nom}{-e}{}{}{\textsuperscript{1}Mitarbeiter/in, Mitwirkende/r, Partner/in \textsuperscript{2}Kollaborateur/in}

\entry{hemmana}{}{adj}{-ye}{}{gleichbedeutend, synonym}

\entry{hemserr}{}{adj}{-e}{}{\textsuperscript{1}gleichaltrig \textsuperscript{2} \textsc{nom} Altersgenosse/in}

\entry{hemşîra}{}{nom}{n, m}{}{Krankenschwester, Krankenpfleger/in}

\entry{hemwelatî}{}{adj}{-ye}{}{Staatsbürger/in, Bürger/in}

\entry{henî}{heyn}{adv}{}{winî}{so, in jener Art, in jener Form}

\entry{hengmên}{}{nom}{m}{}{$\rightarrow$ \textbf{hemgên}}

\entry{hengure}{}{nom}{m}{}{Traube}

\entry{hente}{}{adv}{}{}{$\rightarrow$ \textbf{honde}}

\entry{hepisxane}{}{nom}{n}{}{Gefängnis}

\entry{heq}{heqe \textit{(m)}}{nom}{n}{}{\textsuperscript{1}Recht, Anrecht, Anspruch \textsuperscript{2}Forderung \textsuperscript{3}Gerechtigkeit 
\par\noindent\textit{No heqê tü yo.} - Das steht dir zu.}

\entry{Heq}{Haq}{nom}{n}{}{Gott $\rightarrow$ \textbf{Homa} 
\par\noindent\textit{Heq bo} - Bei Gott
\par\noindent\textit{Heq razî bo!} - Vergelt's Gott!}

\entry{heqdar}{haqdar}{adj}{-e}{}{\textsuperscript{1}recht, gerecht, gerechtfertigt \textsuperscript{2}begründet, berechtigt}

\entry{heqlî}{}{adj}{-ye}{}{$\rightarrow$ \textbf{heqdar}}

\entry{her\textsuperscript{1}}{}{nom}{-e}{}{Esel}

\entry{her\textsuperscript{2}}{}{ptkl}{}{}{jede/r/s
\par\noindent\textit{Her roze} - Jeden Tag}

\entry{her çî}{}{nom}{n}{}{alles; jede Sache}

\entry{her kes}{}{nom}{n}{}{jeder, alle, alle Mann}

\entry{herbî}{}{adv}{}{}{schnell, rasch, zügig}

\entry{herçiqas}{}{ptkl}{}{}{obwohl, obgleich, obschon}

\entry{herd}{hard}{nom}{n}{erd}{Boden, Erde, Erdoberfläche}

\entry{heredan}{}{adj}{-e}{heridan}{erzürnt, böse, beleidigt, verbittert, grollend}

\entry{heredîyayene}{}{verb}{}{heridîyayene}{beleidigt sein, schmollen, jmd. böse sein  $\rightarrow$ \textbf{miradîyayene}}

\entry{herednayene}{}{verb}{}{heridnayene}{vergrämen, verärgern, verstimmen  $\rightarrow$ \textbf{miradnayene}}

\entry{herre}{}{nom}{m}{}{Erde}

\entry{hertim}{}{adv}{}{}{immer, stets, durchgehend, fortwährend $\rightarrow$ \textbf{tim}}

\entry{hes}{}{nom}{-e}{heş}{Bär}

\entry{hes kerdene}{}{verb}{}{}{$\rightarrow$ \textbf{has kerdene}}

\entry{hesnayene}{}{verb}{}{}{hören}

\entry{hesrete}{}{nom}{m}{}{Sehnsucht, Verlangen}

\entry{hest}{}{adj}{-e}{}{dickflüssig, sämig}

\entry{heşt}{}{zahl}{}{}{acht}

\entry{heştay}{}{zahl}{}{}{achtzig}

\entry{heşte}{}{nom}{n}{}{Woche $\rightarrow$ \textbf{hefte, hewte}}

\entry{het}{}{nom}{n}{}{Seite; Neben-}

\entry{het de}{}{post}{}{}{bei, an der Seite von, neben}

\entry{heta}{hata, hetanî, heyanî}{prep}{}{}{bis, bis zu $\rightarrow$ \textbf{hata}}

\entry{hetê ... de}{}{zirk}{}{}{bei, an der Seite von, neben}

\entry{heval}{haval}{nom}{-e}{}{\textsuperscript{1}Freund/in, Kumpel/in $\rightarrow$ \textbf{albaz} {\textsuperscript{2}Weggefährte/in, Kumpane/in \par\noindent\textit{Berfîne hevala min a.} - Berfîn ist meine Freundin. \par\noindent\textit{Xizir hevalê to bo!} - Möge Xizir dich begleiten!}}

\entry{heval û hogir}{}{nom}{}{}{Freunde, Freundeskreis \textit{(Kollektivum)}}

\entry{hew}{}{intrj}{}{}{\textit{Ausruf für Verwunderung, Spott oder Staunen}}

\entry{hewa}{hawa}{}{}{}{\textsuperscript{1}Luft, Wetter \textsuperscript{2}Atmosphäre, Stimmung, Flair \textsuperscript{3}Lied, Melodie}

\entry{hewar}{hawar}{intrj}{}{}{\textit{Ausruf für Hilfe oder Wehklage}}

\entry{hewl}{}{adj}{-e}{}{\textsuperscript{1}gut \textsuperscript{2}(Charakter)gutmütig, großzügig, beherzt $\rightarrow$ \textbf{xirt}}

\entry{hewla}{}{nom}{m}{}{$\rightarrow$ \textbf{helwa}}

\entry{Hewlêr}{}{nom}{n}{}{Erbil}

\entry{hewlîye}{}{nom}{m}{}{\textsuperscript{1}Güte, Gutmütigkeit \textsuperscript{2}Gefallen, Gefälligkeit}

\entry{hewn}{}{nom}{n}{}{\textsuperscript{1}Schlaf \textsuperscript{2}Traum}

\entry{hewn a şîyene}{}{verb}{}{}{schlafen, einschlafen}

\entry{hewr}{höyr}{nom}{n}{}{Wolke}

\entry{hewrin}{höyrin}{adj}{-e}{}{bewölkt}

\entry{hewt}{hawt, hot}{zahl}{}{}{sieben}

\entry{hewtay}{hawtay, hotay}{zahl}{}{}{siebzig}

\entry{hewte}{}{nom}{n}{}{Woche $\rightarrow$ \textbf{hefte, heşte}}

\entry{hey}{}{intrj}{}{}{\textit{Ausruf für Anruf, Lob oder Tadel}}

\entry{heya}{}{ptkl}{}{eya}{ja}

\entry{heyanî}{}{prep}{}{}{$\rightarrow$ \textbf{heta}}

\entry{heyat}{hayat}{nom}{n}{}{Leben $\rightarrow$ \textbf{cuye}}

\entry{heyecan}{hêcan}{nom}{n}{}{die Spannung, die Aufregung, die Nervosität}

\entry{heyf}{}{nom}{n}{}{$\rightarrow$ \textbf{hêf}}

\entry{heyhat}{}{intrj}{}{}{\textit{Ausruf des Bedauerns}}

\entry{heykel}{}{nom}{n}{}{Statue, Skulptur}

\entry{heyn}{}{adv}{}{}{$\rightarrow$ \textbf{henî}}

\entry{heywan}{haywan}{nom}{-e}{}{Tier}

\entry{heywax}{}{intrj}{}{}{\textit{Ausruf für Klage, Bedauern oder Schmerz}}

\entry{hezar}{hazar}{zahl}{}{}{Tausend}

\entry{hêcan}{}{nom}{n}{}{$\rightarrow$ \textbf{heyecan}}

\entry{hêdî}{}{adv}{}{}{langsam}

\entry{hêf}{heyf}{nom}{n}{heyf}{\textsuperscript{1}Rache, Revanche \textsuperscript{2}schade \par\noindent\textit{Çi hêf!} - Wie schade! \par\noindent\textit{Çi hêf ke nêbî.} - Leider hat es nicht geklappt.}

\entry{hêga}{}{nom}{n}{}{Feld, Acker, Anbaufläche}

\entry{hêkate}{}{nom}{m}{hîkaye}{Geschichte, Erzählung $\rightarrow$ \textbf{hîkaye}}

\entry{hênî}{}{nom}{n}{}{Brunnen, Quell}

\entry{hêrdîse}{hêrîse}{nom}{m}{erdîşe}{Bart}

\entry{hêrige}{}{nom}{m}{}{Mirabelle, Braunelle, Pflaume $\rightarrow$ \textbf{alunça}}

\entry{hêrînayene}{}{verb}{}{}{kaufen, einkaufen}

\entry{hêrs}{}{nom}{n}{}{Wut, Zorn, Ärger, Frust}

\entry{hêrs bîyene}{}{verb}{}{}{wütend werden, sich ärgern}

\entry{hi}{}{intrj}{}{}{\textit{Ausruf für Verwunderung, Staunen oder Vorsicht}}

\entry{Hindîstan}{}{nom}{n}{}{Indien}

\entry{hindkî}{}{nom}{m}{}{Hindi}

\entry{hîra}{}{adj}{-ye}{}{breit, weit}

\entry{hîrem}{}{nom}{n}{}{Webstuhl, Webgeschirr}

\entry{hîrê}{}{zahl}{}{}{drei}

\entry{hîris}{hîrîs}{zahl}{}{}{dreißig}

\entry{hîkaye}{}{nom}{m}{}{Geschichte, Erzählung $\rightarrow$ \textbf{hêkate}}

\entry{holike}{}{nom}{m}{}{Laube, Berghütte}

\entry{Hollanda}{}{nom}{m}{}{Niederlande, Holland}

\entry{Homa}{}{nom}{n}{}{$\rightarrow$ \textbf{Heq}}

\entry{hona}{}{adv}{}{hîna}{\textsuperscript{1}noch, immer noch \textsuperscript{2}eben, bereits}

\entry{honde}{hente}{adv}{}{hende}{so, so wie, soviel (wie)}

\entry{honik}{}{adj}{-e}{}{kühl, frisch \textit{(Luft, Wetter)}}

\entry{hoparlor}{}{nom}{n}{}{Lautsprecher}

\entry{hope}{}{nom}{m}{}{\textsuperscript{1}Becken, Teich, Wasserbecken, Schwimmbecken \textsuperscript{2}Schwimmbad}

\entry{hot}{}{zahl}{}{}{$\rightarrow$ \textbf{hewt}}

\entry{hozan}{}{zahl}{}{}{$\rightarrow$ \textbf{ozan}}

\entry{hukmat}{}{nom}{n}{}{Regierung}

\entry{huner}{}{nom}{n}{}{\textsuperscript{1}Kunst \textsuperscript{2}Handwerk \textsuperscript{2}Talent, (Kunst)Fertigkeit, Geschick}

\entry{hunermend}{}{nom}{-e}{}{Künstler/in $\rightarrow$ \textbf{senetkar}}

\entry{hurdî}{}{adj}{-ye}{}{\textsuperscript{1}klein, winzig \textsuperscript{2}zerbröselt, zerstückelt}

\entry{hurdî kerdene}{}{verb}{}{}{\textsuperscript{1}klein machen, zerkleinern \textsuperscript{2}zerstückeln, zerbröseln}

\entry{hurendî}{}{nom}{m}{herinde}{Stelle, Platz}

\entry{hurendîya ... de}{}{zirk}{}{herinda ... de}{an Stelle von..., anstatt...}

\entry{huşk}{husk}{adj}{-e}{}{\textsuperscript{1}trocken \textsuperscript{2}hart}

\entry{huşk kerdene}{}{verb}{}{}{\textsuperscript{1}trocknen, trockenlegen \textsuperscript{2}härten, hart machen}

\entry{huşk û hol}{}{adj}{-e}{}{knochenhart, steinhart}

\entry{huqûq}{}{nom}{n}{}{\textsuperscript{1}Recht, Rechtswesen \textsuperscript{2}Jura}

\entry{huqûqnas}{}{nom}{-e}{}{Jurist/in; Rechtsgelehrter/in}

\end{multicols}

%----------------------------------------------------------------------------------------
%	SECTION I
%----------------------------------------------------------------------------------------

\section*{I}

\begin{multicols}{2}

\entry{ik}{ikî}{konj}{}{}{$\rightarrow$ \textbf{kî}}

\entry{israr}{}{nom}{n}{}{Beharren, Beharrlichkeit, Drängen}

\entry{israr kerdene}{}{verb}{}{}{beharren, bestehen, insistieren, drängen}

\entry{itka}{}{adv}{}{}{$\rightarrow$ \textbf{îta}}

\end{multicols}

%----------------------------------------------------------------------------------------
%	SECTION Î
%----------------------------------------------------------------------------------------

\section*{Î}

\begin{multicols}{2}

\entry{îbrankî}{}{nom}{m}{}{Hebräisch}

\entry{îdare}{}{nom}{n}{}{\textsuperscript{1}Führung, Leitung \textsuperscript{2}Amt, Verwaltung, Regierung \textsuperscript{3}Direktion, Geschäftsleitung $\rightarrow$ \textbf{sermîyanîye}}

\entry{îdare kerdene}{}{verb}{}{}{\textsuperscript{1}leiten, anführen, lenken, weisen, managen, steuern \textsuperscript{2}managen, administrieren, verwalten, regieren \textsuperscript{3}mit etw. auskommen \par\noindent\textit{Îdare keno.} - Es geht.}

\entry{îdarekar}{}{nom}{-e}{}{Leiter/in, Oberhaupt, Vorsteher/in, Manager/in $\rightarrow$ \textbf{sermîyan}}

\entry{îlawe}{}{nom}{n}{}{Anhang, Beigabe, Zusatz, Ergänzung}

\entry{îlawe kerdene}{}{verb}{}{}{anängen, hinzufügen, ergänzen, zugeben}

\entry{îmkan}{}{nom}{n}{}{Gelegenheit, Möglichkeit}

\entry{înan}{înû, îne}{pron}{}{}{\textit{2. Fall von}$\rightarrow$ \textbf{ê\textsuperscript{1}}}

\entry{înkar}{}{nom}{n}{}{Leugnung, Bestreitung, Abrede}

\entry{înkar kerdene}{}{verb}{}{}{leugnen, verleugnen, abstreiten, absprechen, abreden, in Abrede stellen}

\entry{însan}{îso, îsû}{nom}{-e}{}{\textsuperscript{1}Mensch \textsuperscript{2}man $\rightarrow$ \textbf{mordem\textsuperscript{1}}}

\entry{îngiliz}{}{nom}{-e}{}{Engländer/in}

\entry{îngilizkî}{}{nom}{m}{}{Englisch}

\entry{Îngilîstan}{}{nom}{n}{}{England}

\entry{îqaz}{}{nom}{n}{}{(Er)Mahnung, Warnung, Appel}

\entry{îqaz kerdene}{}{verb}{}{}{(er)mahnen, (ver)warnen, hinweisen}

\entry{Îran}{}{nom}{n}{}{Iran}

\entry{Îraq}{}{nom}{n}{}{Irak}

\entry{Îrlanda}{}{nom}{m}{}{Irland}

\entry{îslam}{}{nom}{n}{}{Islam}

\entry{îsot}{}{nom}{n}{}{Paprika, Chilli}

\entry{Îstanbul}{Estamol, Astamol}{nom}{n}{}{Istanbul}

\entry{îstasyon}{}{nom}{n}{}{Station, Haltestelle}

\entry{îsû}{}{nom}{}{}{$\rightarrow$ \textbf{însan}}

\entry{îşlig}{}{nom}{n}{}{Hemd}

\entry{îta}{îtya, îtîka, atika}{nom}{n}{tîya}{hier \par\noindent\textit{Bê îta!} - Komm her!}

\entry{Îtalya}{}{nom}{m}{}{Italien}

\entry{îtalyan}{}{nom}{-e}{}{Italienier/in}

\entry{îzah}{}{nom}{n}{}{Erklärung, Erläuterung}

\entry{îzah kerdene}{}{verb}{}{}{erklären, erläutern, darlegen, klarstellen}

\end{multicols}

%----------------------------------------------------------------------------------------
%	SECTION J
%----------------------------------------------------------------------------------------

\section*{J}

\begin{multicols}{2}

\entry{jan\textsuperscript{1}}{}{nom}{}{}{\textbf{$\rightarrow$ zon}}

\entry{jan\textsuperscript{2}}{zan}{nom}{n}{}{\textsuperscript{1}Schmerz, Leid \textsuperscript{2}ansteckende/virale Krankheit, Seuche $\rightarrow$ \textbf{çorr}}

\entry{japon}{}{nom}{-e}{}{Japaner/in}

\entry{japonkî}{}{nom}{m}{}{Japanisch}

\entry{Japonya}{}{nom}{m}{}{Japan}

\entry{jê}{zê}{präp}{}{sey}{\textsuperscript{1}wie, so wie \textsuperscript{2}als \textbf{$\rightarrow$ zê}}

\entry{jêde}{zêde}{adv}{}{zêde, zîyade}{sehr viel, zu viel; überschüssig \textbf{$\rightarrow$ zêde}}

\entry{jîl}{}{nom}{n}{}{\textbf{$\rightarrow$ zîl}}

\entry{jîyare}{jare}{nom}{}{}{\textbf{$\rightarrow$ zîyare}}

\entry{ju}{jü, zu}{zahl}{}{yew}{eins \textbf{$\rightarrow$ zu}}

\entry{jubînî}{jübînî, jumînî}{pron}{}{yewbînî}{einander, gegenseitig \textbf{$\rightarrow$ zubînî}}

\entry{juna}{jüna, zuna}{adv}{}{yewna}{noch ein/e/er/s, ein weiterer/s, ein andere/s \textbf{$\rightarrow$ zuna}}

\entry{jü}{}{zahl}{}{}{\textbf{$\rightarrow$ ju}}

\entry{ju kî}{zu kî}{adv}{}{}{außerdem, zudem}

\entry{jüya}{zuwa}{adj}{-ye}{ziwa}{trocken}

\entry{jüyê}{}{pron}{}{}{eine/r/s, (irgend)jemand \textbf{$\rightarrow$ zuye}}

\end{multicols}

%----------------------------------------------------------------------------------------
%	SECTION K
%----------------------------------------------------------------------------------------

\section*{K}

\begin{multicols}{2}

\entry{kablo}{}{nom}{n}{}{Kabel}

\entry{kakuç}{}{nom}{n}{}{\textbf{$\rightarrow$ çakuç}}

\entry{kal}{}{adj}{-e}{}{roh}

\entry{\.{k}al}{}{nom}{-e}{kal}{\textsuperscript{1}Greis \textsuperscript{2}Großvater; Vorfahr \textsuperscript{3}\textsc{adj} alt \textit{(Person)}, betagt}

\entry{kaleke}{}{nom}{m}{}{Seite, Flanke \textbf{$\rightarrow$ kîşte}}

\entry{\.{k}alik}{}{nom}{n}{kalik, bapîr}{Großvater, Opa}

\entry{kalîte}{}{nom}{m}{}{Qualität}

\entry{kam}{}{frg}{-e}{}{wer \par\noindent\textit{Kam o?} - Wer ist da?}

\entry{kamcîn}{kamjî}{frg}{}{}{welche/r/s}

\entry{kamera}{}{nom}{m}{}{Kamera}

\entry{kamîye}{}{nom}{m}{}{Identität}

\entry{\.{k}an}{}{adj}{-e}{kan, kehen}{alt \textit{(Gegenstück zu $\rightarrow$ \textbf{newe}})}

\entry{kanûne}{}{nom}{m}{}{$\rightarrow$ \textbf{ga\"{x}an\textsuperscript{2}}}

\entry{kar}{}{nom}{n}{}{\textsuperscript{1}Arbeit, Beschäftigung \textsuperscript{2}Beruf, Job, (Erwerbs)Tätigkeit, Gewerbe \textsuperscript{3}Sache, Aufgabe \textsuperscript{4}Nutzen, Gewinn, Ertrag, Profit \textsuperscript{4}Verb $\rightarrow$ \textbf{fîîl}}

\entry{kar kerdene}{}{Verb}{}{}{arbeiten, einer Arbeit/Beschäftigung nachgehen  \par\noindent\textit{Ti çi kar kena?} - (Als) Was arbeitest du?}

\entry{kardî}{}{nom}{m}{}{Messer}

\entry{karker}{}{nom}{-e}{}{Arbeiter/in}

\entry{karmend}{}{nom}{-e}{}{Beamte/r $\rightarrow$ \textbf{memur}}

\entry{karsaz}{}{nom}{-e}{}{Unternehmer/in; Geschäftsmann/-frau}

\entry{kartole}{}{nom}{m}{}{Kartoffel}

\entry{kas kerdene}{}{verb}{}{}{schleifen, (am Boden entlang) ziehen}

\entry{kata}{kamta}{adv}{}{}{wohin $\rightarrow$ \textbf{kotî}}

\entry{ka\"{x}ite}{}{nom}{m}{}{Papier}

\entry{kay}{}{nom}{n}{}{\textsuperscript{1}Spiel; Partie \textsuperscript{2}\textit{(Aufführung)} Vorstellung, Theaterstück \textsuperscript{3}Tanz}

\entry{kay kerdene}{}{Verb}{}{}{\textsuperscript{1}spielen \textsuperscript{2}tanzen}

\entry{kaybaz}{}{nom}{-e}{}{Schauspieler/in, Darsteller/in}

\entry{kayê komputure}{}{nom}{n}{}{Computerspiel, Videospiel}

\entry{kaykerdo\"{x}}{}{nom}{-e}{}{\textsuperscript{1}Spieler/in \textsuperscript{2}Darsteller/in \textsuperscript{3}Tänzer/in}

\entry{ke}{}{konj}{}{}{\textsuperscript{1}dass \textsuperscript{2}wenn, sobald $\rightarrow$ \textbf{eke}}

\entry{\.{k}ek}{}{nom}{n}{}{\textsuperscript{1}älterer Bruder \textsuperscript{2}\textit{Anrede für Männer}}

\entry{keleke}{}{nom}{m}{}{Floß}

\entry{kelîme}{}{nom}{m}{}{\textbf{$\rightarrow$ çekuye}}

\entry{kemer}{}{nom}{n}{}{Fels, Felsen}

\entry{kemere}{}{nom}{m}{}{Stein}

\entry{kemî}{}{adj}{}{}{\textbf{$\rightarrow$ kêmî}}

\entry{kengezere}{}{nom}{m}{}{\textbf{$\rightarrow$ a\"{x}ustose}}

\entry{kerdene}{}{Verb}{}{}{\textsuperscript{1}machen, tun \textsuperscript{2}tätigen, treiben  \par\noindent\textit{Se kena?} - Was machst du? \par\noindent\textit{Bike!} - Mach!}

\entry{kerem}{}{nom}{n}{}{Gnade \par\noindent\textit{Keremê xo ra} - Bitte \textit{(beim Bitten)} \par\noindent\textit{Kerem ke} - Bitte sehr \textit{(beim Anbieten)}}

\entry{kerge}{}{nom}{m}{}{Huhn, Hähnchen}

\entry{\.{k}err}{}{adj}{-e}{}{taub}

\entry{kes}{}{nom}{n}{}{\textsuperscript{1}jemand \textsuperscript{2}Person, Individuum \textsuperscript{3} \textit{(verneint)} niemand}

\entry{kesegan}{}{nom}{-e}{}{Schwein}

\entry{kesk}{}{adj}{-e}{}{grün}

\entry{keştî}{}{nom}{m}{}{Schiff}

\entry{\.{k}ewe}{}{adj}{-e}{}{\textsuperscript{1}blau \textsuperscript{2}\textit{(Pflanzen, Natur etc.)} grün}

\entry{key}{}{frg}{}{}{wann \par\noindent\textit{Ti key yena?} - Wann kommst du?}

\entry{keyf}{kêf, çêf}{nom}{n}{}{\textsuperscript{1}Laune, Lust, Gemüt, Stimmung \textsuperscript{2}Freude, Fröhlichkeit, Spaß}

\entry{keyfê xo ameyene}{}{verb}{}{}{Lust bekommen, Freude haben, sich freuen \par\noindent\textit{Keyfê ma yeno.} - Wir freuen uns.}

\entry{kê}{}{nom}{}{}{\textbf{$\rightarrow$ çê}}

\entry{kêber}{}{nom}{}{}{\textbf{$\rightarrow$ çêber}}

\entry{kêf}{}{nom}{n}{}{$\rightarrow$ \textbf{keyf}}

\entry{kêfwes}{keyfwes}{adj}{-e}{keyfweş}{froh, fröhlich, gutgelaunt, heiter}

\entry{kêmek}{}{nom}{n}{}{Subtraktion, Minus}

\entry{kêmî}{kemî, kêm}{adj}{-ye}{}{\textsuperscript{1}wenig \textsuperscript{2}knapp, fehlend, nicht ausreichend}

\entry{kêna}{}{nom}{}{}{\textbf{$\rightarrow$ çêna}}

\entry{kêneke}{}{nom}{}{}{\textbf{$\rightarrow$ çêneke}}

\entry{kilame}{}{nom}{m}{}{Lied, Song}

\entry{kile}{}{nom}{m}{}{\textsuperscript{1}Flamme \textsuperscript{2}\textit{(metaphorisch)} Hitze, Erdwärme}

\entry{kilê}{}{vok}{}{}{\textit{Anredefall für Frauen oder Mädchen}}

\entry{kilîme}{}{nom}{m}{}{Kelim, Teppich}

\entry{kilît}{}{nom}{n}{}{Schlüssel}

\entry{kilît kerdene}{}{verb}{}{}{abschließen, absperren, zusperren, verschließen}

\entry{kilm}{}{adj}{-e}{}{kurz}

\entry{kinc}{}{nom}{n}{}{Kleidung, Kleider}

\entry{kird}{}{nom}{-e}{}{Kird, Zaza}

\entry{kirdaskî}{}{nom}{m}{}{Kurmancî}

\entry{kirdkî}{kirdî}{nom}{m}{}{Kirdkî, Zazakî $\rightarrow$ \textbf{kirmanckî}}

\entry{kirmanc}{}{nom}{-e}{}{\textsuperscript{1}Kirmanc, Zaza \textsuperscript{2}kurdischer Alevite}

\entry{Kirmancîye}{}{nom}{m}{}{\textsuperscript{1}Kirmanctum \textsuperscript{2}Gebiet, in dem $\rightarrow$ \textbf{Kirmanc} leben}

\entry{kirmanckî}{}{nom}{m}{}{Kirmanckî, Zazakî}

\entry{kitab}{}{nom}{n}{}{Buch}

\entry{kitabxane}{}{nom}{n}{}{Bücherei, Bibliothek}

\entry{kî}{ik, iç}{konj}{}{}{auch, ebenfalls, gleichfalls}

\entry{kîlo}{}{nom}{n}{}{Kilo}

\entry{kîlogram}{}{nom}{n}{}{Kilogramm}

\entry{kîlometre}{}{nom}{n}{}{Kilometer}

\entry{kîmyon}{}{nom}{n}{}{Kümmel}

\entry{kîraze}{}{nom}{m}{}{Kirsche}

\entry{kîşte}{}{nom}{m}{}{Seite, Flanke \textbf{$\rightarrow$ kaleke}}

\entry{ko}{}{nom}{n}{}{Berg}

\entry{koçike}{}{nom}{m}{}{Löffel}

\entry{\.{k}okim}{}{adj}{-e}{kokim}{alt \textit{(Alter)}}

\entry{kol}{}{adj}{-e}{}{hornlos, unbehornt \textit{(Tiere)}}

\entry{kolî}{}{nom}{m}{}{\textsuperscript{1}Brennholz \textsuperscript{2}Paket}

\entry{komare}{}{nom}{m}{}{Republik}

\entry{komel}{}{nom}{n}{}{Gesellschaft, Community \textbf{$\rightarrow$ cemat}}

\entry{komele}{}{nom}{m}{}{Verein, Gesellschaft}

\entry{komputure}{}{nom}{m}{}{Computer, Rechner}

\entry{konser}{}{nom}{n}{}{Konzert}

\entry{kor}{}{adj}{-e}{}{blind}

\entry{kortqirane}{}{nom}{m}{}{\textbf{$\rightarrow$ êlule}}

\entry{\.{k}oskar}{\.{k}oşkar, goşkar}{nom}{-e}{koşkar}{Schuster/in, Schuhmacher/in}

\entry{kotene}{kutene}{verb}{}{kewtene}{fallen, abfallen}

\entry{kotî}{koytî}{frg}{n}{}{wo}

\entry{kotî ra}{koytî ra}{frg}{}{}{woher, von wo, woraus}

\entry{kovare}{}{nom}{m}{}{Zeitschrift, Magazin}

\entry{koz}{}{nom}{n}{}{Kleinviehstall}

\entry{kozik}{}{nom}{n}{}{Spielhaus, Kinderhütte}

\entry{ku}{}{frg}{}{}{wo, wo denn \textit{(für bewegliche Objekte)} \par\noindent\textit{Ku yo?} - Wo ist es denn?}

\entry{kul}{}{nom}{n}{}{Leid, Schmerz}

\entry{kulfet}{}{nom}{-e}{}{Familienmitglied \par\noindent\textit{çê û kulfet (Kollektivum)} - Familie, Familienmitglieder}

\entry{kultur}{}{nom}{n}{}{Kultur}

\entry{kundire}{}{nom}{m}{}{Kürbis}

\entry{kundure}{}{nom}{n}{}{(Halb)Schuh}

\entry{kurd}{}{nom}{-e}{}{Kurde/in}

\entry{Kurdîstan}{}{nom}{n}{}{Kurdistan}

\entry{kurdkî}{}{nom}{m}{}{Kurdisch}

\entry{kursî}{}{nom}{n}{}{Hocker, Stuhl}

\entry{kutik}{}{nom}{n}{}{Hund; Rüde}

\end{multicols}

%----------------------------------------------------------------------------------------
%	SECTION L
%----------------------------------------------------------------------------------------

\section*{L}

\begin{multicols}{2}

\entry{la\textsuperscript{1}}{lak}{nom}{}{}{\textsuperscript{1}Seil, Tau \textsuperscript{2}Schnur, Strick \textsuperscript{3}Strang, Band}

\entry{la\textsuperscript{2}}{}{konj}{}{}{aber, jedoch $\rightarrow$ \textbf{labelê}}

\entry{lac}{}{}{}{}{$\rightarrow$ \textbf{laz}}

\entry{lacek}{}{}{}{}{$\rightarrow$ \textbf{lazek}}

\entry{labelê}{labirê}{konj}{}{}{aber, jedoch $\rightarrow$ \textbf{la\textsuperscript{2}}}

\entry{lak}{}{nom}{n}{}{$\rightarrow$ \textbf{la\textsuperscript{1}}}

\entry{lal}{}{adj}{-e}{}{stumm (sprachlos)}

\entry{lamba}{}{nom}{m}{}{Lampe $\rightarrow$ \textbf{çila}}

\entry{lane}{lone}{nom}{m}{}{Loch}

\entry{laşêr}{laser}{n}{}{}{\textsuperscript{1}Sturzbach \textsuperscript{2}Flut}

\entry{lawike}{}{nom}{m}{}{Lied, Song $\rightarrow$ \textbf{kilame}}

\entry{lawo}{}{intrj}{}{}{\textsuperscript{1}\textit{Anredefall von}  $\rightarrow$ \textbf{lazek} \textsuperscript{2}\textit{Ausruf der Verwunderung}}

\entry{layek}{}{}{}{}{$\rightarrow$ \textbf{lazek}}

\entry{laz}{lac}{nom}{n}{laj}{Sohn}

\entry{lazek}{lacek, layek}{n}{}{lajek}{Junge, Bub}

\entry{lazê amike}{lacê amike}{nom}{n}{}{\textbf{$\rightarrow$ amiza}}

\entry{lazê apî}{lacê dedî}{nom}{n}{}{\textbf{$\rightarrow$ dereza}}

\entry{lazim}{lozim}{adj}{-e}{}{nötig, notwendig, vonnöten, erforderlich, benötigt}

\entry{lazim bîyene}{lozim bîyene}{verb}{}{}{brauchen, benötigen, bedürfen \par\noindent\textit{To rê çi lazim o?} - Was brauchst du?}

\entry{leçege}{}{nom}{m}{}{Kopftuch}

\entry{legane}{}{nom}{m}{}{Wanne, Becken}

\entry{lerze}{}{nom}{m}{lez}{Eile, Dringlichkeit  \par\noindent\textit{Lerza to çik a?} - Warum die Eile?}

\entry{lerze kerdene}{}{verb}{n}{lez kerdene}{eilen, beeilen, hetzen}

\entry{leşker}{}{nom}{-e}{}{Soldat/in $\rightarrow$ \textbf{esker}}

\entry{leşkerîye}{}{nom}{m}{}{\textsuperscript{1}Militär \textsuperscript{2}Soldatentum \textsuperscript{3}Wehrdienst, Militärdienst $\rightarrow$ \textbf{eskerîye}}

\entry{lete}{}{nom}{n}{}{Teil, Anteil}

\entry{lew}{}{nom}{n}{}{Lippe}

\entry{lew pira nayene}{}{verb}{}{}{küssen, einen Kuss geben}

\entry{lewe}{ley}{nom}{n}{}{Seite, Neben-}

\entry{lewê ... de}{leyê ... de}{zirk}{}{leweyê ... de}{neben, an der Seite von, bei}

\entry{leyê ... de}{}{zirk}{}{}{$\rightarrow$ \textbf{lewê ... de}}

\entry{leyîr}{}{nom}{-e}{}{Junges, Brut \textit{(Tier)}}

\entry{lêmin}{lêminê}{intrj}{}{}{\textit{Ausruf für Schmerz und Wehklage}}

\entry{lên}{}{nom}{n}{}{Kessel, Topf}

\entry{Licê}{}{nom}{m}{}{\textit{Stadt in Kurdistan}}

\entry{linge}{ninge}{nom}{m}{}{Fuß  $\rightarrow$ \textbf{pa}}

\entry{lîmone}{lêmune}{nom}{m}{}{Zitrone}

\entry{lître}{}{nom}{n}{}{Liter}

\entry{lobike}{lobaze, lebaze}{nom}{m}{lobî}{grüne Bohne}

\entry{lolike}{}{nom}{m}{}{Tomate $\rightarrow$ \textbf{tomatêse}}

\entry{lone}{}{nom}{m}{}{$\rightarrow$ \textbf{lane}}

\entry{loqme}{}{nom}{n}{}{\textsuperscript{1}Weihgabe\textit{ (zu religiösen Zwecken zubereitete Speise)} \textsuperscript{2}Happen \textsuperscript{3}Lokum (Süßspeise}

\entry{lovaze}{}{nom}{m}{}{$\rightarrow$ \textbf{lobike}}

\entry{lo\"{x}e}{}{nom}{m}{}{Walze \textit{(i.d.R. genutzt für Dächer oder Äcker)}}

\entry{lozim}{}{adj}{-e}{}{$\rightarrow$ \textbf{lazim}}

\entry{lüye}{}{nom}{m}{luye}{Fuchs}

\end{multicols}

%----------------------------------------------------------------------------------------
%	SECTION M
%----------------------------------------------------------------------------------------

\section*{M}

\begin{multicols}{2}

\entry{ma\textsuperscript{1}}{}{pron}{}{}{wir}

\entry{ma\textsuperscript{2}}{}{pron}{}{}{\textit{2. Fall von} $\rightarrow$ \textbf{ma\textsuperscript{1}}}

\entry{ma\textsuperscript{3}}{}{intrj}{}{}{\textit{Fragepartikel} \par\noindent\textit{Ma se bî?} - Was ist denn passiert?}

\entry{mabên}{}{nom}{n}{}{\textsuperscript{1}Mittelpunkt, Zwischenraum \textsuperscript{2}Pause, Zwischenzeit}

\entry{mabên de}{}{adv}{}{}{\textsuperscript{1}dazwischen \textsuperscript{2}zwischendurch}

\entry{maç}{}{nom}{n}{}{Match, Spiel, Partie}

\entry{maçî kerdene}{}{verb}{}{}{einen Kuss geben, ein Küsschen verteilen \textbf{$\rightarrow$ \.{p}açî kerdene}}

\entry{makeke}{maykeke}{adj}{m}{makî}{feminin, weiblich}

\entry{mal}{}{nom}{n}{}{\textsuperscript{1}Besitz, Eigentum \textsuperscript{2}Ware, Gut \textsuperscript{3}Vieh, Rindvieh}

\entry{mal û milk}{}{nom}{n}{}{Hab und Gut, Besitztum \textit{(Kollektivum)}}

\entry{malim}{}{nom}{-e}{}{Lehrer/in, Lehrkraft, Lehrpersonal \textbf{$\rightarrow$ mamosta}}

\entry{malimîye}{malimênî}{nom}{m}{}{Lehrwesen, Lehrertum, Lehrerschaft}

\entry{Mamekîye}{Kalan}{nom}{n}{}{\textit{Stadt in Dêrsim (Tunceli-Stadt)}}

\entry{mamosta}{}{nom}{n, m}{}{Lehrer/in, Lehrkraft, Lehrpersonal \textbf{$\rightarrow$ malim}}

\entry{mana}{}{nom}{m}{}{Bedeutung, Sinn, Relevanz}

\entry{manga}{}{nom}{m}{}{Kuh}

\entry{manî}{}{nom}{n}{}{Hindernis, Blockade}

\entry{mar}{mor}{nom}{-e}{}{Schlange}

\entry{markete}{}{nom}{m}{}{Supermarkt}

\entry{marte}{}{nom}{m}{adare}{März}

\entry{masa}{}{nom}{m}{}{Tisch}

\entry{mase}{masi, mose}{nom}{n}{}{Fisch}

\entry{mast}{most}{nom}{n}{}{Joghurt}

\entry{matematîk}{}{nom}{n}{}{Mathematik}

\entry{Mazgêrd}{}{nom}{n}{}{\textit{Stadt in Dêrsim (türk. Mazgirt)}}

\entry{may û pî}{}{nom}{zh}{}{Eltern}

\entry{maye}{moye}{nom}{m}{}{Mutter}

\entry{maydanoz}{}{nom}{n}{}{Petersilie}

\entry{mayîne}{}{nom}{m}{}{Stute}

\entry{mecbur}{}{adj}{-e}{}{zwingend, zwangsläufig, gezwungen}

\entry{mekteb}{}{nom}{n}{}{Schule $\rightarrow$ \textbf{wendegeh}}

\entry{mektube}{}{nom}{m}{}{Brief}

\entry{memur}{}{nom}{-e}{mamur}{Beamte/r $\rightarrow$ \textbf{karmend}}

\entry{menajer}{}{nom}{-e}{}{Manager/in}

\entry{mendene}{}{verb}{}{}{bleiben}

\entry{merde}{}{nom}{-îye}{}{Tote/r; tot}

\entry{merdene\textsuperscript{1}}{}{verb}{}{}{sterben}

\entry{merdene\textsuperscript{2}}{}{nom}{m}{merg}{Tod; Sterben}

\entry{mereke}{}{nom}{m}{}{Scheune, Heuboden}

\entry{merge}{}{nom}{m}{}{Weideland, Aue, Prärie}

\entry{merre}{merri}{nom}{n}{}{Maus}

\entry{mesele}{}{nom}{m}{}{\textsuperscript{1}Ereignis, Vorfall \textsuperscript{2}Geschichte, Erzählung \textsuperscript{3}Angelegenheit, Sache}

\entry{meslek}{}{nom}{n}{}{Beruf, Berufsstand, Gewerbe}

\entry{meşte}{}{adv}{}{}{morgen $\rightarrow$ \textbf{sora}}

\entry{metre}{}{nom}{n}{}{Meter}

\entry{metro}{}{nom}{m}{}{U-Bahn}

\entry{mewsîm}{}{nom}{n}{}{Jahreszeit}

\entry{meyman}{}{nom}{-e}{mêman}{Gast, Besucher/in}

\entry{meymanperwer}{}{adj}{-e}{mêmanperwer}{gastfreundlich}

\entry{meywe}{}{nom}{m}{}{\textsuperscript{1}Obst, Früchte \textsuperscript{2}Frucht}

\entry{meyxane}{}{nom}{n}{}{Kneipe, Schenke, Wirtshaus}

\entry{mezele}{}{nom}{m}{}{Grab}

\entry{mêrde}{mêrdi}{nom}{n}{}{Ehemann, Gatte}

\entry{mêrik}{}{nom}{n}{}{Mann, Kerl}

\entry{mêse}{}{nom}{m}{}{\textsuperscript{1}Fliege \textsuperscript{2}Biene}

\entry{mi}{}{pron}{}{}{\textit{2. Fall von} $\rightarrow$ \textbf{ez}}

\entry{milet}{}{nom}{m}{}{\textsuperscript{1}Volk \textsuperscript{2}Leute}

\entry{min}{}{pron}{}{}{$\rightarrow$ \textbf{mi}}

\entry{minetdar}{}{adj}{-e}{}{$\rightarrow$ \textbf{minetkar}}

\entry{minete}{}{nom}{m}{}{\textsuperscript{1}Dank, Dankbarkeit \textsuperscript{2}Gebet, Bittgebet}

\entry{minetkar}{}{adj}{-e}{}{dankbar, verbunden $\rightarrow$ \textbf{sipasdar}}

\entry{miradin}{}{adj}{-e}{mirûzin}{beleidigt, schmollend, grämig, mürrisch $\rightarrow$ \textbf{mirozin}}

\entry{miradîyayene}{}{verb}{}{}{beleidigt sein, schmollen, jmd. böse sein  $\rightarrow$ \textbf{heredîyayene}}

\entry{miradnayene}{}{verb}{}{}{vergrämen, verärgern, verstimmen  $\rightarrow$ \textbf{herednayene}}

\entry{mird}{}{adj}{-e}{}{satt}

\entry{mirozin}{}{adj}{-e}{mirûzin}{beleidigt, schmollend, grämig, mürrisch $\rightarrow$ \textbf{miradin}}

\entry{mix}{}{nom}{n}{}{Nagel}

\entry{miz}{mic}{nom}{n}{mij}{Nebel}

\entry{mizin}{micin}{adj}{-e}{}{neblig}

\entry{mîlçik}{}{nom}{-e}{}{Vogel}

\entry{mîr}{}{nom}{n}{}{Teig}

\entry{mîre}{}{nom}{n}{}{Fürst}

\entry{mîyan}{mîya}{nom}{n}{}{Mitte, Mittelpunkt $\rightarrow$ \textbf{werte}}

\entry{mîyan de}{mîya de}{adv}{}{}{\textsuperscript{1}(in)mitten, mittendrin, dazwischen, zwischendrin \textsuperscript{2}gelegentlich $\rightarrow$ \textbf{werte de}}

\entry{mîyane}{}{nom}{n}{}{Lende, Taille, Kreuz}

\entry{mîyanneteweyî}{}{adj}{-ye}{}{international, zwischenstaatlich}

\entry{mîyaz}{}{nom}{n}{}{$\rightarrow$ \textbf{nîyaz}}

\entry{mîye}{}{nom}{m}{}{Schaf}

\entry{mobîlîte}{}{nom}{m}{}{Mobilität}

\entry{mor}{}{nom}{-e}{}{$\rightarrow$ \textbf{mar}}

\entry{mordem\textsuperscript{1}}{}{nom}{n}{merdim}{\textsuperscript{1}Mann \textsuperscript{2}Mensch \textsuperscript{3}man}

\entry{mordem\textsuperscript{2}}{}{nom}{-e}{merdim}{Verwandte/r}

\entry{mordemek}{mormek}{nom}{n}{}{$\rightarrow$ \textbf{mêrik}}

\entry{mormek}{}{nom}{n}{}{$\rightarrow$ \textbf{mordemek}}

\entry{mose}{}{nom}{n}{}{$\rightarrow$ \textbf{mase}}

\entry{most}{}{nom}{n}{}{$\rightarrow$ \textbf{mast}}

\entry{mozik}{}{nom}{-e}{}{Kalb}

\entry{muhendis}{}{nom}{-e}{}{Ingenieur/in}

\entry{muhîm}{}{adj}{-e}{}{wichtig, bedeutsam}

\entry{mume}{}{nom}{m}{}{Kerze}

\entry{Munzur}{}{nom}{n}{}{$\rightarrow$ \textbf{Muzir}}

\entry{murîye}{}{nom}{m}{}{Birne}

\entry{muşterî}{müşterî}{nom}{-ye}{}{Kunde/in, Kundschaft}

\entry{muxtar}{}{nom}{-e}{}{Dorfvorsitzende/r, Dorfvorsteher/in, Gemeindevorsteher/in}

\entry{muze}{}{nom}{m}{}{Banane}

\entry{Muzir}{Munzur, Munzir}{nom}{n}{Mizur}{\textsuperscript{1}\textit{Fluss in Dêrsim} \textsuperscript{2}\textit{männl. Vorname}}

\entry{muzîk}{}{nom}{n}{}{Musik}

\end{multicols}

%----------------------------------------------------------------------------------------
%	SECTION N
%----------------------------------------------------------------------------------------

\section*{N}

\begin{multicols}{2}

\entry{na}{}{dem}{m}{}{diese, die/das da}

\entry{naca}{}{adv}{}{}{\textbf{$\rightarrow$ naza}}

\entry{nacnîye}{naçike, najnîye}{nom}{m}{}{Frau des Onkels väterlicherseits}

\entry{naçike}{}{nom}{m}{}{$\rightarrow$ \textbf{nacnîye}}

\entry{name}{nami}{nom}{n}{}{name \par\noindent\textit{Namê to çik o?} - Wie heißt du?}

\entry{nan}{non, nû}{nom}{n}{}{\textsuperscript{1}Brot \textsuperscript{2}Essen}

\entry{nan û awe}{non û awe}{nom}{m}{}{Essen und Getränke, Speis und Trank, Speisen $\rightarrow$ \textbf{nan û sole}}

\entry{nan û sole}{non û sole}{nom}{m}{}{\textsuperscript{1}Essen, Nahrung \textsuperscript{2}Grundbedürfnis}

\entry{nane}{none}{nom}{m}{}{Laib Brot, Brotlaib}

\entry{nara}{}{adv}{}{}{diesmal, dieses Mal \textbf{$\rightarrow$ na rey}}

\entry{nas}{}{nom}{-e}{}{Bekannte/r \par\noindent\textit{Nas û dost} - Freunde und Bekannte \textit{(Kollektivum)}}

\entry{nas kerdene}{}{verb}{}{}{\textsuperscript{1}kennen \textsuperscript{2}kennenlernen}

\entry{nasname}{}{nom}{n}{}{\textsuperscript{1}Ausweis \textsuperscript{2}Identität}

\entry{nat}{}{adv}{}{}{hierher, bis hierhin}

\entry{naxilqirane}{}{nom}{m}{}{\textbf{$\rightarrow$ payîzo peyên}}

\entry{naxir}{}{nom}{n}{}{Rinderherde}

\entry{naye}{}{dem}{}{}{\textit{2. Fall von}$\rightarrow$ \textbf{na}}

\entry{naza}{naca}{adv}{}{noca}{hier, an diesem Ort}

\entry{Nazmîya}{}{nom}{m}{}{\textbf{$\rightarrow$ Qisle}}

\entry{nebat}{}{nom}{n}{}{Pflanze}

\entry{neheq}{neq}{adj}{-e}{}{\textsuperscript{1}unrecht, ungerecht, ungerechtfertigt \textsuperscript{2}unbegründet, unberechtigt}

\entry{nergîze}{}{nom}{m}{}{Narzisse}

\entry{nerm}{}{adj}{-e}{}{weich, sanft}

\entry{netewe}{}{nom}{m}{}{National}

\entry{neteweperest}{}{nom}{-e}{}{Nationalist/in}

\entry{new}{}{zahl}{}{}{neun}

\entry{neway}{}{zahl}{}{}{neunzig}

\entry{newe}{}{adj}{-îye}{}{neu}

\entry{newroze}{}{nom}{m}{}{Newroz}

\entry{ney}{}{dem}{}{}{\textit{2. Fall von}$\rightarrow$ \textbf{no}}

\entry{nê\textsuperscript{1}}{}{ptkl}{}{ney}{\textsuperscript{1}nein \textsuperscript{2}nicht}

\entry{nê\textsuperscript{2}}{}{dem}{}{}{2. Fall von $\rightarrow$ \textbf{no}}

\entry{nê\textsuperscript{3}}{}{dem}{zh}{}{diese, die da}

\entry{nêjdî}{}{adv}{}{}{$\rightarrow$ \textbf{nêzdî}}

\entry{nêke}{}{konj}{}{}{ansonsten, sonst, andernfalls}

\entry{nêm}{}{adj}{-e}{}{halb}

\entry{nême}{}{nom}{n}{}{Hälfte}

\entry{nêrî}{}{adj}{-ye}{}{maskulin, männlich}

\entry{nêwes}{}{adj}{-e}{nêweş}{\textsuperscript{1}krank \textsuperscript{2}\textsc{nom} Patient/in}

\entry{nêwesxane}{xestexane}{nom}{n}{nêweşxane}{Krankenhaus, Hospital, Klinik}

\entry{nêzdî}{nêjdî}{adv}{}{}{\textsuperscript{1}nah \textsuperscript{2}\textsc{nom} Nähe}

\entry{nika}{}{adv}{}{}{jetzt, nun, derzeit, gerade (eben)}

\entry{nîjad}{}{nom}{n}{}{Rasse}

\entry{nîjadperest}{}{nom}{-e}{}{Rassist/in}

\entry{nînan}{nînû, nînê}{dem}{}{}{\textit{2. Fall von}$\rightarrow$ \textbf{nê}\textsuperscript{3}}

\entry{nîsane}{}{nom}{m}{}{April}

\entry{nîske}{}{nom}{m}{}{Linse}

\entry{nîya}{}{adv}{}{}{so, in dieser Art, auf diese Art, in dieser Art und Weise}

\entry{nîyadayene}{}{verb}{}{}{gucken, schauen \par\noindent\textit{Nîyade!} - Schau!}

\entry{nîyaz}{mîyaz}{nom}{n}{}{\textsuperscript{1}Weihegabe \textsuperscript{2}\textit{Teigspeise, die an religiösen Feiertagen zubereitet wird}}

\entry{no}{}{dem}{n}{}{dieser, der/das da}

\entry{nobedar}{}{nom}{-e}{}{Wächter/in, Wache}

\entry{non}{}{nom}{}{}{$\rightarrow$ \textbf{nan}}

\entry{none}{}{nom}{}{}{$\rightarrow$ \textbf{nane}}

\entry{normal}{}{adj}{-e}{}{normal, üblich, gewöhnlich}

\entry{normal de}{}{adv}{}{}{normalerweise, üblicherweise, für gewöhnlich}

\entry{Norweç}{}{nom}{n}{}{Norwegen}

\entry{not}{}{nom}{n}{}{Notiz, Anmerkung}

\entry{nover}{}{nom}{n}{}{diese Seite, dieses Ufer, diesseits}

\entry{nuke}{}{nom}{m}{}{Kichererbse}

\entry{numre}{numara}{nom}{n}{}{Nummer}

\entry{nusnayene}{}{verb}{}{}{$\rightarrow$ \textbf{nuştene}}

\entry{nuşte}{yazî}{nom}{n}{}{\textsuperscript{1}Schreiben, Schriftstrück \textsuperscript{2}Schrift \textsuperscript{3}Artikel, Aufsatz, Essay}

\entry{nuştene}{nusnayene}{verb}{}{}{schreiben, ein Schriftstück verfassen, beschriften}

\entry{nuşto\"{x}}{}{nom}{-e}{}{\textsuperscript{1}Schriftsteller/in, Schreiber/in \textsuperscript{2}Autor/in, Verfasser/in}

\end{multicols}

%----------------------------------------------------------------------------------------
%	SECTION O
%----------------------------------------------------------------------------------------

\section*{O}

\begin{multicols}{2}

\entry{o\textsuperscript{1}}{}{pron}{m}{}{er}

\entry{o\textsuperscript{2}}{}{dem}{m}{}{jener; der/das dort}

\entry{od}{}{nom}{n}{}{$\rightarrow$ \textbf{wad}}

\entry{oda}{weda}{nom}{m}{}{Zimmer, Raum}

\entry{ofîs}{}{nom}{n}{}{\textsuperscript{1}Büro, Dienststelle \textsuperscript{2}Kanzlei, Praxis $\rightarrow$ \textbf{buro}}

\entry{olvoz}{}{nom}{-e}{}{$\rightarrow$ \textbf{albaz}}

\entry{onca}{wenca}{adv}{}{}{\textsuperscript{1}allerdings, jedoch \textsuperscript{2}erneut, wieder, abermals}

\entry{ondêr}{}{adj}{-e}{}{verdammt, verflucht, verflixt}

\entry{ontene}{}{verb}{}{antene}{ziehen}

\entry{organîk}{}{adj}{-e}{}{\textsuperscript{1}organisch \textsuperscript{2}bio, Bio-}

\entry{orte}{werte}{nom}{n}{}{Mitte, Mittelpunkt $\rightarrow$ \textbf{mîyan}}

\entry{orte de}{werte de}{adv}{}{}{\textsuperscript{1}(in)mitten, mittendrin, dazwischen, zwischendrin \textsuperscript{2}gelegentlich $\rightarrow$ \textbf{mîyan de}}

\entry{osin}{}{nom}{n}{}{$\rightarrow$ \textbf{asin}}

\entry{ospor}{}{nom}{-e}{}{$\rightarrow$ \textbf{aspar}}

\entry{osta}{westa}{nom}{-ye}{}{Meister}

\entry{ostor}{}{nom}{-e}{}{$\rightarrow$ \textbf{astor}}

\entry{otobuse}{}{nom}{m}{}{Bus, Autobus}

\entry{owe}{}{nom}{n}{}{$\rightarrow$ \textbf{awe}}

\entry{ox}{}{intrj}{}{} {\textit{Ausruf der Freude oder des Genusses}}

\entry{o\"xir}{}{nom}{n}{}{\textsuperscript{1}Reise, Aufbruch \par\noindent\textit{O\"xir bo!} - Tschüss! \par\noindent\textsuperscript{2}Glück, Glücksbringer}

\entry{ozan}{hozan}{nom}{-e}{}{\textsuperscript{1}Dichter/in, Poet/in $\rightarrow$ \textbf{saîr} \textsuperscript{2}Barde, Sänger/in $\rightarrow$ \textbf{deyîrbaz}}

\entry{ozr}{}{nom}{n}{}{Entschuldigung, Abbitte}

\entry{ozr waştene}{}{verb}{}{}{entschuldigen, um Entschuldigung bitten \par\noindent \textit{Ez ozrê to wazeno.} - Ich bitte (dich) um Entschuldigung.}

\end{multicols}

%----------------------------------------------------------------------------------------
%	SECTION P
%----------------------------------------------------------------------------------------

\section*{P}

\begin{multicols}{2}

\entry{pa}{}{nom}{n}{}{Fuß  $\rightarrow$ \textbf{linge}}

\entry{\.{p}açî kerdene}{}{verb}{}{}{einen Kuss geben, ein Küsschen verteilen $\rightarrow$ \textbf{maçî kerdene}}

\entry{pak}{}{adj}{-e}{}{rein, sauber, reinlich, blank}

\entry{paka}{}{adv}{}{}{unbewölkt, klar (Himmel)}

\entry{pakête}{}{nom}{m}{}{Paket}

\entry{\.{p}anc}{\.{p}onc}{zahl}{}{panc}{fünf}

\entry{\.{p}ancas}{\.{p}oncas}{zahl}{}{pancas}{fünfzig}

\entry{panel}{}{nom}{n}{}{Podiumsdiskussion, Panel}

\entry{pano}{}{nom}{n}{}{\textsuperscript{1}Notizbrett, Pinnwand \textsuperscript{2}Tafel}

\entry{pantolî}{pantorî}{nom}{zh}{}{Hose}

\entry{par}{}{adv}{}{}{letztes Jahr}

\entry{pardon}{}{intrj}{}{}{Pardon, Entschuldigung}

\entry{\.{p}are}{}{nom}{a}{}{\textsuperscript{1}Teil, Anteil \textsuperscript{2}Beteiligung}

\entry{\.{p}arek}{}{nom}{n}{}{Division, Teilung}

\entry{parke}{}{nom}{m}{}{Park}

\entry{parse kerdene}{}{verb}{}{}{betteln}

\entry{parsek}{}{nom}{-e}{}{Bettler/in}

\entry{partîye}{}{nom}{m}{}{Partei}

\entry{parzun}{}{nom}{n}{}{Sieb}

\entry{pasa}{}{nom}{n}{paşa}{Pascha}

\entry{\.{p}aseme}{\.{p}oseme}{nom}{n}{panşeme}{Donnerstag}

\entry{paskule}{paskul, paskil}{nom}{m}{}{Tritt, Fußtritt}

\entry{paştî}{poştî}{nom}{m}{piştî}{Rücken}

\entry{patron}{}{nom}{-e}{}{Vorgesetzte/r, Chef, Boss $\rightarrow$ \textbf{serkar}}

\entry{pay kerdene}{}{verb}{}{}{anziehen \textit{(mit den Füßen)}}

\entry{pay ra}{}{adv}{}{}{\textsuperscript{1}im Stehen, stehend, aufrecht \textsuperscript{2}zu Fuß}

\entry{payîz}{}{nom}{n}{}{Herbst}

\entry{payîze\textsuperscript{1}}{}{nom}{m}{teşrîna verêne}{Oktober}

\entry{payîze\textsuperscript{2}}{}{nom}{n}{}{\textit{eine im Herbst gesäte Getreideart}}

\entry{payîzî}{}{adv}{}{}{im Herbst, in der Herbstzeit}

\entry{payîzo peyên}{}{nom}{n}{teşrîna peyêne}{November}

\entry{payîzo virên}{}{nom}{n}{}{September $\rightarrow$ \textbf{êlule}}

\entry{payîzo wertên}{}{nom}{n}{teşrîna verêne}{Oktober $\rightarrow$ \textbf{payîze\textsuperscript{1}}}

\entry{peh}{}{intrj}{}{}{\textit{Ausruf für Spott oder Ekel}}

\entry{pelge}{}{nom}{m}{}{Blatt}

\entry{pencere}{}{nom}{n}{}{Fenster}

\entry{pendîr}{penîr}{nom}{n}{}{Käse}

\entry{.{p}e\.{p}o}{}{intrj}{}{}{\textit{Ausruf für Verwunderung oder Staunen}}

\entry{perde}{}{nom}{n}{}{Vorhang}

\entry{pere}{}{nom}{n}{}{Geld}

\entry{perey}{}{adv}{}{}{vorgestern, vor zwei Tagen}

\entry{perocîye}{}{nom}{m}{}{$\rightarrow$ \textbf{perozîye}}

\entry{perojîye}{}{nom}{m}{}{$\rightarrow$ \textbf{perozîye}}

\entry{peroz\textsuperscript{1}}{peroc}{nom}{n}{peroj}{Mittag}

\entry{peroz\textsuperscript{2}}{peroc}{adv}{}{peroj}{tagsüber \par\noindent\textit{Peroz û pesewe} - Tag und Nacht}

\entry{perozîye}{perojîye, perocîye}{nom}{m}{perojîye}{Mittagessen}

\entry{perr}{}{nom}{n}{}{Flügel}

\entry{pers}{}{nom}{n}{}{Frage \par\noindent\textit{Persê mi est o.} - Ich habe eine Frage.}

\entry{pers kerdene}{}{verb}{}{}{fragen, Fragen stellen}

\entry{persayene}{}{verb}{}{}{fragen, erfragen, nachfragen}

\entry{perwerde}{}{nom}{n}{}{\textsuperscript{1}Bildung, Ausbildung, Fortbildung \textsuperscript{2}Erziehung, Lehre}

\entry{pes}{}{nom}{n}{}{Kleinviehherde}

\entry{pesewe}{}{adv}{}{peşewe}{nachts, in der Nacht}

\entry{pey}{}{nom}{n}{}{hinten, Hinter-}

\entry{pey de}{}{adv, post}{}{}{\textsuperscript{1}dahinter, hinendrauf \textsuperscript{2}danach}

\entry{peyê cû}{peyê coy}{adv}{}{}{später, danach, im Nachhinein}

\entry{peyê ... de}{}{zirk}{}{}{hinter, dahinter}

\entry{peynîye}{}{nom}{m}{}{Ende, Schluss}

\entry{peyser}{}{adv}{}{}{zurück, rückwärts}

\entry{pezkovî}{}{nom}{m}{}{Bergziege $\rightarrow$ \textbf{biza \.{k}irre}}

\entry{pê}{}{präp}{}{}{mit, mit Hilfe, anhand, durch}

\entry{pêleke}{}{nom}{m}{}{Stoffbeutel, Schatzbeutel}

\entry{pêrar}{}{nom}{n}{}{vorletztes Jahr, vor zwei Jahren}

\entry{Pêrtage}{}{nom}{m}{}{\textit{Stadt in Dêrsim (türk. Pertek)}}

\entry{pêro}{}{pron}{}{}{alle, alles $\rightarrow$ \textbf{têde}}

\entry{pêro pîya}{}{adv}{}{}{insgesamt, alle(s) zusammen}

\entry{pêrune}{}{pron}{}{}{\textit{2. Fall von} $\rightarrow$ \textbf{pêro}}

\entry{pêsare}{}{nom}{n}{}{\textit{Ein zu religiösen Feiertagen zubereitetes Brot}}

\entry{pêsayene}{}{verb}{}{}{garen, kochen, gekocht werden}

\entry{pêt}{}{adj}{-e}{}{\textsuperscript{1}stark, kräftig, stramm \textsuperscript{2}stabil, robust, fest \textsuperscript{3}fleißig, tüchtig \textsuperscript{4}schnell, rasch \textsuperscript{5}wohlhabend, reich}

\entry{pêtage}{}{nom}{m}{}{\textsuperscript{1}Scheune \textsuperscript{2}Bienenstock \textsuperscript{3}Honigwabe}

\entry{Pilemurîye}{}{nom}{m}{}{\textit{Stadt in Dêrsim (türk. Pülümür)}}

\entry{pird}{}{nom}{n}{}{Brücke}

\entry{pirnike}{}{nom}{m}{}{Nase}

\entry{pirr}{}{adj}{-e}{}{voll}

\entry{pirr kerdene}{}{verb}{}{}{füllen, befüllen, voll machen}

\entry{pisînge}{}{nom}{m}{}{Katze}

\entry{pî}{}{nom}{n}{}{Vater}

\entry{pîl}{}{adj}{-e}{}{alt, groß \textit{(vom Alter her oder figurativ)}}

\entry{pînayene}{}{verb}{}{}{$\rightarrow$ \textbf{pîtene}}

\entry{pîr}{}{nom}{n}{}{Pîr \textit{(Religionsführer bei Aleviten)} \par\noindent\textit{Pîr bo!} - Wirklich!}

\entry{pîrike}{}{nom}{m}{pîrike, dapîre}{\textsuperscript{1}Großmutter, Oma \textsuperscript{2}Spinne}

\entry{pîroz}{}{adj}{-e}{}{$\rightarrow$ \textbf{bimbarek}}

\entry{pîtene}{pînayene}{verb}{}{}{warten \par\noindent\textit{Bipîye!} - Warte!}

\entry{pîya}{}{adv}{}{}{gemeinsam, zusammen, beisammen, gemeinschaftlich}

\entry{pîyaz}{}{nom}{n}{}{Zwiebel}

\entry{pîyo}{}{vok}{}{}{\textit{Anredefall von} $\rightarrow$ \textbf{pî}}

\entry{pîze}{}{nom}{n}{}{Bauch}

\entry{pîzza}{}{nom}{m}{}{Pizza}

\entry{polîs}{}{nom}{n}{}{\textsuperscript{1}Polizei \textsuperscript{2}Polizist/in}

\entry{polîtîka}{}{nom}{m}{}{Politik {$\rightarrow$ \textbf{sîyaset}}}

\entry{\.{p}onc}{}{zahl}{}{$\rightarrow$ \textbf{\.{p}anc}}

\entry{\.{p}oncas}{}{zahl}{}{$\rightarrow$ \textbf{\.{p}ancas}}

\entry{postal}{}{nom}{n}{}{Schuh}

\entry{poste}{postik}{nom}{n}{}{\textsuperscript{1}Haut \textsuperscript{2}Leder $\rightarrow$ \textbf{çerme}}

\entry{poştî}{}{nom}{m}{}{$\rightarrow$ \textbf{paştî}}

\entry{problem}{}{nom}{n}{}{Problem}

\entry{profesor}{}{nom}{-e}{}{Professor/in}

\entry{program}{}{nom}{n}{}{Programm}

\entry{programviraşto\"{x}}{}{nom}{-e}{}{Programmierer/in, Coder/in}

\entry{puk}{}{nom}{n}{}{Sturm, Gestöber}

\entry{pukeleke}{}{nom}{m}{}{Schneesturm, Gestöber, Blizzard}

\entry{Pulur}{Vacu\"{xe}}{nom}{n}{}{\textit{Stadt in Dêrsim (türk. Ovacık)}}

\entry{pungal}{}{nom}{n}{}{Hühnerstall}

\end{multicols}

%----------------------------------------------------------------------------------------
%	SECTION Q
%----------------------------------------------------------------------------------------

\section*{Q}

\begin{multicols}{2}

\entry{qal}{}{nom}{n}{qale}{Gerede, Ausdruck, Ausspruch}

\entry{qal kerdene}{}{verb}{}{}{erzählen, (be)sprechen}

\entry{qalind}{}{adj}{-e}{}{\textsuperscript{1}dick, prall \textsuperscript{2}\textit{(Haare)} dicht, buschig}

\entry{qapax}{}{nom}{n}{}{Deckel, Verschluss, Klappe}

\entry{qat}{}{nom}{n}{}{\textsuperscript{1}Stock, Stockwerk, Etage, Geschoss \textsuperscript{2}Schicht, Stufe}

\entry{qatir}{}{nom}{n}{qantir}{Maultier}

\entry{qatix}{}{nom}{n}{}{Milchprodukte}

\entry{qawune}{}{nom}{m}{}{Honigmelone}

\entry{qayît kerdene}{}{Verb}{}{}{\textsuperscript{1}ansehen, anschauen, betrachten, zuschauen, zusehen \textsuperscript{2}beobachten, beaufsichtigen \textsuperscript{3}aufpassen, betreuen, kümmern \par\noindent\textit{Qayîtê xo bike!} - Pass auf dich auf!}

\entry{qe}{}{adv}{}{}{$\rightarrow$ \textbf{qet}}

\entry{qedênayene}{}{verb}{}{}{beenden, fertig machen, fertig stellen, zu Ende bringen, ausmachen}

\entry{qedîyayene}{}{verb}{}{}{enden, fertig werden aus sein, ausgehen, auslaufen}

\entry{qelb}{}{nom}{n}{}{$\rightarrow$ \textbf{zerrî}}

\entry{qeleme}{}{nom}{m}{}{Stift}

\entry{qesab}{}{nom}{-e}{}{Metzger/in, Fleischer/in}

\entry{qese}{}{nom}{n}{qisa}{\textsuperscript{1}Spruch, Wort \textsuperscript{2}Rede}

\entry{qesey kerdene}{qeşî kerdene}{verb}{}{qisey kerdene}{reden, sprechen \par\noindent\textit{Ma kirmanckî qesey kenîme} - Wir sprechen Kirmanckî}

\entry{qet}{qe}{adv}{}{}{\textsuperscript{1}nie, niemals; je, jemals \textsuperscript{2}gar nicht}

\entry{qet çî}{}{pron}{n}{}{nichts, gar nichts}

\entry{qet kes}{}{pron}{n}{}{niemand, gar niemand}

\entry{qewe}{}{nom}{n}{}{Kaffee}

\entry{qewet}{}{nom}{n}{}{Kraft, Stärke \par\noindent\textit{Qewet bo!} - Frohes Schaffen!}

\entry{qewexane}{}{nom}{n}{}{\textsuperscript{1}Café, Kaffeehaus \textsuperscript{2}Spielkneipe, Wirtshaus}

\entry{qey}{}{frg}{}{}{wozu, weshalb, weswegen}

\entry{qeysî}{}{nom}{m}{}{Aprikose}

\entry{qijkek}{qişkek, qickek}{adj}{-e}{}{klein}

\entry{qilêr}{}{nom}{n}{}{Schmutz, Dreck}

\entry{qilêrin}{}{adj}{-e}{}{schmutzig, dreckig, verschmutzt, verdreckt}

\entry{qisim}{}{nom}{n}{}{\textsuperscript{1}Teil, Part, Abschnitt \textsuperscript{2}Sektion, Abteilung \textsuperscript{3}Kapitel, Folge}

\entry{Qisle}{Nazmîya}{nom}{n}{}{\textit{Stadt in Dêrsim (türk. Nazımiye)}}

\entry{qiz}{qic}{adj}{-e}{}{jünger, klein \textit{(Alter)}}

\entry{qîymet}{}{nom}{n}{}{\textsuperscript{1}Wert, Stellenwert \textsuperscript{2}Würde, Dignität, Achtung}

\entry{qîymetin}{}{adj}{-e}{}{\textsuperscript{1} wertvoll, werthaltig \textsuperscript{2}ehrbar, kostbar, edel \textsuperscript{3}verdienstlich, tüchtig}

\entry{qom}{}{nom}{n}{qewm}{Volk $\rightarrow$ \textbf{sar}}

\entry{qonax}{}{nom}{n}{}{Residenz, Herberge}

\entry{qoltuxe}{}{nom}{m}{}{Sofa; Sessel}

\entry{qorre}{}{nom}{m}{}{\textsuperscript{1}Oberschenkel \textsuperscript{2}Bein}

\entry{qurban}{}{nom}{-e}{}{\textsuperscript{1}Opfer \textsuperscript{2}Opfergabe \textsuperscript{3}\textsc{intrj}\textit{Ausdruck der Zuneigung}}

\entry{qusur}{}{nom}{n}{}{\textsuperscript{1}Fehler, Fauxpas \textsuperscript{2}Mangel, Makel, Missstand \par\noindent\textit{Qusurî de nîyamede} - Entschuldige}

\entry{qutîye}{qutî}{nom}{m}{}{\textsuperscript{1}Box, Schachtel, Büchse \textsuperscript{2}Kiste, Kasten, Truhe}

\end{multicols}

%----------------------------------------------------------------------------------------
%	SECTION R
%----------------------------------------------------------------------------------------

\section*{R}

\begin{multicols}{2}

\entry{ra\textsuperscript{1}}{}{post}{}{}{von, aus}

\entry{ra\textsuperscript{2}}{}{präp}{}{}{hin, gen, nach}

\entry{ra\textsuperscript{3}}{}{post}{}{}{\textbf{$\rightarrow$ a\textsuperscript{3}}}

\entry{rajî}{}{adj}{-ye}{}{\textbf{$\rightarrow$ razî}}

\entry{rakerdene}{}{verb}{}{}{\textsuperscript{1}öffnen, aufmachen \textsuperscript{2}anmachen, anschalten \textbf{$\rightarrow$ yakerdene}}

\entry{rame}{}{nom}{m}{rehme}{\textsuperscript{1} Segen, Überfluss; Bamherzigkeit \textsuperscript{2}Regen \textbf{$\rightarrow$ şîlîye}}

\entry{ramitene}{}{verb}{}{}{fahren}

\entry{ramito\"{x}}{}{nom}{-e}{}{Fahrer/in}

\entry{raşt}{rast}{adj}{-e}{}{\textsuperscript{1}richtig, rechtens, stimmig, korrekt \textsuperscript{2}echt, real, wahr \textsuperscript{3}gerade, aufrecht \textsuperscript{4}rechts}

\entry{raştikên}{}{adj}{-e}{}{wirklich, wahrhaftig, waschecht}

\entry{raştîye}{rastîye}{nom}{m}{}{\textsuperscript{1}Richtigkeit, Korrektheit, Exaktheit \textsuperscript{2}Wahrheit, Realität, Echtheit, Wahrhaftigkeit \textsuperscript{3}Geradheit, Aufrichtigkeit}

\entry{raver}{}{adv}{}{aver}{\textsuperscript{1}nach vorne, voran, vorwärt, voraus, vorweg \textsuperscript{2}vorher, im Voraus \textsuperscript{3}erst, zuerst, zuvor, zu Beginn}

\entry{raye}{}{nom}{m}{rayîr}{\textsuperscript{1}Weg, Pfad \textsuperscript{2}Durchweg, Durchgang}

\entry{rayber}{}{nom}{-e}{}{\textsuperscript{1}Wegweiser/in, Fremdenführer/in, Führer/in \textsuperscript{2} \textit{Religiöser Posten bei Aleviten} \textsuperscript{3}Handbuch, Leitfaden}

\entry{raywan}{}{nom}{-e}{}{\textsuperscript{1}Reisende/r \textsuperscript{2}Passagier, Fahrgast}

\entry{raywanîye}{}{nom}{m}{}{Reise}

\entry{razî}{rajî}{adj}{-ye}{}{zufrieden, bereitwillig, billigend \par\noindent\textit{Heq to ra razî bo} - Vergelt's Gott}

\entry{ref}{}{nom}{n}{}{\textsuperscript{1}Regal \textsuperscript{2}Vogelschwarm}

\entry{rejîsor}{}{nom}{-e}{}{Regisseur/in}

\entry{rehet}{}{adj}{-e}{}{\textsuperscript{1} angenehm, bequem, gemütlich \textsuperscript{2}einfach, leicht, mühelos}

\entry{remayene}{}{verb}{}{}{fliehen}

\entry{repvaj}{}{nom}{-e}{}{Rapper/in}

\entry{resayene}{}{verb}{}{}{\textsuperscript{1}erreichen, hingelangen, angelangen, ankommen \textsuperscript{2}\textit{(Pflanzen, Früchte u. dgl.)} heranwachsen, erblühen}

\entry{resim}{}{nom}{n}{}{Bild}

\entry{rew}{}{adv}{}{}{früh}

\entry{rew ra}{}{adv}{}{}{\textsuperscript{1}seit langem, von früher \textsuperscript{2}frühzeitig, in aller Frühe}

\entry{rey}{}{nom}{m}{}{Mal \textbf{$\rightarrow$ fay} \par\noindent\textit{Reyê} - Einmal \par\noindent\textit{Di rey} - Zwei mal}

\entry{reykî}{}{nom}{n}{}{Multiplikation, Mal}

\entry{reyna}{}{adv}{}{}{nochmal, erneut, wieder \textbf{$\rightarrow$ fayna}}

\entry{rê}{}{post}{}{}{\textsuperscript{1}hin, zu, an \textsuperscript{2}für \textit{(Benefaktiv)} \par\noindent\textit{Xidir mi rê kitab ano} - Xidir bringt mir ein Buch}

\entry{rêçe}{}{nom}{m}{}{\textsuperscript{1}Spur, Abdruck \textsuperscript{2}Pfad, Fährte}

\entry{rind}{}{adj}{-e}{}{gut}

\entry{Rinde}{}{nom}{m}{}{\textit{weibl. Vorname}}

\entry{rindek}{}{adj}{-e}{}{schön, hübsch}

\entry{rindîye}{rindênî}{nom}{m}{}{Güte, Gunst, Wohltat}

\entry{rî}{}{nom}{n}{}{\textsuperscript{1}Gesicht, Visage, Antlitz \textsuperscript{2}Oberfläche, Fläche}

\entry{ro\textsuperscript{1}}{}{post}{}{}{\textsuperscript{1}hinab, herab \textsuperscript{2}hindurch, durch \par\noindent\textit{O gina bin ro} - Er fiel darunter hindurch}

\entry{ro\textsuperscript{2}}{}{nom}{n}{ruh}{Geist, Seele}

\entry{robot}{}{nom}{n}{}{\textsuperscript{1}Roboter \textsuperscript{2}Küchenmaschine}

\entry{roc}{}{nom}{}{}{\textbf{$\rightarrow$ roz}}

\entry{roce}{}{nom}{}{}{\textbf{$\rightarrow$ roze}}

\entry{rojbîyayîş}{}{}{}{}{\textbf{$\rightarrow$ rozbîyene}}

\entry{rojname}{}{nom}{n}{}{Zeitung}

\entry{ron}{}{nom}{n}{}{\textbf{$\rightarrow$ rûn}}

\entry{ronayene}{}{verb}{}{}{hinlegen, hinstellen, absetzen, auflegen}

\entry{ronîştene}{}{verb}{}{}{\textsuperscript{1}sitzen, sich hinsetzen \textsuperscript{2}wohnen}

\entry{roportaj}{}{nom}{n}{}{Reportage, Interview}

\entry{roştî}{}{nom}{m}{}{\textsuperscript{1}Licht \textsuperscript{2}Helligkeit}

\entry{rotene}{}{verb}{}{}{verkaufen, veräußern}

\entry{roz}{roc}{nom}{n}{roj}{Sonne}

\entry{rozbîyene}{rocbîyene}{nom}{m}{}{Geburtstag \par\noindent\textit{Rozbîyena to bimbarek bo!} - Alles Gute zum Geburtstag!}

\entry{roze\textsuperscript{1}}{roce, roci}{nom}{n}{roje}{Fasten, Fastenzeit}

\entry{roze\textsuperscript{2}}{roce}{nom}{m}{roje}{Tag}

\entry{ruj}{}{nom}{n}{}{Lippenstift}

\entry{rut}{}{adj}{-e}{}{nackt}

\entry{rûn}{ron}{nom}{n}{}{\textsuperscript{1}Butter \textsuperscript{2}Öl}

\entry{Rûsya}{}{nom}{m}{}{Russland}

\end{multicols}

%----------------------------------------------------------------------------------------
%	SECTION S
%----------------------------------------------------------------------------------------

\section*{S}

\begin{multicols}{2}

\entry{sa}{}{adj}{-ye}{şa}{fröhlich, froh, glücklich, heiter}

\entry{sabun}{}{nom}{n}{}{Seife}

\entry{sac}{soz}{nom}{n}{}{Fladenbäcker \textit{(Kochplatte)}}

\entry{saete}{saate}{nom}{m}{}{\textsuperscript{1}Uhr \textsuperscript{2}Uhrzeit \textsuperscript{3}Stunde \par\noindent\textit{Na saete (de)} - Zu dieser Zeit, gerade}

\entry{saîr}{sayîr}{nom}{-e}{şaîr}{\textsuperscript{1}Dichter/in, Poet/in  \textsuperscript{2}Barde $\rightarrow$ \textbf{ozan}}

\entry{salon}{}{nom}{n}{}{Salon, Saal}

\entry{samî}{}{nom}{m}{şamî}{Abendessen}

\entry{san}{son, sond}{nom}{n}{şan}{Abend \par\noindent\textit{San de} - Am Abend}

\entry{sandane}{sondane}{adv}{}{şandane}{abends}

\entry{sandalîya}{}{nom}{m}{sandalî}{Stuhl}

\entry{sandiqe}{}{nom}{m}{sindoqe}{Truhe, Kiste}

\entry{sane}{}{nom}{n}{şane}{Kamm}

\entry{sane kerdene}{}{verb}{}{şane kerdene}{kämmen}

\entry{sanike}{}{nom}{m}{}{Märchen, Mär}

\entry{sanîye}{}{nom}{m}{}{Sekunde}

\entry{santîm}{}{nom}{n}{}{$\rightarrow$ \textbf{santîmetre}}

\entry{santîmetre}{}{nom}{n}{}{Zentimeter}

\entry{sar}{}{nom}{n}{şar}{\textsuperscript{1}Volk \textsuperscript{2}Leute}

\entry{sare}{}{nom}{n}{}{$\rightarrow$ \textbf{sere}}

\entry{saye}{soye}{nom}{m}{}{Apfel}

\entry{se\textsuperscript{1}}{sa}{frg}{}{}{was \textit{(mit den Verben bîyene\textsuperscript{2}, kerdene, vatene)} $\rightarrow$ \textbf{çi} \par\noindent\textit{Se kena?} - Was machst du? \par\noindent\textit{Se vana?} - Was sagst du? \par\noindent\textit{Se bî?} - Was ist passiert?}

\entry{se\textsuperscript{2}}{}{zahl}{}{}{hundert}

\entry{sebeb}{}{nom}{n}{}{Grund, Ursache, Anlass}

\entry{sebebo ke}{sebeto ke, seweta ke}{konj}{}{}{\textsuperscript{1}da, weil \textsuperscript{2}aufgrund (dessen), dass $\rightarrow$ \textbf{serba ke}}

\entry{sebir}{sebr}{nom}{n}{}{Geduld, Geduldsfaden, Langmut}

\entry{seke}{}{adv}{}{}{als ob, als wenn, so wie}

\entry{selam}{}{nom}{n}{}{Gruß, Begrüßung \par\noindent\textit{Her kesî rê selamê mi est o} - Grüße an alle (von mir)}

\entry{seme}{}{nom}{n}{şeme}{Samstag}

\entry{sen}{}{frg}{}{}{$\rightarrow$ \textbf{senîn}}

\entry{senetkar}{}{nom}{-e}{}{Künstler/in $\rightarrow$ \textbf{hunermend}}

\entry{senî}{senê}{frg}{}{}{wie; inwiefern $\rightarrow$ \textbf{çiturî, çitan} \par\noindent\textit{No senî beno?} - Wie geht das? \par\noindent\textit{No senî kar o?} - Was ist das für eine Arbeit?}

\entry{senîn}{senên, sen}{frg}{}{}{$\rightarrow$ \textbf{çiturî, çitan}}

\entry{ser}{}{nom}{n}{}{Oben, Ober-, das Obere}

\entry{ser de}{}{adv, post}{}{}{auf, oben}

\entry{ser o}{}{adv}{}{}{auf, drauf, über, drüber; oben}

\entry{ser o kî}{}{adv}{}{}{zudem, zusätzlich, außerdem}

\entry{ser şîyene}{}{verb}{}{}{begreifen, auffassen, kapieren, verstehen}

\entry{seranser}{}{adv}{}{}{\textsuperscript{1}durchgehend, durchweg, gänzlich \textsuperscript{2}über und über, von Anfang bis Ende}

\entry{serba}{}{präp}{}{}{\textsuperscript{1}für \textit{(Benefaktiv)} \textsuperscript{2}zwecks, wegen, aufgrund}

\entry{serba ke}{}{konj}{}{}{\textsuperscript{1}da, weil \textsuperscript{2}aufgrund (dessen), dass}

\entry{serd}{}{adj}{-e}{}{kalt \textit{(i.d.R. Wetter, Luft)}}

\entry{serdin}{}{adj}{-e}{}{kalt \textit{(i.d.R. Oberflächen, Gegenstände etc.)}}

\entry{sere}{sare}{nom}{n}{}{\textsuperscript{1}Kopf \textsuperscript{2}Anfang, Beginn}

\entry{serek}{}{nom}{-e}{}{Anführer/in, Führer/in, (Ober)Haupt, Vorsitzende/r}

\entry{serekkomar}{}{nom}{-e}{}{Präsident, Staatsoberhaupt}

\entry{serekwezîr}{}{nom}{-e}{}{Ministerpräsident, Premierminister}

\entry{serê ... de}{}{zirk}{}{}{auf, oben, über}

\entry{serê sodirî}{}{nom}{n}{}{Morgendämmerung, Tagesanbruch}

\entry{sergarson}{}{nom}{-e}{}{Oberkellner/in, Chefkellner/in}

\entry{sergovend}{}{nom}{-e}{}{Vortänzer/in im $\rightarrow$ \textbf{govende}}

\entry{serkar}{}{nom}{-e}{}{\textsuperscript{1}Chef/in, Vorgesetzte/r \textsuperscript{2}Vorarbeiter/in $\rightarrow$ \textbf{şef, patron}}

\entry{serkay}{}{nom}{-e}{}{Spielführer/in}

\entry{serm}{}{nom}{n}{şerm}{Scham, Scheu}

\entry{sermayene}{}{verb}{}{şermayene}{sich schämen, sich scheuen}

\entry{sermîyan}{}{nom}{-e}{}{\textsuperscript{1}Leiter/in, Oberhaupt, Vorsteher/in, Manager/in $\rightarrow$ \textbf{îdareker} \textsuperscript{2}Administrator/in, Direktor/in}

\entry{sermîyanîye}{}{nom}{m}{}{\textsuperscript{1}Führung, Leitung \textsuperscript{2}Amt, Verwaltung, Regierung \textsuperscript{3}Direktion, Geschäftsleitung $\rightarrow$ \textbf{îdare}}

\entry{serre}{}{nom}{m}{}{\textsuperscript{1}Jahr \textsuperscript{2}Alter \par\noindent\textit{Ti çend serrî ya?} - Wie alt bist du?}

\entry{serrna}{}{adv}{}{}{nächstes Jahr, das kommende Jahr, das Folgejahr}

\entry{ses}{}{zahl}{}{şeş}{sechs}

\entry{seştî}{}{zahl}{}{şeştî}{sechzig}

\entry{seterna}{}{adv}{}{}{übernächstes Jahr, in zwei Jahren}

\entry{sewe}{}{nom}{m}{şewe}{Nacht}

\entry{sewe de}{}{adv}{}{şewe de}{in der Nacht}

\entry{sewenême}{}{nom}{n}{şewenême}{Mitternacht $\rightarrow$ \textbf{sewlete}}

\entry{seweta}{}{präp}{}{}{$\rightarrow$ \textbf{serba}}

\entry{seweta ke}{}{konj}{}{}{$\rightarrow$ \textbf{serba ke}}

\entry{sewle}{}{nom}{m}{şewle}{Licht, Schein}

\entry{sewlete}{}{nom}{n}{şewlete}{Mitternacht $\rightarrow$ \textbf{sewenême}}

\entry{sê kerdene}{şêr kerdene}{verb}{}{seyr kerdene}{betrachten, ansehen, anschauen}

\entry{sêm}{}{nom}{m}{}{$\rightarrow$ \textbf{şêm}}

\entry{sêseme}{şêseme}{nom}{n}{sêşeme}{Dienstag}

\entry{sibate}{}{nom}{m}{şibate}{Februar $\rightarrow$ \textbf{gucige}}

\entry{sil}{}{nom}{n}{}{\textsuperscript{1}Dung, Dünger \textsuperscript{2}Kuhfladen}

\entry{silayîye}{}{nom}{m}{}{Einladung}

\entry{silayî kerdene}{}{verb}{}{}{einladen}

\entry{silîye}{}{nom}{m}{}{$\rightarrow$ \textbf{şîlîye}}

\entry{sifir}{}{zahl}{}{}{null}

\entry{siknayene}{}{verb}{}{şiknayene}{brechen, zerbrechen, kaputt machen}

\entry{Silîvan}{Farqîn}{nom}{n}{}{\textit{Stadt in Dîyarbekir (türk. Silvan)}}

\entry{Sine}{}{nom}{m}{}{\textit{Stadt in Kurdistan (pers. Sanandaj)}}

\entry{sinife}{}{nom}{m}{}{\textsuperscript{1}Klasse, Gattung \textsuperscript{2}Klassenzimmer, Lehrzimmer}

\entry{sima\textsuperscript{1}}{}{pron}{}{şima}{ihr (Plural)}

\entry{sima\textsuperscript{2}}{}{pron}{}{şima}{\textit{2. Fall von} $\rightarrow$ \textbf{sima\textsuperscript{1}}}

\entry{simer}{}{nom}{n}{}{Heu, Stroh}

\entry{simitene}{}{verb}{}{}{trinken}

\entry{sipas}{}{ptkl}{}{}{Danke $\rightarrow$ \textbf{Wes be}}

\entry{sipasdar}{}{adj}{-e}{}{dankbar, verbunden $\rightarrow$ \textbf{minetkar}}

\entry{sipê}{sipî}{adj}{-îye}{}{weiß}

\entry{sipî}{sipê}{adj}{-ye}{}{weiß}

\entry{sira}{}{nom}{m}{}{\textsuperscript{1}Reihe \textsuperscript{2}Warteschlange}

\entry{sist}{}{adj}{-e}{}{locker, lasch, lose}

\entry{sit}{}{nom}{n}{şit}{Milch}

\entry{sivik}{}{adj}{-e}{}{\textsuperscript{1}leicht \textit{(Gewicht)} \textsuperscript{2}schwach, sanft, mild}

\entry{sîyaset}{}{nom}{n}{}{Politik}

\entry{sîyasetmedar}{}{nom}{-e}{}{Politiker/in}

\entry{soba}{}{nom}{m}{}{(Heiz)Ofen}

\entry{sodir}{}{nom}{n}{şodir, şewdir}{Morgen, Früh}

\entry{sodir ra}{}{adv}{}{}{am Morgen, morgens, in der Früh}

\entry{sole}{}{nom}{m}{}{Salz}

\entry{solin}{}{adj}{-e}{}{\textsuperscript{1}salzig \textsuperscript{2}versalzen}

\entry{son}{}{nom}{n}{}{$\rightarrow$ \textbf{san}}

\entry{sondane}{}{adv}{}{}{$\rightarrow$ \textbf{sandane}}

\entry{sora}{}{adv}{}{}{morgen $\rightarrow$ \textbf{meşte}}

\entry{sorbike}{sorvike}{nom}{m}{şorbike}{Suppe}

\entry{soye}{}{nom}{m}{}{$\rightarrow$ \textbf{saye}}

\entry{soz}{}{nom}{n}{}{$\rightarrow$ \textbf{sac}}

\entry{sunî}{}{nom}{-ye}{}{Sunnite/in}

\entry{surprîz}{}{nom}{n}{}{Überraschung}

\entry{suse}{}{nom}{n}{şuşe}{Flasche}

\entry{suzike}{sujîye}{nom}{m}{}{Brise}

\entry{sûke}{}{nom}{m}{}{\textsuperscript{1}Innenstadt, Bazar \textsuperscript{2}Stadt}

\entry{Sûrîye}{}{nom}{m}{}{Syrien}

\entry{sür}{sur}{adj}{-e}{sûr}{\textsuperscript{1}rot \textsuperscript{2}heiß}

\entry{sür kerdene}{sur kerdene}{verb}{}{sûr kerdene}{anbraten, braten, frittieren}

\entry{Swêd}{}{nom}{n}{}{Schweden}

\entry{Swîs}{}{nom}{n}{}{Schweiz}

\end{multicols}

%----------------------------------------------------------------------------------------
%	SECTION Ş
%----------------------------------------------------------------------------------------

\section*{Ş}

\begin{multicols}{2}

\entry{şa}{}{adj}{}{}{$\rightarrow$ \textbf{şîya}}

\entry{şans}{}{nom}{n}{}{\textsuperscript{1}Chance, Gelegenheit \textsuperscript{2}Glück, Glücksfall}

\entry{şar}{}{nom}{m}{}{$\rightarrow$ \textbf{sar}}

\entry{şaristan}{}{nom}{n}{}{Stadt}

\entry{şef}{}{nom}{-e}{}{Chef/in, Vorgesetzte/r, Boss $\rightarrow$ \textbf{serkar}}

\entry{şêm}{}{nom}{n}{sêm}{Silber}

\entry{şên}{}{adj}{-e}{}{freudig, heiter, fröhlich, gehoben}

\entry{şêne}{}{nom}{n}{sêne}{Brust}

\entry{şênîye}{}{nom}{m}{}{\textsuperscript{1}Freude, Fröhlichkeit, Frohsinn, Heiterkeit \textsuperscript{2}Fest, Festivität, Festlichkeit}

\entry{şêr}{}{nom}{-e}{}{Löwe/Löwin}

\entry{şêr kerdene}{sê kerdene}{verb}{}{seyr kerdene}{betrachten, ansehen, anschauen}

\entry{şêseme}{}{nom}{}{}{$\rightarrow$ \textbf{sêseme}}

\entry{şima}{}{pron}{}{}{$\rightarrow$ \textbf{sima}}

\entry{şîfre}{}{nom}{n}{}{}

\entry{şîî}{}{nom}{-ye}{}{Schiite/in}

\entry{şîkîyayene\textsuperscript{1}}{}{verb}{}{eşkayene}{können, fähig sein, zu tun vermögen $\rightarrow$ \textbf{besekerdene}}

\entry{şîkîyayene\textsuperscript{2}}{}{verb}{}{şikîyayene}{brechen, zerbrechen, zu Bruch gehen, kaputtgehen}

\entry{şîlane}{}{nom}{m}{}{\textsuperscript{1}Hagebutte \textsuperscript{2}\textit{weibl. Vorname}}

\entry{şîlîye}{silîye}{nom}{m}{şilîye}{Regen}

\entry{şîlîyin}{silîyin}{adj}{-e}{}{regnerisch, verregnet}

\entry{şîlpa\"{x}e}{şîlpate}{nom}{m}{}{Ohrfeige, Backpfeife, Schelle $\rightarrow$ \textbf{çapale}}

\entry{şîr\textsuperscript{1}}{}{nom}{n}{sîr}{Knoblauch}

\entry{şîr\textsuperscript{2}}{}{nom}{n}{}{\textit{traditionelles Gericht aus Dêrsim und Umgebung}}

\entry{şîranî}{}{nom}{m}{}{Süßspeise, Nachtisch, Dessert}

\entry{şîrên}{}{adj}{-e}{şîrin}{\textsuperscript{1}süß \textsuperscript{2}niedlich}

\entry{şîrkete}{}{nom}{m}{}{Unternehmen, Gesellschaft, Firma}

\entry{şîse}{}{nom}{m}{}{$\rightarrow$ \textbf{suse}}

\entry{şîving}{}{nom}{n}{}{Vordach, Sims}

\entry{şîya}{şa}{adj}{-e}{sîya}{schwarz}

\entry{şîyane}{}{nom}{-îye}{}{$\rightarrow$ \textbf{şüane}}

\entry{şîye}{}{nom}{m}{sîye}{Schatten}

\entry{şîyene}{şîyayene}{verb}{}{}{gehen, begeben, hingehen \par\noindent\textit{Şîme!} - Gehen wir!}

\entry{şüane}{şîyane, şane}{nom}{-îye}{şiwane}{Hirte/Hirtin}

\end{multicols}

%----------------------------------------------------------------------------------------
%	SECTION T
%----------------------------------------------------------------------------------------

\section*{T}

\begin{multicols}{2}

\entry{taba}{\.{t}owa}{nom}{n}{teba}{etwas \par\noindent\textit{Taba nêbeno} - Macht nichts}

\entry{tabî}{}{adv}{}{}{natürlich, selbstverständlich, freilich}

\entry{tajî}{}{nom}{-ye}{}{$\rightarrow$ \textbf{tazî}}

\entry{\.{t}al}{\.{t}ol}{adj}{-e}{}{\textsuperscript{1}leer \textsuperscript{2}bitter}

\entry{talebe}{}{nom}{-îye}{telebe}{Schüler/in; Student/in \textbf{$\rightarrow$ wendekar}}

\entry{talib}{}{nom}{-e}{}{Jünger, religiöses Gefolge}

\entry{\.{t}am}{\.{t}om}{nom}{n}{tehm}{Geschmack, Aroma}

\entry{tanî}{}{nom}{m}{}{(Ausstrahlungs)Wärme}

\entry{taqîb kerdene}{}{verb}{}{}{folgen, verfolgen}

\entry{tarî}{}{adj}{-ye}{}{dunkel, düster}

\entry{tarîx}{}{nom}{n}{}{Geschichte}

\entry{tarrûtur}{}{nom}{n}{}{\textsuperscript{1}Gemüse \textsuperscript{2}Kräuter}

\entry{tatîl}{}{nom}{n}{}{Urlaub, Ferien}

\entry{tavare}{}{}{}{}{$\rightarrow$ \textbf{ga\"{x}an\textsuperscript{2}}}

\entry{taybetî}{}{adj}{-ye}{}{speziell, besonders, eigen}

\entry{tayê}{}{pron}{}{}{einige, ein paar}

\entry{tayêna}{}{adv}{}{}{etwas mehr, noch ein paar, noch ein wenig}

\entry{tayîne}{}{pron}{}{}{\textit{2. Fall von} $\rightarrow$ \textbf{tayê}}

\entry{tazî}{tajî}{nom}{-ye}{}{Windhund, Jagdhund}

\entry{tebaxe}{}{}{}{}{$\rightarrow$ \textbf{a\"{x}ustose}}

\entry{teber}{}{nom}{n}{}{das Freie, das Äußere, Außen(bereich)}

\entry{teber ra}{}{adv}{}{}{außen, draußen}

\entry{tebîat}{tabîat}{nom}{n}{tebîet}{Natur $\rightarrow$ \textbf{xoza}}

\entry{tede}{}{adv}{}{}{darin}

\entry{telefon}{}{nom}{n}{}{Telefon}

\entry{telefon kerdene}{}{verb}{}{}{telefonieren, anrufen}

\entry{televîzyon}{}{nom}{n}{}{Fernseher}

\entry{temam\textsuperscript{1}}{tamam}{ptkl}{}{}{\textsuperscript{1}in Ordnung, okay \textsuperscript{2}einverstanden, abgemacht}

\entry{temam\textsuperscript{2}}{tamam}{adj}{-e}{}{ganz, vollständig, komplett}

\entry{temam kerdene}{}{verb}{}{}{\textsuperscript{1}vervollständigen, vollenden, fertig stellen \textsuperscript{1}abschließen, beenden}

\entry{tembûr}{\.{t}omir}{nom}{n}{}{Tenbur}

\entry{temmuze}{}{nom}{m}{}{Juli}

\entry{tene}{}{nom}{m}{}{Stück \textit{(belebt)}, Person}

\entry{tenê}{}{adv}{}{}{ein wenig, ein bisschen etwas}

\entry{tenêna}{}{adv}{}{}{noch ein wenig, noch ein bisschen, etwas mehr}

\entry{tenik}{}{adj}{-e}{}{dünn, fein}

\entry{teng}{}{adj}{-e}{}{eng}

\entry{tenîye}{}{nom}{m}{}{Ruß}

\entry{tenya}{}{adv}{}{}{$\rightarrow$ \textbf{teyna}}

\entry{tepîya}{}{adv}{}{}{\textsuperscript{1}hinter, zurück \textsuperscript{2}danach, hinterher, nachher}

\entry{tern}{}{adj}{-, e}{}{feucht, nass \textit{(Holz u.ä.)}  \par\noindent\textit{Destê to tern bê!} - Danke! \textit{(für händische Arbeiten)}}

\entry{ters}{}{nom}{n}{}{Angst, Furcht}

\entry{tersayene}{}{verb}{}{}{Angst haben, (sich) fürchten}

\entry{terzî}{}{nom}{-îye}{}{Schneider/in}

\entry{tesaduf}{}{nom}{n}{}{Zufall, Gelegenheit}

\entry{tesdîq}{tasdîq}{nom}{n}{}{\textsuperscript{1}Bestätigung, Anerkennung, Annahme \textsuperscript{2}Beglaubigung, Beurkundung}

\entry{tesdîq kedene}{tasdîq kerdene}{verb}{}{}{\textsuperscript{1}bestätigen, anerkenenn, annehmen \textsuperscript{2}beglaubigen, attestieren, bescheinigen, beurkunden}

\entry{teselîye}{}{nom}{m}{}{\textsuperscript{1}Trost, Zuspruch \textsuperscript{2}Lichtblick, Hoffnung}

\entry{tewewur}{tasawur}{nom}{n}{}{\textsuperscript{1}Vorstellung, Vorhaben \textsuperscript{2}Entwurf, Design, Konzept}

\entry{tew}{}{intrj}{}{}{\textit{Ausruf für Verwunderung, Spott, Staunen}}

\entry{tewir}{}{nom}{n}{}{Art, Typ, Sorte, Gattung}

\entry{tewir be tewir}{}{adv}{}{}{vielfältig, verschiedenartig, vielseitig}

\entry{tey}{}{adv, pron}{}{}{\textsuperscript{1}darin \textsuperscript{2}dabei, mit}

\entry{teyara}{}{nom}{m}{}{Flugzeug}

\entry{teyna}{tenya}{adv}{}{tena}{\textsuperscript{1}allein, alleine, einzig \textsuperscript{2}einsam, isoliert \textsuperscript{3}nur, lediglich}

\entry{\.{t}eyr}{}{nom}{-e}{}{Vogel \par\noindent\textit{\.{T}eyr û tur} - Vögel \textit{(Kollektivum)}}

\entry{têde}{}{pron}{}{}{alle, alles $\rightarrow$ \textbf{pêro}}

\entry{têduşt}{}{adj}{-e}{}{\textsuperscript{1}gleich, gleichgestellt, gleichmäßig \textsuperscript{2}äquivalent, gleichwertig}

\entry{têpey}{}{adv}{}{}{hintereinander}

\entry{têsan}{}{adj}{-e}{têşan}{durstig}

\entry{ti}{tu, to}{pron}{}{}{du}

\entry{tim}{}{adv}{}{}{immer, stets}

\entry{timûtim}{}{adv}{}{}{immer wieder, stets, durchgehend, andauernd}

\entry{tiramî}{}{nom}{m}{}{Glut}

\entry{tirêm}{}{ptkl}{}{}{wohl, etwa \textit{(Fragepartikel)} $\rightarrow$ \textbf{derê}}

\entry{tirk}{}{nom}{-e}{}{Türke/Türkin}

\entry{tirkî}{}{nom}{m}{}{Türkisch}

\entry{Tirkîya}{}{nom}{m}{}{Türkei}

\entry{tirs}{}{adj}{-e}{tirş}{sauer \textit{(Geschmack)}}

\entry{tî}{}{pron}{}{}{$\rightarrow$ \textbf{to}}

\entry{tîjî}{tîcî}{nom}{m}{tîje}{\textsuperscript{1}Sonne \textsuperscript{2}Sonnenlicht, Sonnenstrahlen \textsuperscript{3}Sonnenwärme}

\entry{tîjin}{tîcin}{adj}{-e}{}{sonnig}

\entry{tîşort}{}{nom}{n}{}{T-Shirt}

\entry{\.{t}îtik}{}{nom}{n}{}{Spielzeug}

\entry{tîyatro}{}{nom}{n}{}{Theater}

\entry{to}{tu, tü, tî}{pron}{}{}{\textit{2. Fall von} $\rightarrow$ \textbf{ti}}

\entry{\.{t}ol}{}{}{}{}{$\rightarrow$ \textbf{\.{t}al}}

\entry{tomatêse}{tamatêse}{nom}{m}{}{Tomate $\rightarrow$ \textbf{lolike}}

\entry{\.{t}omir}{}{nom}{n}{}{$\rightarrow$ \textbf{tembûr}}

\entry{\.{t}oraq}{toraq}{nom}{n}{}{\textit{eine Art Magermilchkäse}}

\entry{torge}{}{nom}{m}{}{Hagel}

\entry{torn}{}{nom}{-e}{}{Enkel/in, Enkelkind}

\entry{\.{t}owa}{taba}{nom}{n}{teba}{etwas \par\noindent\textit{\.{T}owa nêbeno} - Macht nichts}

\entry{traktor}{}{nom}{n}{}{Traktor}

\entry{tu}{}{pron}{}{}{$\rightarrow$ \textbf{ti} $\rightarrow$ \textbf{to}}

\entry{tum}{}{nom}{n}{}{Hügel}

\entry{turik}{}{nom}{n}{}{Tüte, Sack}

\entry{turîst}{}{nom}{-e}{}{Tourist/in}

\entry{tusk}{}{nom}{-e}{}{Geißlein, Zicklein}

\entry{tü\textsuperscript{1}}{}{nom}{n}{}{Spucke, Speichel}

\entry{tü\textsuperscript{2}}{}{intrj}{}{}{\textit{Ausruf für Spott, Ekel, Entrüstung oder Bedauern}}

\entry{tü\textsuperscript{3}}{}{pron}{}{}{$\rightarrow$ \textbf{to}}

\entry{tüye}{}{nom}{m}{}{Maulbeere}

\end{multicols}

%----------------------------------------------------------------------------------------
%	SECTION U
%----------------------------------------------------------------------------------------

\section*{U}

\begin{multicols}{2}

\entry{unîversîte}{}{nom}{m}{}{Universität}

\entry{uris}{oris}{nom}{-e}{ûris}{Russe/Russin}

\entry{usar}{}{nom}{}{}{$\rightarrow$ \textbf{wusar}}

\entry{usta}{}{}{}{}{$\rightarrow$ \textbf{osta} $\rightarrow$ \textbf{westa}}

\entry{uska}{}{adv}{}{}{$\rightarrow$ \textbf{uza}}

\entry{ustine}{}{nom}{m}{}{Säule, Stütze, Stützpfeiler}

\entry{uşîre}{}{nom}{m}{}{Stab, Stock, Rute}

\entry{uwe}{}{nom}{m}{}{$\rightarrow$ \textbf{awe}}

\entry{uza}{uca, uja}{nom}{n}{uca}{dort}

\end{multicols}

%----------------------------------------------------------------------------------------
%	SECTION Û
%----------------------------------------------------------------------------------------

\section*{Û}

\begin{multicols}{2}

\entry{û}{u, o}{konj}{}{}{und}

\entry{Ûganda}{}{nom}{m}{}{Uganda}

\entry{ûêb. (û ê bînî)}{}{}{}{}{usw. (und so weiter)}

\entry{ûsn. (û sey nînan)}{}{}{}{}{u.ä. (und ähnliche), u. dgl. (und dergleichen)}

\entry{ûsul}{}{nom}{n}{}{Art, Art und Weise, Manier, Stil}

\end{multicols}

%----------------------------------------------------------------------------------------
%	SECTION V
%----------------------------------------------------------------------------------------

\section*{V}

\begin{multicols}{2}

\entry{va}{}{nom}{n}{}{Wind}

\entry{Vacu\"{xe}}{}{nom}{m}{}{\textbf{$\rightarrow$ Pulur}}

\entry{varan}{}{nom}{n}{}{\textbf{$\rightarrow$ şîlîye}}

\entry{varason}{}{nom}{m}{}{\textbf{$\rightarrow$ verason}}

\entry{vare}{}{nom}{m}{}{\textbf{$\rightarrow$ vore}}

\entry{vartivare}{}{nom}{m}{}{\textbf{$\rightarrow$ hazîrane}}

\entry{vas}{}{nom}{n}{vaş}{Gras, Kraut}

\entry{vatene}{}{verb}{}{}{sagen}

\entry{vaûvarike}{}{nom}{m}{}{Böe, Getöse, mittelstarker Wind}

\entry{vayin}{}{adj}{-e}{}{windig}

\entry{vejîyayene}{vecîyayene}{verb}{}{}{rauskommen, herauskommen, heraustreten, austreten, entspringen, herrühren}

\entry{velg}{}{nom}{n}{}{\textsuperscript{1}Baumblätter \textsuperscript{2}getrocknete Blätter \textit{(z.B. Tierfutter)}}

\entry{veng}{}{nom}{n}{}{\textsuperscript{1}Stimme \textsuperscript{2}Geräusch, Laut, Klang, Ton}

\entry{ver}{}{nom}{n}{}{Vorne, vor, Vor-}

\entry{ver de}{}{adv, post}{}{}{vor \textit{(räumlich)}, davor}

\entry{vera}{}{präp}{}{}{\textsuperscript{1}vor \textsuperscript{2}gegenüber, gegen}

\entry{verasan}{varason}{nom}{n}{veraşan}{später Nachmittag, Abenddämmerung, gegen Abend}

\entry{verba}{}{}{}{ver bi}{\textsuperscript{1}gegen, entgegen, gen \textsuperscript{2}gegenüber, vor}

\entry{verdayene}{}{verb}{}{}{lassen, loslassen, belassen, stehen lassen, hinterlassen}

\entry{verê}{}{adv}{}{}{\textsuperscript{1}zuerst, zuvor \textsuperscript{2}davor, vorher}

\entry{verê cû}{verê coy}{adv}{}{}{früher, damals}

\entry{verê ...de}{}{zirk}{}{}{vor, davor \textit{(räumlich)}}

\entry{verên}{}{adj}{-e}{}{$\rightarrow$ \textbf{virên}}

\entry{verênîye}{}{nom}{m}{}{$\rightarrow$ \textbf{virênîye}}

\entry{verg}{}{nom}{n}{}{Wolf}

\entry{verroz}{verroc}{nom}{n}{verroj}{\textsuperscript{1}Sonnenplatz, Sonnenstelle \textsuperscript{2}Sonnenblume}

\entry{vetene}{}{verb}{}{}{\textsuperscript{1}herausnehmen, herausziehen, entfernen \textsuperscript{2}herausbringen, veröffentlichen}

\entry{veyve\textsuperscript{1}}{veyvi}{nom}{n}{}{Hochzeit}

\entry{veyve\textsuperscript{2}}{}{nom}{m}{}{\textsuperscript{1}Schwiegertochter \textsuperscript{2}Schwägerin \textsuperscript{3}angeheiratete Verwandte}

\entry{veyvike}{}{nom}{m}{veyveke}{Braut}

\entry{vêsan}{}{adj}{-e}{vêşan}{hungrig}

\entry{vêve}{}{nom}{n}{}{$\rightarrow$ veyve}

\entry{vila kerdene}{}{verb}{}{}{verteilen, (ver)verstreuen, verbreiten}

\entry{vildane}{}{nom}{m}{}{Blumenvase}

\entry{vile}{}{nom}{n}{}{Nacken}

\entry{vindarnayene}{}{verb}{}{}{anhalten, aufhalten, Einhalt gebieten, stilllegen, stoppen}

\entry{vindetene}{vindertene}{verb}{}{}{\textsuperscript{1}stehen, stehen bleiben \textsuperscript{2}einhalten, stillstehen, stoppen \textsuperscript{3}verweilen, bleiben, warten \par\noindent\textit{Îta vinde!} - Warte hier! }

\entry{virare}{}{nom}{m}{}{Schoß}

\entry{viraşte}{}{adj}{-îye}{}{\textsuperscript{1}geschaffen, erschaffen, angefertigt, hergestellt, errichtet, gebaut \textsuperscript{2}künstlich, gemacht}

\entry{viraştene}{}{verb}{}{}{schaffen, herstellen, errichten, anfertigen, bauen}

\entry{viraşto\"{x}}{}{nom}{-e}{}{Erschaffer/in, Hersteller/in, Bauer/in}

\entry{virên}{}{adj}{-e}{verên}{\textsuperscript{1}erste/r/s \textsuperscript{2}vordere/r/s}

\entry{virênîye}{verênîye}{nom}{m}{vernî}{Vorne, Vorder-, Vor-}

\entry{vistewre}{vistore, vistöre}{nom}{n}{}{\textsuperscript{1}Schwiegervater \textsuperscript{2}Schwager \textit{(Bruder des Ehemanns oder der Ehefrau)}}

\entry{vistewrîye}{vistorîye, visturîye, vistörîye}{nom}{m}{}{Schwiegermutter}

\entry{vistore}{nom}{n}{}{}{\textbf{$\rightarrow$ vistewre}}

\entry{vistorîye}{nom}{m}{}{}{\textbf{$\rightarrow$ vistewrîye}}

\entry{vîlike}{}{nom}{m}{vilike}{Blume}

\entry{vîndî kerdene}{}{verb}{}{}{verlieren}

\entry{vîr}{}{nom}{n}{}{Gedächtnis, Erinnerung}

\entry{vîşt}{}{zahl}{}{vîst}{zwanzig}

\entry{vorayene}{}{verb}{}{varayene}{regnen}

\entry{vore}{vare}{nom}{m}{vewre}{Schnee}

\entry{vorek}{}{nom}{-e}{verek}{Lamm, Schäfchen}

\entry{vorin}{varin}{adj}{-e}{vewrin}{verschneit, schneebedeckt}

\entry{vurnayene}{}{verb}{}{}{ändern, abändern, verändern, umändern}

\end{multicols}

%----------------------------------------------------------------------------------------
%	SECTION W
%----------------------------------------------------------------------------------------

\section*{W}

\begin{multicols}{2}

\entry{wa}{}{ptkl}{}{}{\textit{Konjunktivpartikel (Optativ)}}

\entry{wad}{od}{nom}{n}{}{Versprechen, Wort \par\noindent\textit{Wad bo!} - Versprochen!}

\entry{war\textsuperscript{1}}{}{adv}{}{}{hinab, herab, hinunter, herunter, runter}

\entry{war\textsuperscript{2}}{}{nom}{n}{}{Bereich, Gebiet}

\entry{ware}{}{nom}{n}{}{Alm, Alp, Hochebene}

\entry{waştene}{}{verb}{}{}{\textsuperscript{1}wollen, möchten \textsuperscript{2}verlangen, fordern, (er)bitten}

\entry{waştî}{}{nom}{-ye}{}{\textsuperscript{1}Verlobte/r \textsuperscript{2}Geliebte/r}

\entry{wax}{}{intrj}{}{}{\textit{Ausruf für Mitleid oder Bedauern}}

\entry{waxt}{}{nom}{n}{}{\textbf{$\rightarrow$ wext}}

\entry{waxto ke}{}{adv}{}{}{\textbf{$\rightarrow$ wexto ke}}

\entry{way}{}{intrj}{}{}{\textit{Ausruf für Bedauern oder Stolz}}

\entry{waye}{}{nom}{m}{}{Schwester}

\entry{wayîr}{}{nom}{-e}{}{\textsuperscript{1}Besitzer, Eigner, Inhaber \textsuperscript{2}Herr \textit{(relig.)}}

\entry{wendegeh}{}{nom}{n}{}{Schule $\rightarrow$ \textbf{mekteb}}

\entry{wendekar}{}{nom}{-e}{}{Schüler/in; Student/in \textbf{$\rightarrow$ talebe}}

\entry{wendene}{wanitene}{verb}{}{}{\textsuperscript{1}lesen \textsuperscript{2}vorlesen \textsuperscript{3}studieren, lernen}

\entry{welat}{}{nom}{n}{}{\textsuperscript{1}Land \textsuperscript{2}Heimat, Heimatland, Vaterland \textsuperscript{3}\textit{männl. Vorname}}

\entry{welatperwer}{}{nom}{-e}{}{Patriot, Vaterlandsliebende/r}

\entry{werdene}{}{verb}{}{}{\textsuperscript{1}essen \textsuperscript{2}fressen \textsuperscript{3}\textit{(mit manchen Flüssigkeiten)} trinken}

\entry{wereza}{}{nom}{n}{wareza}{Neffe \textit{(Sohn des Bruders)}}

\entry{werezaye}{}{nom}{m}{warezaye}{Nichte \textit{(Tochter der Schwester)}}

\entry{werte}{orte}{nom}{n}{orte}{Mitte, Mittelpunkt $\rightarrow$ \textbf{mîyan}}

\entry{werte de}{orte de}{adv}{}{orte de}{\textsuperscript{1}(in)mitten, mittendrin, dazwischen, zwischendrin \textsuperscript{2}gelegentlich $\rightarrow$ \textbf{mîyan de}}

\entry{wes}{}{adj}{-e}{weş}{\textsuperscript{1}gut, wohl \textsuperscript{2}gesund \textsuperscript{3}lecker  
\par\noindent\textit{Wes be!} - Danke (dir)! 
\par\noindent\textit{Wes o} - Das schmeckt gut/lecker}

\entry{westa}{osta}{nom}{-e}{}{Meister/in, Lehrmeister/in}

\entry{weşîye}{}{nom}{m}{}{\textsuperscript{1}Güte, Wohl, Heil \textsuperscript{2}Gesundheit, Wohlergehen 
\par\noindent\textit{Weşîya to} - Danke der Nachfrage}

\entry{wext}{waxt}{nom}{n}{}{Zeit, Zeitraum \par\noindent\textit{Wextê xo de} - Zu seiner Zeit \par\noindent\textit{O wext} - Dann}

\entry{wexto ke}{}{adv}{}{}{\textsuperscript{1}wenn \textsuperscript{2}sobald, als}

\entry{wey}{}{intrj}{}{} {\textit{Ausruf der Verwunderung}}

\entry{wezîfe}{wazîfe}{nom}{n}{}{\textsuperscript{1}Aufgabe, Dienst \textsuperscript{2}Pflicht}

\entry{wezîr}{}{nom}{-e}{}{Minister/in}

\entry{wîy}{}{intrj}{}{} {\textit{Ausruf des Schmerzes oder der Angst}}

\entry{wusar}{usar}{nom}{n}{wisar}{Frühling, Frühjahr}

\entry{wusarî}{usarî}{adv}{}{wisarî}{im Frühling, in der Frühlingszeit}

\entry{wusaro peyên}{}{nom}{n}{}{Mai $\rightarrow$ \textbf{gulane}}

\entry{wusaro virên}{}{nom}{n}{}{März $\rightarrow$ \textbf{marte}}

\entry{wusaro wertên}{}{nom}{n}{}{April $\rightarrow$ \textbf{nîsane}}

\end{multicols}

%----------------------------------------------------------------------------------------
%	SECTION X
%----------------------------------------------------------------------------------------

\section*{X}

\begin{multicols}{2}

\entry{xaç}{}{nom}{n}{}{Kreuz}

\entry{xaçepers}{}{nom}{n}{}{Kreuzworträtsel}

\entry{xafil de}{}{adv}{}{}{plötzlich, auf einmal, schlagartig}

\entry{xafila}{}{adv}{}{}{\textbf{$\rightarrow$ xafil de}}

\entry{xal}{}{nom}{n}{}{Onkel mütterlicherseits}

\entry{xalcênîye}{xancilîye}{nom}{m}{}{Frau des Onkels mütterlicherseits}

\entry{xale}{}{nom}{m}{}{Tante mütterlicherseits \textbf{$\rightarrow$ xalike}}

\entry{xalike}{}{nom}{m}{}{Tante mütterlicherseits \textbf{$\rightarrow$ xale}}

\entry{xalî}{}{nom}{m}{}{Teppich}

\entry{xam}{}{adj}{-e}{}{\textsuperscript{1}roh, unreif \textsuperscript{2}unerfahren \textsuperscript{3}befremdlich, fremd, merkwürdig}

\entry{xancilîye}{}{nom}{m}{}{\textbf{$\rightarrow$ xalcênîye}}

\entry{xane}{}{nom}{n}{}{Haus, Gebäude}

\entry{xatir}{}{nom}{n}{}{\textsuperscript{1}Abschied \par\noindent\textit{Xatir be to!} - Tschüss! \par\noindent\textsuperscript{2}Andenken, Gefälligkeit}

\entry{xayîn}{}{adj}{-e}{}{hinterlistig, hinterhältig, heimtückisch, verräterisch $\rightarrow$ \textbf{bêbext}}

\entry{xebere}{}{nom}{m}{}{\textsuperscript{1}Nachricht, Neuigkeit \textsuperscript{2}Botschaft, Meldung, Kunde \textsuperscript{3}Ahnung, Wissen \par\noindent\textit{Xebera mi çîn a} - Ich weiß nichts davon}

\entry{xelate}{halete}{nom}{m}{}{\textsuperscript{1}Preis, Auszeichnung, Orden \textsuperscript{2}Geschenk}

\entry{xem}{}{nom}{n}{}{Gram, Kummer, Leid}

\entry{xestexane}{}{nom}{n}{}{\textbf{$\rightarrow$ nêwesxane}}

\entry{xeyal}{}{nom}{n}{}{\textsuperscript{1}Vorstellung, Fantasie \textsuperscript{2}Vision, Idol \textsuperscript{3}Einbildung \textbf{$\rightarrow$ \"{x}eyal}}

\entry{xeylê}{}{adv}{}{}{\textbf{$\rightarrow$ xêlê}}

\entry{xêlê}{xeylê}{adv}{}{xeylê}{viel, reichlich, eine Menge}

\entry{xêr}{xeyr}{nom}{n}{xeyr}{\textsuperscript{1}Wohltat, Guttat \textsuperscript{2}Wonne, Nutzen, Glück, Segen \par\noindent\textit{Xêr o?} - Was ist los? \par\noindent\textit{Xêrê xo} - Bitte}

\entry{xêrlî}{}{adj}{-ye}{}{\textbf{$\rightarrow$ bexêr}}

\entry{xê\"{x}}{}{adj}{-e}{}{verrückt, irre, irrsinnig, bekloppt}

\entry{xêz kerdene}{}{verb}{}{}{\textsuperscript{1}zeichnen \textsuperscript{2}durchstreichen}

\entry{xêze}{}{nom}{m}{}{Linie, Strich}

\entry{xinamî}{}{adj}{-ye}{}{verschwägert}

\entry{Xinûs}{}{nom}{n}{}{\textit{Stadt in Kurdistan (türk. Hınıs)}}

\entry{xirîstîyan}{}{nom}{-e}{}{Christ/in}

\entry{xirt}{}{adj}{-e}{}{mutig, tapfer, wacker, beherzt}

\entry{xizmet}{}{nom}{n}{}{\textsuperscript{1}Dienst \textsuperscript{2}Service, Bedienung}

\entry{xizmet kerdene}{}{verb}{}{}{dienen, bedienen}

\entry{xîyar}{}{nom}{n}{}{Gurke}

\entry{xo}{}{pron}{}{}{\textit{Reflexivpronomen}}

\entry{xo vîr ra kerdene}{}{verb}{}{}{vergessen \par\noindent\textit{Xo vîr ra meke!} - Vergiss (es) nicht!}

\entry{xone}{}{nom}{n}{}{Kater}

\entry{xora}{}{adv}{}{}{\textsuperscript{1}eh, sowieso \textsuperscript{2}bereits, schon}

\entry{xorî}{}{adj}{-e}{}{$\rightarrow$ xorîn}

\entry{xorîn}{xorî}{adj}{-e}{}{tief}

\entry{xoser}{}{adj}{-e}{}{\textsuperscript{1}selbstständig, eigenständig \textsuperscript{2}unabhängig, ungebunden \textsuperscript{3}autonom, autark, souverän}

\entry{xoserîye}{xoserênî}{nom}{m}{}{\textsuperscript{1}Selbstständigkeit, Eigenständigkeit \textsuperscript{2}Unabhängigkeit \textsuperscript{3}Autonomie, Autarkie, Souveränität}

\entry{xort}{}{nom}{n}{}{junger Mann, Knabe, Bursche}

\entry{xoz}{}{nom}{-e}{}{Schwein}

\entry{xoza}{}{nom}{n}{}{Natur $\rightarrow$ \textbf{tebîat}}

\entry{Xozat}{}{nom}{n}{}{\textit{Stadt in Dêrsim (türk. Hozat)}}

\end{multicols}

%----------------------------------------------------------------------------------------
%	SECTION Ẍ
%----------------------------------------------------------------------------------------

\section*{\"{X}}

\begin{multicols}{2}

\entry{\"{x}elet}{}{adj}{-e}{}{falsch, inkorrekt}

\entry{\"{x}eletîye}{}{nom}{m}{}{Fehler, Irrtum}

\entry{\"{x}erîb}{}{adj}{-e}{}{\textsuperscript{1}fremd, fremdartig, andersartig \textsuperscript{2}ausländisch}

\entry{\"{x}erîbîye}{}{nom}{m}{}{\textsuperscript{1}Fremde, Fremdartigkeit \textsuperscript{2}Ausland}

\entry{\"{x}eyal}{xeyal}{nom}{n}{}{\textsuperscript{1}Vorstellung, Fantasie \textsuperscript{2}Vision, Idol \textsuperscript{3}Einbildung \textbf{$\rightarrow$ xeyal}}

\entry{\"{x}ezale}{}{nom}{m}{}{Gazelle}

\entry{\"{x}ezeb}{}{nom}{n}{}{Zorn}

\end{multicols}

%----------------------------------------------------------------------------------------
%	SECTION Y
%----------------------------------------------------------------------------------------

\section*{Y}

\begin{multicols}{2}

\entry{ya\textsuperscript{1}}{}{konj}{}{yan}{oder}

\entry{ya\textsuperscript{2}}{}{post}{}{}{\textbf{$\rightarrow$ a\textsuperscript{3}}}

\entry{ya\textsuperscript{3}}{}{ptkl}{}{}{\textbf{$\rightarrow$ heya}}

\entry{ya kî}{}{konj}{}{yan zî}{oder}

\entry{yadîgar}{}{nom}{n}{}{Andenken, Souvenir}

\entry{yakerdene}{}{verb}{}{}{\textbf{$\rightarrow$ rakerdene}}

\entry{yanî}{}{ptkl}{}{}{\textsuperscript{1}also, das heißt \textsuperscript{2}nämlich}

\entry{yar}{}{nom}{-e}{}{Geliebte/r, Liebhaber/in}

\entry{yaranîye}{}{nom}{m}{yarenîye}{Scherz, Jux, Spaß}

\entry{yaw}{}{intrj}{}{}{\textit{Ausruf der Verwunderung und Frage}}

\entry{yazî}{}{nom}{m}{}{\textbf{$\rightarrow$ nuşte}}

\entry{yazî kerdene}{yazmiş kerdene}{verb}{}{}{\textbf{$\rightarrow$ nuştene}}

\entry{yene}{êne}{nom}{n}{îne}{Freitag}

\entry{yew}{}{zahl}{}{}{$\rightarrow$ \textbf{ju}}

\entry{yewşeme}{}{nom}{n}{}{$\rightarrow$ \textbf{bazar}}

\entry{yê}{}{ptkl}{}{}{\textit{Absolute Îzafe}}

\end{multicols}

%----------------------------------------------------------------------------------------
%	SECTION Z
%----------------------------------------------------------------------------------------

\section*{Z}

\begin{multicols}{2}

\entry{zaf}{zof}{adv}{}{}{viel, sehr}

\entry{zafane}{}{adv}{}{}{\textsuperscript{1}meistens, meist, zumeist \textsuperscript{2}für gewöhnlich, normalerweise}

\entry{zafêrî}{}{adv}{}{}{\textsuperscript{1}mehrheitlich, größtenteils \textsuperscript{2}meistenteils, meistens}

\entry{zahmet}{}{nom}{n}{}{$\rightarrow$ \textbf{zehmet}}

\entry{zalim}{}{nom}{-e}{}{Tyrann/in, Despot/in}

\entry{zama}{}{nom}{n}{}{\textsuperscript{1}Bräutigam \textsuperscript{2}Schwager \textsuperscript{3}Schwiegersohn \textsuperscript{4}angeheirateter Onkel}

\entry{zan}{}{nom}{n}{}{$\rightarrow$ \textbf{jan}}

\entry{zanayene}{zonayene, zanitene}{verb}{}{}{\textsuperscript{1}wissen \textsuperscript{2}können \textsuperscript{3}kennen \par\noindent\textit{Ez nêzaneno.} - Ich weiß es nicht.}

\entry{zanitene}{}{verb}{}{}{$\rightarrow$ \textbf{zanayene}}

\entry{zanî}{zonî}{nom}{n}{}{Knie $\rightarrow$ \textbf{çok}}

\entry{Zara}{}{nom}{m}{}{\textsuperscript{1}\textit{Stadt in Kurdistan (türk. Zara)} \textsuperscript{2}\textit{weibl. Vorname}}

\entry{zarance}{}{nom}{m}{zerence}{Rebhuhn}

\entry{zav û zêçî}{zav-zêçî}{nom}{zh}{zav û zêç}{Kinder \textit{(Kollektivum)}, Kind und Kegel}

\entry{zayene}{}{verb}{}{}{kalben, gebären (Tiere)}

\entry{zaza}{}{nom}{-ye}{}{Zaza}

\entry{zazakî}{}{nom}{m}{}{Zazakî $\rightarrow$ \textbf{kirmanckî}}

\entry{zehmet}{zahmet}{nom}{n}{}{\textsuperscript{1}Anstrengung, Mühe, Umstand \textsuperscript{2}Problem, Not \par\noindent\textit{Zehmetê xo ra} - Wenn es keine Umstände macht}

\entry{zelal}{}{adj}{-e}{}{klar, rein}

\entry{Zelale}{}{nom}{m}{}{\textit{weibl. Vorname}}

\entry{zelalîye}{}{nom}{m}{}{Klarheit, Reinheit}

\entry{zemperîye}{}{nom}{m}{}{$\rightarrow$ \textbf{çele}}

\entry{zengil}{}{nom}{n}{}{Klingel, Glocke}

\entry{zengîn}{}{adj}{-e}{}{reich, wohlhabend, vermögend}

\entry{zerd}{}{adj}{-e}{}{goldgelb, eigelb}

\entry{zere}{}{nom}{n}{}{das Innere, Innen(bereich)}

\entry{zere de}{}{adv}{}{}{innen, drinnen}

\entry{zerfet}{}{nom}{n}{}{\textit{traditionelles Gericht aus Dêrsim und Umgebung}}

\entry{zerrî}{}{nom}{m}{}{Herz}

\entry{zerrîserdin}{}{adj}{-e}{}{\textsuperscript{1}kaltherzig, gefühllos, hartherzig  \textsuperscript{2}unbarmherzig, mitleidlos}

\entry{zerrîvêsayîye}{}{adj}{-îye}{zêrrîveşn}{\textsuperscript{1}weichherzig, weitherzig, mild \textsuperscript{2}fürsorglich, barmherzig, mitleidig}

\entry{zerrîya bele}{}{nom}{m}{}{$\rightarrow$ \textbf{zerrî}}

\entry{zerrn}{}{nom}{n}{}{Gold}

\entry{zewbîna}{}{adv}{}{}{$\rightarrow$ \textbf{zobîna}}

\entry{zeyîye}{}{nom}{m}{}{\textsuperscript{1}Schwägerin \textit{(Schwester des Ehemanns)} \textsuperscript{2}\textit{Ausgeheiratete weibliche Mitglieder der Familie}}

\entry{zeytune}{}{nom}{m}{}{Olive}

\entry{zê}{jê}{präp}{}{sey}{\textsuperscript{1}wie, so wie \textsuperscript{2}als \textbf{$\rightarrow$ jê}}

\entry{zêde}{jêde}{adv}{}{zêde, zîyade}{sehr viel, zu viel; überschüssig \textbf{$\rightarrow$ jêde}}

\entry{zêdek}{jêdek}{nom}{n}{}{Addition, Plus}

\entry{zimbêlî}{zimêlî}{nom}{zh}{}{Schnurrbart, Schnauzbart}

\entry{zimistan}{zimiston}{nom}{n}{}{Winter}

\entry{zimistanî}{zimistonî}{adv}{}{}{im Winter, zur Winterzeit}

\entry{zimiston}{}{nom}{n}{}{$\rightarrow$ \textbf{zimistan}}

\entry{ziqnawute}{}{nom}{m}{}{$\rightarrow$ \textbf{çele}}

\entry{zirçayene}{}{verb}{}{}{schreien, aufschreien}

\entry{ziwan}{}{nom}{}{}{\textbf{$\rightarrow$ zon}}

\entry{zî}{}{konj}{}{}{\textbf{$\rightarrow$ kî}}

\entry{zîl}{jîl}{nom}{n}{}{Spross, Trieb}

\entry{zîndan}{}{nom}{n}{}{Kerker, Gefängniszelle}

\entry{zîyare}{}{nom}{m}{}{Ziyaret \textit{(heilige Stätte)} \par\noindent\textit{Zîyar û dîyarî} - Heilige Stätten \textit{(Kollektivum)}}

\entry{zobîna}{zewbîna, zöbîna}{adv}{}{}{(was anderes, ein/e andere/r/s, (sonst) noch, anderweitig}

\entry{zon}{zû, jan, zan}{nom}{n}{ziwan}{\textsuperscript{1}Zunge \textsuperscript{2}Sprache \par\noindent\textit{Zonê ma zanena?} - Sprichst du unsere Sprache?}

\entry{zonayene}{}{verb}{}{}{$\rightarrow$ \textbf{zanayene}}

\entry{zonitene}{}{verb}{}{}{$\rightarrow$ \textbf{zanayene}}

\entry{zonî}{}{nom}{n}{}{$\rightarrow$ \textbf{zanî}}

\entry{zor}{}{adj}{-e}{}{\textsuperscript{1}anstrengend, mühsam, beschwerlich \textsuperscript{2}schwierig, hart}

\entry{zu}{ju, jü}{zahl}{}{yew}{eins \textbf{$\rightarrow$ ju}}

\entry{zu kî}{}{adv}{}{}{außerdem, zudem}

\entry{zulm}{}{nom}{n}{}{\textsuperscript{1}Gräuel, Grausamkeit, Drangsal \textsuperscript{2}Despotie, Unterdrückung}

\entry{zuwa}{jüya}{adj}{-ye}{ziwa}{trocken}

\entry{zuyê}{}{pron}{}{}{eine/r/s, (irgend)jemand \textbf{$\rightarrow$ jüye}}

\entry{zubînî}{jubînî, jumînî}{pron}{}{yewbînî}{einander, gegenseitig \textbf{$\rightarrow$ jubînî}}

\entry{zuna}{zewna, jüna}{adv}{}{yewna}{noch ein/e/er/s, ein weiterer/s, ein andere/s \textbf{$\rightarrow$ juna}}

\end{multicols}

%----------------------------------------------------------------------------------------
%	Deutsch - Kirmanckî
%----------------------------------------------------------------------------------------
\newcounter{mypage}
\setcounter{mypage}{\value{page}} % Save page number

\begin{titlepage}
    \thispagestyle{empty} % Hide number on title page
    \begin{center}
        \vspace*{\stretch{1}}
        {\Huge\textbf{Deutsch - Kirmanckî}} 
        \vspace*{\stretch{1}}
    \end{center}
\end{titlepage}

\setcounter{page}{\numexpr\value{mypage}+2} 

%----------------------------------------------------------------------------------------
%	A
%----------------------------------------------------------------------------------------

\section*{A}

\begin{multicols}{2}

\gentry{Anklage}{nom}{dawa, doze}{m}{}

\gentry{Ausstrahlungswärme}{nom}{tanî}{m}{}

\gentry{besprechen}{verb}{qal kerdene}{}{}

\gentry{ermahnen}{verb}{îqaz kerdene}{}{}

\gentry{Ermahnung}{nom}{îqaz}{n}{}

\gentry{(er)scheinen}{verb}{asayene}{}{}
\gentry{(Gemischtwaren)Händler/in}{nom}{dikandar}{-e}{}
\gentry{(Halb)Schuh}{nom}{kundure}{n}{}
\gentry{(Heiz)Ofen}{nom}{soba}{m}{}
\gentry{(her)bringen}{verb}{ardene}{}{}
\gentry{(her)holen}{verb}{ardene}{}{}
\gentry{(Ober)Haupt}{nom}{serek}{-e}{}
\gentry{(sich) fürchten}{verb}{tersayene}{}{}
\gentry{(sonst) noch}{adv}{zobîna}{}{}
\gentry{(ver)verstreuen}{verb}{vila kerdene}{}{}
\gentry{(ver)warnen}{verb}{îqaz kerdene}{}{}
\gentry{(was anderes}{adv}{zobîna}{}{}
\gentry{(zweckgebundene oder entgeltliche) Arbeit}{nom}{gure}{n}{}

\gentry{ab und an}{adv}{ge-gane}{}{}
\gentry{ab und zu}{adv}{ge-gane}{}{}
\gentry{Abbitte}{nom}{ozr}{n}{}
\gentry{Abenddämmerung}{nom}{verasan}{n}{veraşan}
\gentry{Abendessen}{nom}{samî}{m}{şamî}
\gentry{abends}{adv}{sandane}{}{şandane}
\gentry{aber}{konj}{hama}{}{}
\gentry{aber}{konj}{labelê}{}{}
\gentry{abernten}{verb}{çînitene}{}{}
\gentry{Abfall}{nom}{çop}{n}{}
\gentry{Abfalleimer}{nom}{çopdane}{m}{}
\gentry{abfallen}{verb}{kotene}{}{kewtene}
\gentry{Abrechnung}{nom}{fatura}{m}{}
\gentry{Abrede}{nom}{înkar}{n}{}
\gentry{abreden}{verb}{înkar kerdene}{}{}
\gentry{abschließen}{verb}{kilît kerdene}{}{}
\gentry{absetzen}{verb}{ronayene}{}{}
\gentry{Absicht}{nom}{armanc}{n}{}
\gentry{absperren}{verb}{kilît kerdene}{}{}
\gentry{absprechen}{verb}{înkar kerdene}{}{}
\gentry{abstreiten}{verb}{înkar kerdene}{}{}
\gentry{Abzäuning; Abgrenzung}{nom}{çeper}{n}{}
\gentry{abändern}{verb}{vurnayene}{}{}
\gentry{acht}{zahl}{heşt}{}{}
\gentry{Acht geben}{verb}{diqet kerdene}{}{}
\gentry{achten}{verb}{diqet kerdene}{}{}
\gentry{achtungswert}{adj}{balkêş}{-e}{}
\gentry{achtzig}{zahl}{heştay}{}{}
\gentry{Acker}{nom}{hêga}{n}{}
\gentry{Addition}{nom}{zêdek}{n}{}
\gentry{Afrika}{nom}{Afrîka}{m}{}
\gentry{Afrikaner/in}{nom}{afrîkayiz}{-e}{afrîkayij}
\gentry{Album}{nom}{album}{n}{}
\gentry{Alevite/in}{nom}{elewî}{-ye}{}
\gentry{alle}{nom}{her kes}{n}{}
\gentry{alle}{pron}{pêro}{}{}
\gentry{alle}{pron}{têde}{}{}
\gentry{alle Mann}{nom}{her kes}{n}{}
\gentry{alle(s) zusammen}{adv}{pêro pîya}{}{}
\gentry{alles}{pron}{}{pêro, têde}{}{}
\gentry{alles; jede Sache}{nom}{her çî}{n}{}
\gentry{allgegenwärtig; anwesend und bereitstehend}{adj}{hazir û nazir}{-e}{}
\gentry{allmählich}{adv}{giran-giran}{}{}
\gentry{allzeit bereit}{adj}{hazir û nazir}{-e}{}
\gentry{allzeit; unaufhörlich}{adv}{daîma}{}{}
\gentry{Alm}{nom}{ware}{n}{}
\gentry{Alp}{nom}{ware}{n}{}
\gentry{Alphabet}{nom}{alfabe}{m}{}
\gentry{als ob}{adv}{seke}{}{}
\gentry{als wenn}{adv}{seke}{}{}
\gentry{alsbald}{adv}{desinde}{}{}
\gentry{alt}{adj}{pîl}{-e}{}
\gentry{am Morgen}{adv}{sodir ra}{}{}
\gentry{Amerika}{nom}{Amerîka}{m}{}
\gentry{an der Seite von}{post}{het de}{}{}
\gentry{an der Seite von}{zirk}{hetê ... de}{}{}
\gentry{an diesem Ort}{adv}{naza}{}{noca}
\gentry{an jenem Ort}{}{haza}{adv}{}
\gentry{an Stelle von...}{zirk}{hurendîya ... de}{}{herinda ... de}
\gentry{Anbaufläche}{nom}{hêga}{n}{}
\gentry{anbraten}{verb}{sür kerdene}{}{sûr kerdene}

\gentry{andauernd}{adv}{timûtim}{}{}

\gentry{Andenken}{nom}{yadîgar}{n}{}

\gentry{ander Seite von}{zirk}{lewê ... de}{}{leweyê ... de}

\gentry{andere/r/s; sonstige/r/s}{adj}{bîn}{-e}{}

\gentry{andernfalls}{konj}{nêke}{}{}

\gentry{anderweitig}{adv}{zobîna}{}{}

\gentry{anfertigen}{verb}{viraştene}{}{}

\gentry{Anführer/in}{nom}{serek}{-e}{}

\gentry{Angst}{nom}{ters}{n}{}

\gentry{Angst haben}{verb}{tersayene}{}{}

\gentry{anhalten}{verb}{vindarnayene}{}{}

\gentry{anhand}{präp}{pê}{}{}

\gentry{Anhang}{nom}{îlawe}{n}{}

\gentry{Anhöhe; Hoheit}{nom}{berzîye}{m}{}

\gentry{anhören}{verb}{gosdarî kerdene}{}{goşdarî kerdene}

\gentry{Anlass}{nom}{sebeb}{n}{}

\gentry{Anmerkung}{nom}{not}{n}{}

\gentry{anrufen}{verb}{telefon kerdene}{}{}

\gentry{anschauen}{verb}{sê kerdene \textasciitilde{} şêr kerdene}{}{seyr kerdene}

\gentry{ansehen}{verb}{sê kerdene}{}{seyr kerdene}
\gentry{ansehen}{verb}{şêr kerdene}{}{seyr kerdene}
\gentry{ansonsten}{konj}{nêke}{}{}
\gentry{anstatt...}{zirk}{hurendîya ... de}{}{herinda ... de}
\gentry{ansteckende/virale Krankheit}{nom}{çorr}{n}{}
\gentry{Anteil}{nom}{lete}{n}{}
\gentry{Anwalt/Anwältin}{nom}{avûkat}{-e}{}
\gentry{anziehen \textit{(mit den Füßen)}}{verb}{pay kerdene}{}{}
\gentry{anängen}{verb}{îlawe kerdene}{}{}
\gentry{Apfel}{nom}{saye}{m}{}
\gentry{Appel}{nom}{îqaz}{n}{}
\gentry{Aprikose}{nom}{qeysî}{m}{}
\gentry{April}{nom}{nîsane}{m}{}
\gentry{April}{nîsane}{nom}{wusaro wertên}{n}{}
\gentry{Araber/in}{nom}{arab}{-e}{ereb}
\gentry{Arabisch}{nom}{arabkî}{m}{erebkî}
\gentry{arbeiten}{verb}{gureyayene}{}{}
\gentry{Arbeiter/in}{nom}{karker}{-e}{}
\gentry{arbeitsam}{adj}{gurekar}{-e}{}
\gentry{arbeitslos}{adj}{bêkar}{-e}{}
\gentry{arm}{adj}{feqîr}{-e}{}
\gentry{Arm}{nom}{harme}{n}{}
\gentry{Art}{nom}{tewir}{n}{}
\gentry{Art}{nom}{ûsul}{n}{}
\gentry{Art und Weise}{nom}{ûsul}{n}{}
\gentry{Astronaut/in}{nom}{astronot}{-e}{}
\gentry{auch}{konj}{kî}{}{}
\gentry{Aue}{nom}{merge}{m}{}

\gentry{auf}{adv, post}{ser de}{}{}

\gentry{auf}{adv}{ser o}{}{}

\gentry{auf}{zirk}{serê ... de}{}{}

\gentry{auf der Stelle}{adv}{desinde}{}{}

\gentry{auf die Knie fallen}{verb}{çok ronayene}{}{}

\gentry{auf die Knie gehen}{verb}{çok ronayene}{}{}

\gentry{auf diese Art}{adv}{nîya}{}{}

\gentry{auf einmal}{adv}{xafil de}{}{}

\gentry{auffassen}{verb}{ser şîyene}{}{}

\gentry{aufhalten}{verb}{vindarnayene}{}{}

\gentry{auflegen}{verb}{ronayene}{}{}

\gentry{aufpassen}{verb}{diqet kerdene}{}{}

\gentry{aufschreien}{verb}{zirçayene}{}{}

\gentry{Augenbraue(n)}{nom}{burî}{zh}{birûyî}

\gentry{August}{}{nom}{amnano peyên}{n}{}

\gentry{Ausdruck}{nom}{qal}{n}{qale}

\gentry{ausgehen}{verb}{qedîyayene}{}{}

\gentry{Aushilfe}{nom}{ardim}{n}{}

\gentry{auslaufen}{verb}{qedîyayene}{}{}

\gentry{ausmachen}{verb}{qedênayene}{}{}

\gentry{ausreichend}{adj}{bes}{-e}{}

\gentry{aussehen}{verb}{asayene}{}{}

\gentry{Ausspruch}{nom}{qal}{n}{qale}

\gentry{austreten}{verb}{vejîyayene}{}{}

\gentry{Autobus}{nom}{otobuse}{m}{}

\gentry{außen}{adv}{teber ra}{}{}

\gentry{Außen(bereich)}{nom}{teber}{n}{}

\gentry{außerdem}{adv}{ju kî / zu kî, ser o kî}{}{}

\end{multicols}

\section*{Z}

\begin{multicols}{2}

\gentry{Zahn}{nom}{didan}{n}{}

\gentry{Zahnarzt, -ärztin}{nom}{didansaz}{-e}{}

\gentry{Zaun}{nom}{çeper}{n}{}

\gentry{Zaza}{nom}{zaza, kird, dimilî, kirmanc}{-e}{}

\gentry{Zazakî}{nom}{zazakî, kirdkî, dimilî, kirmanckî}{m}{}

\gentry{zehn}{zahl}{des}{}{}

\gentry{Zeit}{nom}{\textsuperscript{1}dem, wext \textit{(n)} \textsuperscript{2}\textit{(Zeitraum, Epoche)} dewr}{n}{}

\gentry{Zeitraum}{nom}{dem}{}{}{}

\gentry{Zeitschrift}{nom}{kovare}{m}{}

\gentry{Zeitung}{nom}{rojname}{n}{}

\gentry{Zentimeter}{nom}{santîmetre, santîm}{n}{}

\gentry{zerbrechen}{verb}{$\rightarrow$ brechen}{}{}

\gentry{Zicklein}{nom}{bijêk, bicêk, tusk}{-e}{bizêk}

\gentry{Ziege}{nom}{bize}{m}{}

\gentry{ziehen}{verb}{\textsuperscript{1}antene \textasciitilde{} ontene \textsuperscript{2}(am Boden schleifend) kas kerdene}{}{}

\gentry{Ziel}{nom}{armanc}{n}{}

\gentry{Zimmer}{nom}{oda}{m}{}

\gentry{Zitrone}{nom}{lîmone}{m}{}

\gentry{Ziyaret \textit{(heilige Stätte der Aleviten)}}{nom}{zîyare \textasciitilde{} jîyare}{m}{}{}

\gentry{Zorn}{nom}{hêrs}{n}{}

\gentry{zu Ende bringen}{verb}{qedênayene}{}{}

\gentry{zu guter Letzt}{adv}{axirî}{}{}

\gentry{zu tun vermögen}{şîkîyayene}{verb}{besekerdene}{}{}

\gentry{zu viel; überschüssig}{adv}{zêde}{}{zêde, zîyade}

\gentry{zu viel; überschüssig}{adv}{jêde}{}{zêde, zîyade}

\gentry{Zucker (und Nüssen)}{nom}{helwa}{m}{}

\gentry{zudem}{adv}{ju kî \textasciitilde{} zu kî, ser o kî}{}{}

\gentry{Zufall}{nom}{tesaduf}{n}{}

\gentry{zugeben}{verb}{îlawe kerdene}{}{}

\gentry{zuhören}{verb}{gosdarî kerdene}{}{goşdarî kerdene}

\gentry{zur Winterzeit}{adv}{zimistanî}{}{}

\gentry{zurück}{adv}{peyser}{}{}

\gentry{zusammen}{adv}{pîya}{}{}

\gentry{Zusatz}{nom}{îlawe}{n}{}

\gentry{zusperren}{verb}{kilît kerdene}{}{}

\gentry{zusätzlich}{adv}{ser o kî}{}{}

\gentry{zwangsläufig}{adj}{mecbur}{-e}{}

\gentry{zwanzig}{zahl}{vîşt}{}{vîst}

\gentry{Zweck}{nom}{armanc}{n}{}

\gentry{zwei}{zahl}{di}{}{}

\gentry{Zwiebel}{nom}{pîyaz}{n}{}

\gentry{zwingend}{adj}{mecbur}{-e}{}

\gentry{zwischenstaatlich}{}{\textbf{$\rightarrow$ international}}{}{}

\gentry{Zyklus}{}{\textbf{$\rightarrow$ Zeitraum}}{}{}

\end{multicols}
%------------------------------------------------
\end{document}

